\chapter{Kodiranje matematičnih objektov z množicami}

Z množicami smo izrazili številne matematične objekte, na primer:

* preslikavo `f : A → B` lahko izrazimo kot funkcijsko relacijo med `A` in `B`, torej kot
  podmnožico `A × B`,

* kvocientna množica `A/R` je množica ekvivalenčnih razredov, ekvivalenčni razredi so spet
  množice,

Ali je možno vse matematične objekte predstaviti z množicami? Da!

\subsubsection{Urejeni pari}

Par `(x, y)` lahko predstavimo z množico `{{x}, {x,y}}`. Tako dobimo

    A × B := { {{x}, {x,y}} | x ∈ A, y ∈ B }

\subsubsection{Vsota}

Elemente vsote `A + B` kodiramo takole:

     in₁(x) = (x, 0) = {{x}, {x,∅}}
     in₂(x) = (x, 1) = {{x}, {x,{∅}}}

\subsubsection{Naravna števila}

Na množicah definiramo operacijo naslednik:

    x⁺ := x ∪ {x}

Naravna števila nato kodiramo tako, da za ničlo vzamemo `∅` in uporabljamo
operacijo naslednik:

    0 = ∅
    1 = ∅⁺ = {0} = {∅}
    2 = 1⁺ = {0, 1} = {∅, {∅}}
    3 = 2⁺ = {0, 1, 2} = {∅, {∅}, {∅, {∅}}}
    4 = 3⁺ = {0, 1, 2, 3} = {∅, {∅}, {∅, {∅}}, {∅, {∅}, {∅, {∅}}}}
    ...

Vidimo, da je vsako naravno število kar množica svojih predhodnikov.

\subsubsection{Cela števila}

Cela števila so kvocient `N × N`:

    Z = (N × N)/∼

kjer je

    (a,b) ∼ (c,d) ⇔ a + d = c + b.

Urejeni par `(a, b)` predstavlja razliko števil `a` in `b`.

\subsubsection{Racionalna števila}

Racionalna števila so kvocient:

    Q = (Z × {n ∈ N | n > 0})/≈

kjer je

    (a,m) ≈ (b,n) ⇔ a n = b m.

\subsubsection{Realna števila}

Realno število je Dedekindov rez, torej podmnožica `Q`.

In tako naprej. Ne pravimo, da je kodiranje vseh matematičnih objektov z množicami vedno
dobra ideja, vendar pa je dejstvo, da je to možno, pomembno spoznanje osnov matematike. Iz
njega na primer sledi tole: če je teorija množic neprotislovna, potem je neprotislovna
tudi vsa matematika, ki jo lahko kodiramo z množicami (torej več ali manj vsa običajna
matematika).

\section{Aksiomi teorije množic}

Zermelo-Fraenkelovi aksiomi teorije množic:

1. **Ekstenzionalnost:** množici `A` in `B`, ki imata iste elemente, sta enaki.

2. **Neurejeni par**: za vsak `x` in `y` je `{x, y}` množica, ki vsebuje natanko `x` in `y`:

        ∀ x y z . z ∈ {x, y} ⇔ z = x ∨ z = y

   Okrajšava: `{x} = {x, x}`.

3. **Unija:** za vsako množico `A` je `⋃ A` množica, ki vsebuje natanko vse
   elemente množic iz `A`

        ∀ A x . x ∈ ⋃ A ⇔ ∃ B ∈ A . x ∈ B

4. **Prazna množica:** množica `∅` nima elementa:

        ∀ x . x ∉ ∅

5. **Neskončna množica:** obstaja množica, ki vsebuje `∅` in je zaprta za operacijo naslednik
   (`x⁺ = x ∪ {x}`).

        ∃ A . ∅ ∈ A ∧ ∀ x ∈ A . x⁺ ∈ A

6. **Podmnožica:** za vsako množico `A` in formulo `φ` je `{x ∈ A | φ(x)}`
   množica, ki vsebuje natanko vse element iz `A`, ki zadoščajo `φ`:

        ∀ y . y ∈ {x ∈ A | φ(x)} ⇔ φ(y)

7. **Potenčna množica:** za vsako množico `A` je `P(A)` množica, ki vsebuje
   natanko vse njene podmnožice:

        ∀ S . S ∈ P(A) ⇔ S ⊆ A

8. Zamenjava: če je `A` množica in `f : A → Set` preslikava, je razred

        { y ∈ V | ∃ x ∈ A . y = f(x) }

   množica.

9. **Dobra osnovanost:** relacija `∈` je dobro osnovana.

10. **Aksiom izbire:** vsaka družina nepraznih množic ima funkcijo izbire


\section{Aksiom izbire}

**Definicija:** **Veriga** v delni urejenosti `(P, ≤)` je taka podmnožica `V ⊆
P`, ki je z `≤` linearno urejena, kar pomeni `∀ x y ∈ V . x ≤ y ∨ y ≤ x`.

Primeri:

* Če je `(P, ≤)` linearno urejena, je vsaka podmnožica veriga

* V `(P(Q), ⊆)` imamo neštevno verigo

        V = {S ∈ P(Q) | S je doljna množica}

  Množica `S ⊆ Q` je *doljna*, če velja `∀ x y ∈ Q . x ≤ y ∧ y ∈ Q ⇒ x ∈ Q`.

**Zornova lemma:** Če ima v delni urejenosti `(P, ≤)` vsaka veriga zgornjo mejo,
potem ima `P` maksimalni element.

Dokaz: dokaz se naslanja na aksiom izbire in Bourbaki-Wittov izrek o negibnih točkah (glej
spodaj). Naj bo `C` množica vseh verig v `P`. Uredimo jo z `⊆`. Na njej definiramo preslikavo
`f : C → C`, ki razširi verigo, če ni maksimalna, sicer je ne spremeni (tu uporabimo
izbiro):

* Če je `V ∈ C` maksimalna veriga v `P` (glede na `⊆`), definiramo `f(V) := V`.
* Če `V ∈ C` ni maksimalna veriga v `P`, potem obstaja tak `x ∈ P \ V`, da je `V
  ∪ {x}` spet veriga. V tem primeru *izberemo* tak `x` in definiramo `f(V) := V
  ∪ {x}`.

Po izreku Bourbaki-Witt ima `f` negibno vrednost `V ∈ C`. Ta `V` je maksimalna
veriga `V`, saj bi sicer veljalo, da je `V = f(V) = V ∪ {x}` za neki `x ∉ V`,
kar ni možno. Naj bo `m` zgornja meja za verigo `V`. Trdimo, da je `m`
maksimalni element v `P`: denimo, da velja `m ≤ y` za `m ∈ P`. Ker je `V ∪ {y}`
veriga, ki vsebuje maksimalno verigo `V`, sledi `V = V ∪ {y}`, od tod pa `y ∈ V`
ter `y ≤ m`. Torej je `m = y` in `m` je res maksimalni element. □

**Definicija:** Naj bo `(P, ≤)` delna ureditev. Preslikava `f : P → P` je **progresivna**, ko
velja `x ≤ f(x)` za vsak `x ∈ P`.

*Opomba:* progresivna preslikav ni nujno monotona (poiščite protiprimer!).

**Izrek (Bourbaki-Witt):** Naj bo `(P, ≤)` neprazna delna ureditev, v kateri ima
vsaka veriga zgornjo mejo in `f : P → P` progresivna preslikava. Tedaj ima `f`
negibno točko: to je tak `x ∈ P`, da velja `f(x) = x`.

Dokaz: opuščen.

**Izrek:** V teoriji množic *brez* aksioma izbire so naslednje izjave ekvivalentne:

1. Aksiom izbire
2. Zornova lema
3. Princip dobre urejenosti: vsaka množica ima dobro ureditev

Dokaz:

(1 ⇒ 2) Glej Zornovo lemo.

(2 ⇒ 3) Skica dokaza: naj bo `A` poljubna množica, ki jo želimo dobro urediti.

Definirajmo *delne* dobre ureditev množice `A`: to so pari `(B,R)`, kjer je `B ⊆ A`
in `R ⊆ B × B` dobra ureditev na `B`. Za delni dobri ureditvi `(B,R)` in
`(C,Q)` pravimo, da je `(C,Q)` *razširitev* `(B,R)`, kadar velja `B ⊆ C`, `R ⊆ Q` in
še, da je `B` začetni segment v `C`, kar pomeni:

     ∀ x y ∈ C: x Q y ∧ y ∈ B ⇒ x ∈ B.

Kadar je `(C,Q)` razširitev `(B,R)`, pišemo `(B,R) ≼ (C,Q)`. Naj bo `P` množica vseh delnih
dobrih ureditev množice `A`,

    P = { (B, R) | B ⊆ A in R ⊆ B × B in R je dobra ureditev B },

urejena z relacijo `≼`. Očitno je `≼` delna ureditev. Trdimo, da imajo verige v
`P` zgornje meje glede na `≼`: če je `V ⊆ P` veriga dobro urejenih podmnožic
`A`, je njena zgornja meja `(D,S)` kar unija po komponentah:

     D := ⋃ {B | (B, R) ∈ V}
     S := ⋃ {R | (B, R) ∈ V}

Preverimo, da velja `(D,S) ∈ P`. Očitno je `(D,S)` stroga linearna ureditev
(vaja). Denimo, da bi v `(D,S)` imeli neskončno padajočo verigo

     ... S x₃ S x₂ S x₁ S x₀.

Obstaja `(B,R) ∈ V`, da je `x₀ ∈ B`. Potem bi bila `x₀, x₁, x₂, x₃, ...`
padajoča veriga v `(B,R)`, kar ni možno, saj je `(B,R)` dobro urejena. Res, ker
je `xᵢ ∈ V`, obstaja `(C,Q)`, da je `xᵢ ∈ C`. Če velja `(B,R) ≼ (C,Q)`, potem
`xᵢ ∈ B` po definicijo `≼`. Če velja `(C,Q) ≼ (B,R)`, potem `xᵢ ∈ B`, ker velja
`C ⊆ B`. Torej je `(D,S)` res delna ureditev `P`.

Preverimo še, da velja `(B,R) ≼ (D,S)` za vsak `(B,R) ∈ V`. Denimo, da je `y ∈ D`,
`x ∈ B` in `y S x`. Obstaja `(C,Q) ∈ V`, da je `y ∈ C`. Če velja `(C,Q) ≼ (B,R)`,
potem `y ∈ C ⊆ B`. Če pa velja `(B,R) ≼ (C,Q)`, potem je `y ∈ B` po definiciji `≼`.

Po Zornovi lemi obstaja maksimalni element `(B,R)` v `P`. Trdimo, da je `B = A`. Če bi namreč
obatajal `x ∈ B \ A`, bi lahko razširili `(B,R)` na večjo dobro ureditev tako, da bi dodali `x`
na konec `B`:

    (B ∪ {x}, R')

    y R' z ⇔ z = x ∧ (y,z) ∈ R

To ni možno, ker je `(B,R)` maksimalna delna ureditev. Torej je res `A = B` in
našli so dobro ureditev `A`.

(3 ⇒ 1) Naj bo `A : I → Set` družina nepraznih množic. Naj bo `≺` dobra ureditev
na uniji `⋃ A`. Ker so vse množice `Aᵢ` neprazne, ima vsaka od njih prvi element
glede na `≺`. Torej lahko definiramo funkcijo izbire `f` s predpisom

    f(i) = prvi element Aᵢ. □

**Izrek:** Vsak vektorski prostor ima vektorsko bazo.

Dokaz: Naj bo `L` vektorski prostor. Definiramo množico

    P = { B ⊆ L | B je linearno neodvisna }.

Množico `P` delno uredimo z relacijo `⊆`. Trdimo, da imajo verige v `P` zgornje
meje: zgornja meja verige `V ⊆ P`, je kar njena unija `⋃_(B ∈ V) B`. Seveda je
treba preveriti, da je unija verige linearno neodvisnih množic spet linearno
neodvisna (vaja). Po Zornovi lemi obstaja maksimalni element v `P`, torej
maksimalna linearno neodvisna množica `B` v `L`. To pa je seveda vektorksa baza
za `L`. □

