\section{Kumulativna hierahija}

Če lahko vse matematične objekte kodiramo z množicami, potem lahko na razred
vseh množic `Set` gledamo kot na celotni matematični svet. Razred `Set` ima
zanimivo strukturo, ki ji pravimo **kumulativna hierarhija**. Namreč, s pomočjo
aksiomov, ki jih bomo spoznali kasneje, lahko tvorimo vse množice iz `∅` z
oparacijama potenčna množica in unija. Postopek je **transfiniten**, kar pomeni,
da se nikoli ne konča in da po svoje številu presega moč vsake množice.

    V₀ = ∅
    V₁ = P(V₀) = {∅}
    V₂ = P(V₁} = {∅, {∅}}
    V₃ = P(V₂) = {∅, {∅}, {{∅}}, {∅, {∅}}}
    ...
    V_ω = ⋃ {Vᵢ | i < ω}
    V_(ω+1) = P(V_ω)
    V_(ω+2) = P(V_(ω+1))
    ...
    V_(ω + ω) = ⋃ {Vᵢ | i < ω + ω)}
    ...

Stopnje konstrukcija indeksiramo s t.i. **ordinalnimi števili**, ki jih bomo spoznali.

\section{Ordinalna števila}

**Definicija:** Množica $x$ je **tranzitivna**, če za vsak `y ∈ x` velja `y ⊆ x`.

Izraz *tranzitivna* je smiselen, ker govori o tranzitivnosti relacije `∈`, saj lahko pogoj `y ∈ x ⇒ y ⊆ x` zapišemo kot `z ∈ y ∧ y ∈ x ⇒ z ∈ x`.

Primeri tranzitivnih množic: `∅`, `{∅}`, `{∅, {∅}}`, `{{{∅}}, {∅}, ∅}`

**Definicija:** Množica je **hereditarno tranzitivna**, če so vsi njeni elementi tranzitivne množice.

(V splošnem se izraz "hereditarno" uporablja, kadar se lastnost nanaša na elemente, pomdnožice, ali podstrukture, se pravi na "potomce".)

**Definicija:** **Ordinalno število** je tranzitivna in hereditarno tranzitivna množica. Razred vseh ordinalnih števil označimo z `On`.

Definirajmo relacijo *naslednik* na množicah: `x^+ = x ∪ {x}`.

Preverimo lahko tole:

1. `∅ ∈ On`
2. če je `α ∈ On`, potem je tudi `α^+ ∈ On`.
3. `On` je zaprt za unije: če je `S ⊆ On` množica, potem je `U S ∈ On`.

Sedaj lahko gradimo `On` iterativno:

* `0 = ∅`
* z opreacijo naslednik dobimo naravna števila `n = {0, ..., n-1}`
* unija vseh naravnih števil je `ω`.
* z naslednik gradimo `ω`, `ω + 1`, `ω + 2`, ...
* unija teh je `ω + ω`
* in tako naprej

\section{Kardinalna števila}

**Definicija:** Ordinalno število `α` je **kardinalno**, če za vsak `β < α`
velja, da ne obstaja injektivna preslikava `α → β`.

\subsection{Zakon trihotomije}

V tem razadelku podamo še oris dokaza, da je aksiom izbire ekvivalenten zakonu trihotomije.

**Definicija:** Naj bo `(P, <)` dobra urejenost. Podmnožica `I ⊆ P` je **začetni
segment**, če je doljna množica: iz `x < y` in `y ∈ I` sledi `x ∈ I`.

**Definicija:** Naj bosta `(P, <_P)` in `(Q, <_Q)` dobri urejenosti. Pravimo, da
je preslikava `e : P → Q` **vložitev**, kadar velja:

1. `e` je strogo monotona in
2. slika `e_(P)` je začetni segment v `Q`.

Vložitev je injektivna preslikava.

**Lemma 1:** Naj bosta `(P, <_P)` in `(Q, <_Q)` dobri urejenosti. Če obstaja
injektivna preslikava `P → Q`, potem obstaja tudi vložitev `P → Q`.

Dokaz: opuščen.

**Lemma 2:** Naj bosta `(P, <_P)` in `(Q, <_Q)` dobri urejenosti. Tedaj bodisi
obstaja vložitev `P → Q` ali vložitev `Q → P`.

Dokaz: opuščen.

**Izrek:** Aksiom izbire je ekvivalenten zakonu trihotomije: za vse množice `X` in `Y` velja
`|X| ≤ |Y|` ali `|Y| ≤ |X|`.

Dokaz:

Najprej predpostavimo, da velja aksiom izbire. Naj bosta `X` in `Y` množici. Ker
velja aksiom izbire, lahko `X` in `Y` dobro uredimo, denimo z relacijama `<_X`
in `<_Y`. Iz zgornje leme sledi, da obstaja vložitev `X → Y` ali `Y → X`.
Ker so vložitve injektivne, torej velja `|X| ≤ |Y|` ali `|Y| ≤ |X|`.

Predpostavimo zdaj, da za vse množice `X` in `Y` velja `|X| ≤ |Y|` ali `|Y| ≤
|X|`. Dokazali bomo, da lahko vsako množico dobro uredimo, iz česar sledi aksiom izbire.
