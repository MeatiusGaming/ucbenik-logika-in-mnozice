\chapter{Aksiomatska teorija množic}
\label{chap:zfc}

\section{Osnovni aksiomi teorije množic}
\label{sec:osnovni-aksiomi-mnozic}


\section{Aksiom izbire}
\label{sec:aksiom-izbire}


\section{Moč množic in kadrinalna števila}
\label{sec:kardinalna-stevila}

%Za naravno število $n \in \NN$ naj bo
%% 
%\begin{equation*}
%  [n] = \set{k \in \NN \such k < n} = \set{0, 1, \ldots, n-1}
%\end{equation*}
%% 
%množica prvih $n$ naravnih števil.
%% 
%Na predavanjih smo povedali, da je moč množice $A$ enaka $|A| = n$, če
%velja $A \iso [n]$. Obravnavali smo naslednje enačbe:
%%
%\begin{align*}
%  |\emptyset| &= 0 \\
%  |\set{\star}| &= 1 \\
%  |A \times B| &= |A| \cdot |B| \\
%  |A + B| &= |A| + |B| \\
%  |B^A| &= |B|^{|A|} \\
%  |\pow{A}| &= 2^{|A|} \\
%  |A \cup B| + |A \cap B| &= |A| + |B|
%\end{align*}
%%
%Obravnavali smo tudi pravilo vključitve-izključitve za~$k$ množic
%%
%\begin{align*}
%  |A_1 \cup \cdots A_k| &= |A_1| + \cdots + |A_k| \\
%  &- (|A_1 \cap A_2| + \cdots + |A_{k-1} \cap A_k|) \\
%  &+ (|A_1 \cap A_2 \cap A_3| + \cdots + |A_{k-2} \cap A_{k-1} \cap A_k|) \\
%  & \cdots
%\end{align*}
%%
%Prešteli smo, koliko je injekcij $[m] \to [n]$ in koliko bijekcij $[m]
%\to [m]$.

%Povedali smo, da ima množica $\set{1, \ldots, n}$ natanko $\ceil{n/k}$
%večkratnikov števila $k$. S pravilom vključitve-izklučitve smo
%prešteli, koliko je števil med $1$ in $2009$, ki so večkratniki $5$
%ali večkratniki $7$ ali večkratniki $11$.



%Definirali smo pojem kardinalnega števila, neformalno kot ``količina''
%$|A|$, ki pomeni število elementov množice~$A$. Definirali smo $|A|
%\leq |B|$ kot ``obstaja injekcija $A \to B$''. Dokazali smo, da
%obstaja injekcija $A \to B$ natanko tedaj, ko je $A = \emptyset$ ali
%obstaja surjekcija $B \to A$.

%Povedali smo Cantor-Schröder-Bernsteinov izrek, ki pravi, da obstaja
%bijekcija $A \to B$, če obstajata injekciji $A \to B$ in $B \to A$.
%Izreka nismo dokazali.

%Dokazali smo Cantorjev izrek $|A| < |\pow{A}|$ (obstaja injekcija $A
%\to \pow{A}$ in ne obstaja injekcija $\pow{A} \to A$).

%Definirali smo naslednja kardinalna števila: $\aleph_0$ je moč množice
%naravnih števil, povedali smo, da je to najmanjše neskončno kardinalno
%število. Kontinuum $c$ je moč množice $\pow{\NN}$.

%Definirali smo števno množico kot tako množico, katere moč je manj ali
%enaka $\aleph_0$. Ugotovili smo, da je to ekvivalentno temu, da je
%množica prazna ali pa obstaja surjekcija iz $\NN$ nanjo.

%Povedali smo, da so kardinalna števila dobro urejena in na tablo
%narisali zaporedje
%%
%\begin{equation*}
%  0, 1, 2, \ldots, \aleph_0, \aleph_1, \aleph_2, \ldots,
%  \aleph_\omega, \aleph_{\omega+1}, \ldots
%\end{equation*}
%%
%Vprašali smo se, kateri $\aleph$ je kontinuum $c$. Omenil sem, da to
%vprašanje nima odgovora.

%Definirali smo poljubno velike kartezične produkte $\prod_{i \in I}
%A_i = \set{f : I \to \bigcup_i A_i \such \xall{i}{I}{f(i) \in A_i}}$
%in koprodukt $\coprod_{i \in I} A_i = \sum_{i \in I} A_i = \set{(i,x)
%  \such i \in I \land x \in A_i}$.

%Definirali smo kardinalno aritmetiko. Če je $\kappa = |A|$ in $\lambda
%= |B|$, je
%%
%\begin{align*}
%  \kappa + \lambda &= |A + B| \\
%  \kappa \cdot \lambda &= |A \times B| \\
%  \kappa^\lambda &= |A^B|
%\end{align*}
%%
%Za družino množic $(A_i)_{i \in I}$, kjer je $\kappa_i = A_i$, smo
%definirali
%%
%\begin{align*}
%  \prod_{i \in I} \kappa_i &= |\prod_{i \in I} A_i| \\
%  \sum_{i \in I} \kappa_i &= |\sum_{i \in I} A_i|.
%\end{align*}
%%
%Razpravljali smo o tem, zakaj odštevanje kardinalnih števil ni dobro
%definirano.


\section{Ordinalna števila}
\label{sec:ordinalna-stevila}


%%% Local Variables: 
%%% mode: latex
%%% TeX-master: "lmn"
%%% End: 
