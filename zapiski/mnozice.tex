\chapter{Množice}
\label{chap:mnozice}

V drugem delu predmeta bomo spoznali osnove teorije množic. Najprej pa
se bomo posvetili še naravnim številom in Peanovim aksiomom.

%%%%%%%%%%%%%%%%%%%%%%%%%%%%%%%%%%%%%%%%%%%%%%%%%%%%%%%%%%%%%%%%%%%%%%
\section{Naravna števila}
\label{sec:naravna-stevila}

Naravna števila
%
\begin{equation*}
  0, 1, 2, 3, 4, 5, 6, 7, 8, 9, 10, 11, 12, \ldots
\end{equation*}
%
vsi že dobro poznamo iz osnovne šole.\footnote{V teh zapiskih in v
  logiki nasploh vzamemo za prvo naravno število $0$. V osnovni šoli
  in drugje pa ponavadi za prvo naravno število jemljemo~$1$.} V tem
razdelku pokažimo, kako uvedemo naravna števila kot formalno teorijo v
logiki. V splošnem \emph{formalna teorija} opisuje neko matematično
strukturo ali družino struktur in je podana z osnovnimi simboli
(konstantami in operacijami), aksiomi in pravili sklepanja.

Teorijo naravnih števil, ki jo imenujemo tudi \emph{Peanova
  aritmetika}, sestoji iz konstante $0$, enočlene operacije
naslednik~$\suc{n}$ ter dvočlenih operacij seštevanje~$m + n$ in
množenje~$m \cdot n$. Množenje ia prednost pred seštevanjem, se pravi,
da je $k + m \cdot n = k + (m \cdot n)$ in ne $(k + m) \cdot n$.
Aksiomi in pravila sklepanja se glasijo:
%
\begin{enumerate}
  \item Nič ni naslednik:
  %
  \begin{equation*}
    \inferrule{ }{\suc{n} \neq 0}    
  \end{equation*}
  %
  \item Če sta naslednika enaka, sta števili enaki:
  %
  \begin{equation*}
    \inferrule{\suc{m} = \suc{n}}{m = n}
  \end{equation*}
  %
  \item Pravili za seštevanje:
  %
  \begin{mathpar}
    \inferrule{ }{0 + n = n}
    \and
    \inferrule{ }{\suc{m} + n = (m + n\suc{)}}    
  \end{mathpar}
  %
  \item Pravili za množenje:
  %
  \begin{mathpar}
    \inferrule{ }{0 \cdot n = 0}
    \and
    \inferrule{ }{\suc{m} \cdot n = m \cdot n + n}
  \end{mathpar}
  %
  \item Princip indukcije:
  %
  \begin{equation*}        
    \inferrule{\phi(0) \\ \xall{m}{\NN}{\phi(m) \lthen \phi(\suc{m})}}{\phi(n)}
  \end{equation*}
\end{enumerate}
%
Pri običajnem računanju z naravnimi števili uporabljamo vse znanje, ki
smo ga pridobili v šoli. Ko pa obravnavamo naravna števila kot
formalno teorijo, smemo uporabljati \emph{samo} konstante in simbole,
ki jih vpeljemo v teoriji, in se sklicevati \emph{samo} na Peanove
aksiome. Denimo, ker teorija ne vpelje simbolov $1$ in $2$, ju ne
smemo uporabljati, razen če ju prej definiramo kot okrajšavi za
$\suc{0}$ in $\suc{(\suc{0})}$. Prav tako ne smemo omenjati odštevanja
števil, ker to ni ena od operacij $+$ in $\cdot$, ne smemo govoriti o
parnosti števil, ne da bi prej ta pojem definirali, itn. Tudi osnovne
lastnosti seštevanja in množenja, kot sta komutativnost in
asociativnost, ne smemo uporabiti, če ju prej ne dokažemo. Matematiki
so seveda preverili, da vse običajne lastnosti števil dejansko sledijo
iz Peanovih aksiomov.

Glavno orodje pri dokazovanju lastnosti naravnih števil je princip
indukcije. V besedilu ga uporabimo takole:
%
\begin{quote}
  \em
  %
  Dokazujemo $\phi(n)$ z indukcijo po~$n$:
  %
  \begin{enumerate}
    \item Baza indukcije: (Dokaz, da velja $\phi(0)$.)
    \item Indukcijski korak: denimo, da za naravno število $m$ velja
      $\phi(m)$. (Dokaz, da velja $\phi(\suc{m})$.)
  \end{enumerate}
\end{quote}
%
Za zgled dokažimo, da je seštevanje komutativno. To naredimo v nekaj
korakih.

\begin{izjava}
  \label{izjava:peano-n-plus-0}
  Za vsako naravno število $m$ velja $m + 0 = m$.
\end{izjava}

\begin{dokaz}
  Dokazujemo z indukcijo. Baza indukcije: $0 + 0 = 0$ po enem od
  Peanovih aksiomov.
  %
  Indukcijski korak: denimo, da za naravno število $k$ velja $k + 0 =
  k$. Tedaj je $\suc{k} + 0 = \suc{(k + 0)} = \suc{k}$, kjer smo v
  prvem koraku uporabili enega od Peanovih aksiomov in v drugem
  indukcijsko predpostavko.
\end{dokaz}


\begin{izjava}
  \label{izjava:peano-m-plus-suc-n}
  Za vsaki naravni števili $m$ in $n$ velja $m + \suc{n} = \suc{(m + n)}$.
\end{izjava}

\begin{dokaz}
  Izjavo dokažemo z indukcijo po $m$.
  Baza indukcije: $0 + \suc{n} = \suc{n} = \suc{(0 + n)}$.
  %
  Indkucijski korak: denimo, da za naravno število $k$ velja $k +
  \suc{n} = \suc{(k + n)}$. Tedaj je
  %
  \begin{equation*}
    \suc{k} + \suc{n} = 
    \suc{(k + \suc{n})} =
    \suc{{\suc{(k + n)}}} =
    \suc{(\suc{k} + n)}.
  \end{equation*}
  %
\end{dokaz}

\begin{izjava}
  Za vsaki naravni števili $m$ in $n$ velja $m + n = n + m$.
\end{izjava}

\begin{dokaz}
  Izjavo dokažemo z indukcijo po $m$.
  Baza indukcije: $0 + n = n = n + 0$, kjer smo v prvem koraku uporabili Peanov aksiom in v drugem Izjavo~\ref{izjava:peano-n-plus-0}.
  %
  Indukcijski korak: denimo, da za naravno število $k$ velja $k + n = n + k$. Tedaj je
  %
  \begin{equation*}
    \suc{k} + n =
    \suc{(k + n)} =
    \suc{(n + k)} =
    n + \suc{k}.
  \end{equation*}
  %
  V prvem koraku smo uporabili Peanov aksiom, v drugem indukcijsko predpostavko, v tretjem pa Izjavo~\ref{izjava:peano-m-plus-suc-n}.
\end{dokaz}

%%%%%%%%%%%%%%%%%%%%%%%%%%%%%%%%%%%%%%%%%%%%%%%%%%%%%%%%%%%%%%%%%%%%%%
\section{Množice}
\label{sec:naivne-mnozice}

Množice so osnovni gradniki matematičnih objektov in struktur. V tem razdelku obravnavamo množice \emph{naivno}, se pravi s pomočjo neformalnih razlag. Samo formalno teorijo množic in aksiome bomo obravnavali v razdelku~\ref{sec:zfc}.

Množico si predstavljamo kot skupek ali zbirko poljubnih objektov, jim pravimo \emph{elementi} množice. Dejstvo, da je $x$ element množice $A$ zapišemo $x \in A$. Če $x$ ni element $S$, pišemo $x \not\in S$ kot okrajšavo za $\lnot (x \in S)$. Množica ni odvisna od tega, kako jo opišemo ali skonstruiramo, ampak le od tega, kateri elementi so v njej. To dejstvo izraža \emph{aksiomom o ekstenzionalnosti}, ki pravi, da sta množici $A$ in $B$ enaki natanko tedaj, ko vsebujeta iste elemente, kar zapišemo s formulo kot
%
\begin{equation*}
  A = B \liff \uall{x}{x \in A \liff x \in B}.
\end{equation*}
%
Množice gradimo iz osnovnih množic s pomočjo operacij.

\subsection{Osnovne množice}
\label{sec:osnovne-mnozice}

Najpreprostejša osnovna množica je \emph{prazna množica}, ki jo označimo z $\emptyset$. Dejstvo, da prazna množica ne vsebuje nobenih elementov izrazimo z aksiomom o prazni množici,
%
\begin{equation*}
  \xuall{x}{x \not\in \emptyset}.
\end{equation*}
%
V zvezi s prazno množico omenimo, da za vsako izjavo $\phi$ velja
%
\begin{equation*}
  \xall{x}{\emptyset}{\phi},
\end{equation*}
%
kar dokažemo takole: naj bo $x \in \emptyset$ poljuben. Ker velja $x \not\in \emptyset$, je to protislovje, od koder smemo sklepati $\phi$. Podobno za vsako izjavo $\phi$ velja
%
\begin{equation*}
  \lnot\xsome{x}{\emptyset}{\phi}.
\end{equation*}

\begin{naloga}
  Za katere množice $S$ velja $(\xall{x}{S}{\phi(x)}) \lthen   \xsome{x}{S}{\phi(x)}$?
\end{naloga}

Za osnovno množico vzamemo tudi množico naravnih števil~$\NN$, ki smo jo že spoznali v razdelku~\ref{sec:naravna-stevila}.

\subsection{Konstrukcije množic}
\label{sec:konstrukcije-mnozic}

Iz osnovnih množic lahko konstruiramo nove s pomočjo naslednjih operacij.

\subsubsection{Končne množice}
\label{sec:koncne-mnozice}

Naj bodo $a_1, \ldots, a_n$ poljubni objekti. Tedaj lahko tvorimo množico
%
\begin{equation*}
  \set{a_1, a_2, \ldots, a_n}
\end{equation*}
%
ki sestoji iz naštetih elementov, to je
%
\begin{equation*}
  \uall{x}{x \in \set{a_1, \ldots, a_n} \liff
    x = a_1 \lor \cdots x = a_n}.
\end{equation*}
%
Poseben primer take množice je \emph{enojec} $\set{a}$, za katerega velja
%
\begin{equation*}
  \uall{x}{x \in \set{a} \liff x = a}.
\end{equation*}


\begin{naloga}
  Ali je $\set{a, b} = \set{b, a}$? Ali je $\set{a, a, b} = \set{a, b}$? Uporabi aksiom o ekstenzionalnosti.
\end{naloga}

\subsubsection{Unija in presek}
\label{sec:unija-presek}

Družina množic.

Unije, preseki.

\subsubsection{Podmnožica}
\label{sec:podmnozica}

Podmnožica (separacija).

\subsubsection{Potenčna množica}
\label{sec:potencna-mnozica}

Potenčna množica.

\subsubsection{Kartezični produkt}
\label{sec:kartezicni-produkt}

Kartezični produkt. Produkt s prazno.

\subsubsection{Eksponentna množica}
\label{sec:eksponentna-mnozica}

Eksponentna množica. Eksponent s prazno.

\subsubsection{Vsota}
\label{sec:vsota-mnozic}

Disjunktna unija.

\subsubsection{Razlika in komplement}
\label{sec:vsota-mnozic}


\section{Funkcije}
\label{sec:funkcije}


Funkcija, neformalna definicija.

Kompozitum, asociativnost kompozituma. Identiteta.

Inverz funkcije. Inverz je enoličen, če obstaja.

Slika in inverzna slika.

Kdaj obstaja inverz? Surjektivna, injektivna, bijektivna funckija.

Epi in mono.

Sekcija in retrakcija.

Sekcija je mono, retrakcija je epi.

Standardne bijekcije za vsoto, produkt in eksponent.

\section{Relacije}
\label{sec:relacije}

Definicija relacije.

Nasprotna relacija. Komplement, unija.

\subsection{Funkcijske relacije}
\label{sub:funkcijske_relacije}


\subsection{Ekvivalenčne relacije}
\label{sub:ekvivalencne_relacije}


%Definirali smo pojem ekvivalenčne relacije in kvocienta množice po
%ekvivalenčni relaciji. Pokazali smo razne primere. Dokazali smo, da
%smemo definirati $f : A/{\sim} \to B$ na kvocientu tako, da definiramo
%$g : A \to B$, ki je skladen s~$\sim$.




Defincije. Primeri.

Ekvivalenčna relacija, generirana z relacijo.

Faktorska množica. Kako definiramo preslikavo na faktorski množici.

Kanonični razcep funkcije.

\subsection{Delna ureditev}
\label{sub:delna_ureditev}

Definicija delne ureditve. Primeri.

Linearna ureditev. Stroga linearna ureditev. Veriga.

Zgornja meja, spodna meja, infimum, supremum, maksimum, minimum, minimalni element, maksimalni element.



%%% Local Variables: 
%%% mode: latex
%%% TeX-master: "lmn"
%%% End: 
