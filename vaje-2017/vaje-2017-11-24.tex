\documentclass{article}

\usepackage[utf8]{inputenc}
\usepackage{amsmath}
\usepackage{amssymb}
\usepackage{amsthm}
\usepackage{ifthen}
\usepackage{xparse}
%\usepackage{tikz}
%\usetikzlibrary{decorations.fractals}

\newcommand{\sizedescriptor}[2]
{
\ifthenelse{\equal{#1}{0}}{}{
\ifthenelse{\equal{#1}{1}}{\big}{
\ifthenelse{\equal{#1}{2}}{\Big}{
\ifthenelse{\equal{#1}{3}}{\bigg}{
\ifthenelse{\equal{#1}{4}}{\Bigg}{
#2}}}}}
}

\NewDocumentCommand{\set}
{O{auto} m G{\empty}}
{\sizedescriptor{#1}{\left}\{ {#2} \ifthenelse{\equal{#3}{}}{}{ \; \sizedescriptor{#1}{\middle}| \; {#3}} \sizedescriptor{#1}{\right}\}}

\newcommand{\all}[1]{\forall #1 .\,}
\newcommand{\some}[1]{\exists #1 .\,}
\newcommand{\exactlyone}[1]{\exists{!} #1 .\,}
\newcommand{\lam}[1]{\lambda #1 .\,}
\newcommand{\that}[1]{\iota #1 .\,}

\usepackage[slovene]{babel}
\newcommand{\lthen}{\Rightarrow}
\newcommand{\two}{\mathbf{2}}
\newcommand{\true}{\top}
\newcommand{\false}{\bot}

\newcommand{\NN}{\mathbb{N}}
\newcommand{\ZZ}{\mathbb{Z}}
\newcommand{\QQ}{\mathbb{Q}}
\newcommand{\RR}{\mathbb{R}}

%%%%%%%%%%%%%%%%%%%%%%%%%%%%%%%%%%%%%%%%%%%%%%%%%%%%%%%%%%%%%%%%%%%%%%%%%%%%%%%%%%%%%%%%%%%%%%%%%%%%%%%%%%%%%%%%%%%%%%%
%%%  Commands
%%%%%%%%%%%%%%%%%%%%%%%%%%%%%%%%%%%%%%%%%%%%%%%%%%%%%%%%%%%%%%%%%%%%%%%%%%%%%%%%%%%%%%%%%%%%%%%%%%%%%%%%%%%%%%%%%%%%%%


%%%%%%%%%%%%%%%%%%%%%%%%%%%%%%%%%%%%%%%%%%%%%%%%%%%%%%%%%%%%%
%%%  Theorems etc.
%%%%%%%%%%%%%%%%%%%%%%%%%%%%%%%%%%%%%%%%%%%%%%%%%%%%%%%%%%%%%
{
\theoremstyle{theorem}
\newtheorem{izrek}{Izrek}[chapter]
\newtheorem{lema}[izrek]{Lema}
\newtheorem{trditev}[izrek]{Trditev}
\newtheorem{posledica}[izrek]{Posledica}
\newtheorem{pravilo}[izrek]{Pravilo}
}

{
\theoremstyle{definition}
\newtheorem{definicija}[izrek]{Definicija}
\newtheorem{opomba}[izrek]{Opomba}
\newtheorem{primer}[izrek]{Primer}
\newtheorem{zgled}[izrek]{Zgled}
\newtheorem{naloga}[izrek]{Naloga}
}


%%%%%%  Proofs
%%%%%%%%%%%%%%%%%%%%%%%%%%%%%%%%%%%%%%%%%%%%%%%%%%%%%%%%%%%%%
% Za dokaze uporabimo amsmath proof, sicer ne deluje \qedhere.


%%%%%%  Auxiliary
%%%%%%%%%%%%%%%%%%%%%%%%%%%%%%%%%%%%%%%%%%%%%%%%%%%%%%%%%%%%%
\newcommand{\sizedescriptor}[2]
{
\ifthenelse{\equal{#1}{0}}{}{
\ifthenelse{\equal{#1}{1}}{\big}{
\ifthenelse{\equal{#1}{2}}{\Big}{
\ifthenelse{\equal{#1}{3}}{\bigg}{
\ifthenelse{\equal{#1}{4}}{\Bigg}{
#2}}}}}
}

\newcommand{\someref}{{\small\textcolor{blue}{[\textbf{ref.}]}}}
\newcommand{\intermission}{\bigskip\medskip}
\newcommand{\ltc}[1]{$\backslash$\texttt{#1}}  % LaTeX command
\newcommand{\nls}[1]{``\textit{#1}''}  % sentence in a natural language

%%%%%%  Logical Quantifiers, λ- and ι-Expressions
%%%%%%%%%%%%%%%%%%%%%%%%%%%%%%%%%%%%%%%%%%%%%%%%%%%%%%%%%%%%%

\newcommand{\all}[1]{\forall #1 .\,}
\newcommand{\some}[1]{\exists #1 .\,}
\newcommand{\exactlyone}[1]{\exists{!} #1 .\,}
\newcommand{\lam}[1]{\lambda #1 .\,}
\newcommand{\that}[1]{\iota #1 .\,}

%%%%%%  Logic
%%%%%%%%%%%%%%%%%%%%%%%%%%%%%%%%%%%%%%%%%%%%%%%%%%%%%%%%%%%%%
\newcommand{\tvs}{\Omega}  % set of truth values
\newcommand{\true}{\top}  % truth
\newcommand{\false}{\bot}  % falsehood
\newcommand{\etrue}{\boldsymbol{\top}}  % emphasized truth
\newcommand{\efalse}{\boldsymbol{\bot}}  % emphasized falsehood
\newcommand{\impl}{\Rightarrow}  % implication sign
\newcommand{\revimpl}{\Leftarrow}  % reverse implication sign
\newcommand{\lequ}{\Leftrightarrow}  % equivalence sign
\newcommand{\xor}{\mathbin{\veebar}}  % exclusive disjunction sign
\newcommand{\shf}{\mathbin{\uparrow}}  % Sheffer connective
\newcommand{\luk}{\mathbin{\downarrow}}  % Łukasiewicz connective


%%%%%%  Sets
%%%%%%%%%%%%%%%%%%%%%%%%%%%%%%%%%%%%%%%%%%%%%%%%%%%%%%%%%%%%%
%  \set{1, 2, 3}         ->  {1, 2, 3}
%  \set{a \in X}{a < 1}  ->  {a ∈ X | a < 1}
\NewDocumentCommand{\set}
{O{auto} m G{\empty}}
{\sizedescriptor{#1}{\left}\{ {#2} \ifthenelse{\equal{#3}{}}{}{ \; \sizedescriptor{#1}{\middle}| \; {#3}} \sizedescriptor{#1}{\right}\}}
%\newcommand{\vsubset}{\Mapstochar\cap}
%\newcommand{\finseq}[1]{{#1}^*}
\newcommand{\pst}{\mathcal{P}}
\renewcommand{\complement}[1]{{#1}^C}


%%%%%%  Number Sets, Intervals
%%%%%%%%%%%%%%%%%%%%%%%%%%%%%%%%%%%%%%%%%%%%%%%%%%%%%%%%%%%%%
\newcommand{\NN}{\mathbb{N}}
\newcommand{\ZZ}{\mathbb{Z}}
\newcommand{\QQ}{\mathbb{Q}}
\newcommand{\RR}{\mathbb{R}}
\newcommand{\CC}{\mathbb{C}}
\newcommand{\intoo}[3][\RR]{{#1}_{(#2, #3)}}
\newcommand{\intcc}[3][\RR]{{#1}_{[#2, #3]}}
\newcommand{\intoc}[3][\RR]{{#1}_{(#2, #3]}}
\newcommand{\intco}[3][\RR]{{#1}_{[#2, #3)}}


%%%%%%  Maps and Relations
%%%%%%%%%%%%%%%%%%%%%%%%%%%%%%%%%%%%%%%%%%%%%%%%%%%%%%%%%%%%%
\newcommand{\id}[1][]{\mathrm{id}_{#1}}  % identity map
\newcommand{\argbox}{{\;\!\fbox{\phantom{M}}\;\!}}  % box for a function argument
\newcommand{\konst}[1]{\mathrm{k}_{#1}} % constant map
\newcommand{\rstr}[1]{\left.{#1}\right|}  % map restriction
\newcommand{\im}{\mathrm{im}}  % map image
\newcommand{\parto}{\mathrel{\rightharpoonup}}  % partial mapping sign
\NewDocumentCommand{\rel}
{O{\empty} O{\empty}}
{\ifthenelse{\equal{#1}{}}{\mathscr{R}}{{#1} \mathrel{\mathscr{R}} {#2}}}  % a relation
\NewDocumentCommand{\srel}
{O{\empty} O{\empty}}
{\ifthenelse{\equal{#1}{}}{\mathscr{S}}{{#1} \mathrel{\mathscr{S}} {#2}}}  % a second relation
\newcommand{\dom}{\mathrm{dom}}  % domain
\newcommand{\cod}{\mathrm{cod}}  % codomain
\newcommand{\dd}[1]{D_{#1}}  % domain of definition
\newcommand{\rn}[1]{Z_{#1}}  % range
\newcommand{\graph}[1]{\Gamma_{#1}}  % graph of a (partial) function
\NewDocumentCommand{\img}  % image
{O{\empty} m G{\empty}}
{{#2}_*\ifthenelse{\equal{#3}{}}{}{\!\sizedescriptor{#1}{\left}( {#3} \sizedescriptor{#1}{\right})}}
\NewDocumentCommand{\pim}  % preimage
{O{\empty} m G{\empty}}
{{#2}^*\ifthenelse{\equal{#3}{}}{}{\!\sizedescriptor{#1}{\left}( {#3} \sizedescriptor{#1}{\right})}}
\newcommand{\ec}[2][]{[\:\!{#2}\:\!]_{#1}}  % equivalence class
\newcommand{\transposed}[1]{\widehat{#1}}


%%%%%%  Projections and Injections
%%%%%%%%%%%%%%%%%%%%%%%%%%%%%%%%%%%%%%%%%%%%%%%%%%%%%%%%%%%%%
\NewDocumentCommand{\fst}
{O{\empty} O{\empty}}
{\pi_1^{{#1}\ifthenelse{\equal{#2}{}}{}{,}{#2}}}
\NewDocumentCommand{\snd}
{O{\empty} O{\empty}}
{\pi_2^{{#1}\ifthenelse{\equal{#2}{}}{}{,}{#2}}}
\NewDocumentCommand{\inl}
{O{\empty} O{\empty}}
{\iota_1^{{#1}\ifthenelse{\equal{#2}{}}{}{,}{#2}}}
\NewDocumentCommand{\inr}
{O{\empty} O{\empty}}
{\iota_2^{{#1}\ifthenelse{\equal{#2}{}}{}{,}{#2}}}


%%%%%%  Categories
%%%%%%%%%%%%%%%%%%%%%%%%%%%%%%%%%%%%%%%%%%%%%%%%%%%%%%%%%%%%%
\newcommand{\ct}[1]{\mathbf{#1}}
\newcommand{\mnoz}{\ct{Mno\check{z}}}
\newcommand{\pkol}{\ct{PKol}}  % category of semirings
\newcommand{\upkol}{\pkol_1}  % category of unital semirings
\newcommand{\kol}{\ct{Kol}}  % category of rings
\newcommand{\ukol}{\kol_1}  % category of unital rings


%%%%%%  Exercises and Solutions
%%%%%%%%%%%%%%%%%%%%%%%%%%%%%%%%%%%%%%%%%%%%%%%%%%%%%%%%%%%%%
\Newassociation{resitev}{Resitev}{resitve}
\renewcommand{\Resitevlabel}[1]{\emph{Re\v{s}itev~#1}}
{
\theoremstyle{definition}
\newtheorem{vaja}{Vaja}[chapter]
}


%%%%%%  Misc.
%%%%%%%%%%%%%%%%%%%%%%%%%%%%%%%%%%%%%%%%%%%%%%%%%%%%%%%%%%%%%
\renewcommand{\divides}{\,|\,}
% Načeloma bi morala biti navpična črta v \divides obdana z \mathrel, ampak to vodi do prevelikih presledkov.
\newcommand{\df}[1]{\emph{\textbf{#1}}}  % defined notion
\newcommand{\oper}{\mathop{\circledast}\nolimits}  % symbol for a generic operation
\newcommand{\soper}{\mathop{\boxasterisk}\nolimits}  % symbol for a second generic operation
\newcommand{\tconc}{\mathop{\bullet}\nolimits}  % symbol for binary tree concatenation
\newcommand{\ism}{\cong}  % isomorphic
\newcommand{\inv}[1]{#1^{-1}} % inverz preslikave
\newcommand{\equ}{\sim}  % equivalent
\newcommand{\dfeq}{\mathrel{\mathop:}=}  % definitional equality
\newcommand{\revdfeq}{=\mathrel{\mathop:}}  % reverse definitional equality
\newcommand{\isdefined}[1]{{#1}\!\downarrow}  % given value is defined
\newcommand{\kleq}{\simeq}  % Kleene equality
\newcommand{\claim}[3]{{#1} \;\colon\; \frac{#2}{#3}}  % claim, divided on context, assumptions, conclusions
\newcommand{\one}{\mathtt{\mathbf{1}}}  % generic singleton
\newcommand{\unit}{\mathord{()}}  % element in a generic singleton
\newcommand{\nul}{\mathtt{N}}  % null map
\newcommand{\suc}{\mathtt{S}}  % successor
\newcommand{\prd}{\mathtt{P}}  % predecessor
\newcommand{\tprd}{\tilde{\prd}}  % predecessor as a total function
\newcommand{\monus}{\mathbin{\vphantom{+}\text{\mathsurround=0pt \ooalign{\noalign{\kern-.35ex}\hidewidth$\smash{\cdot}$\hidewidth\cr\noalign{\kern.35ex}$-$\cr}}}}
% Definicija za monus pobrana s TeX Stack Exchange
\newcommand{\wf}{\prec}  % well-founded order
\NewDocumentEnvironment{implproof}  % proof of an implication
{O{\empty} G{\empty} O{=>} G{\empty}}
{
\begin{description}
\item[\quad$\sizedescriptor{#1}{\left}({#2}
\ifthenelse{\equal{#3}{=>}}{\impl}{
\ifthenelse{\equal{#3}{<=}}{\revimpl}{
\ifthenelse{\equal{#3}{->}}{\rightarrow}{
\ifthenelse{\equal{#3}{<-}}{\leftarrow}{
#3
}}}} {#4}\sizedescriptor{#1}{\right})$]\ \vspace{0.3em}\\
}
{
\end{description}
}


%%%%%%%%%%%%%%%%%%%%%%%%%%%%%%%%%%%%%%%%%%%%%%%%%%%%%%%%%%%%%%%%%%%%%%%%%%%%%%%%%%%%%%%%%%%%%%%%%%%%%%%%%%%%%%%%%%%%%%

%%% Local Variables:
%%% mode: latex
%%% TeX-master: "ucbenik-lmn"
%%% End:

{
\theoremstyle{definition}
\newtheorem{vaja}{Vaja}
}


\begin{document}

\title{Logika in množice -- vaje}
\date{24.~11.~2017}
\maketitle

\bigskip

\noindent
Za relacijo $R \subseteq A \times B$ definiramo
%
\begin{align*}
  R^T &= \set{(y,x) \in B \times A}{(x,y) \in R},\\
  R^C &= \set{(x,y) \in A \times B}{(x,y) \notin R}.
\end{align*}
%
Kompozitum relacij $R \subseteq A \times B$ in $S \subseteq B \times
C$ je relacija $S \circ R \subseteq A \times C$, definirana z
%
\begin{equation*}
  S \circ R = \set{(x,z) \in A \times C}{\some{y \in B}{(x,y) \in R \land (y,z) \in S}}.
\end{equation*}

\bigskip\bigskip

\begin{vaja}
  Dani sta relaciji $R$ in $S$ na množici $A = \set{1,2,3,4,5,6}$:
  \[R = \set{(1,2),(1,4),(1,6),(2,1),(3,4),(3,6),(5,6)}\]
  in
  \[S = \set{(2,4),(2,6),(4,4),(6,6)}.\]
  \begin{enumerate}
    \item
      Narišite relaciji $R$ in $S$.
    \item
      Določite relacije $R^T$, $R \circ S$ in $S \circ R$.
    \item
      Katere od naslednjih lastnosti ima relacija $R$: refleksivnost, irefleksivnost, simetričnost, asimetričnost, antisimetričnost, tranzitivnost, intranzitivnost, sovisnost, strogo sovisnost?
  \end{enumerate}
\end{vaja}

\begin{vaja}
  \
  \begin{enumerate}
    \item
      Naj $a D b$ pomeni "`$a$ je delitelj $b$"'. Izračunajte kompozituma $D \circ D^T$ in $D^T \circ D$.
    \item
      Naj $a M b$ pomeni "`$a$ je mati od $b$"'. Izračunajte kompozituma $M^T \circ M$ in $M \circ M^T$.
    \item
      Naj $a R b$ pomeni $|a - b| = 1$, definirana za $a, b \in \RR$. Izračunajte $R \circ R$ in $R \circ R \circ R$.
  \end{enumerate}
\end{vaja}

\begin{vaja}
  Na množici vseh ravnin v $\RR^3$ vpeljemo relacijo $\perp$:
  \[r_1 \perp r_2 \iff \text{ravnini $r_1$ in $r_2$ sta pravokotni}.\]
  Katere od naslednjih lastnosti ima relacija $\perp$: refleksivnost, irefleksivnost, simetričnost, asimetričnost, antisimetričnost, tranzitivnost, intranzitivnost, sovisnost, strogo sovisnost?
\end{vaja}

\begin{vaja}
  Naj bosta $ab$ in $cd$ dvomestni števili s števkami $a$, $b$, $c$ in $d$. Pravimo, da sta $ab$ in $cd$ v relaciji $Q$, ko velja $a \geq c$ ali $b > d$. Na primer, $32 \; Q \; 15$ velja in prav tako $28 \; Q \; 93$. Ne velja pa $13 \; Q \; 23$.
  \begin{enumerate}
    \item
      Katera izmed števil $72$, $75$, $82$ in $85$ so v relaciji $Q$?
    \item
      Katere od naslednjih lastnosti ima relacija $Q$: refleksivnost, irefleksivnost, simetričnost, asimetričnost, antisimetričnost, tranzitivnost, intranzitivnost, sovisnost, strogo sovisnost?
  \end{enumerate}
\end{vaja}

\begin{vaja}
  Naj bo $A = \set{a,b,c,d}$ in $R = \set{(a,b),(b,c),(c,d),(c,a)}$. Narišite relacijo $R$ in izračunajte $R^3$, $R^{2017}$, $R^+$ in $R^*$.
\end{vaja}

\begin{vaja}
  Polji šahovnice sta v relaciji $R_F$, če lahko figura $F$ pride s prvega polja na drugo
  v eni potezi. Za vse šahovske figure $F$ ugotovite, katere lastnosti (refleksivnost, irefleksivnost, simetričnost, asimetričnost, antisimetričnost, tranzitivnost, intranzitivnost, sovisnost, strogo sovisnost) imajo relacije $R_F$.
\end{vaja}

\begin{vaja}
  Na neskončni šahovnici $\ZZ \times \ZZ$ definiramo naslednjo relacijo:
  \begin{equation*}
    (x,y) R (z,w) \iff \text{šahovski konjiček lahko skoči z $(x,y)$ na $(z,w)$}.
  \end{equation*}
  \begin{enumerate}
    \item Za vsak $n \in \NN$ opišite relacijo $R^n$. (Kdaj sta polji v relaciji $R^n$?)
    \item Za katere $n \in \NN$ je $R^n$ refleksivna?
    \item Ali je $R$ tranzitivna? Za katere $n \in \NN$ je $R^n$ tranzitivna?
    \item Poiščite tranzitivno ovojnico relacije $R$.
  \end{enumerate}
\end{vaja}

\begin{vaja}
  \
  \begin{enumerate}
    \item Dokažite za vse relacije $R$ na množici $A$: če je $R^T \subseteq R$, potem je $R = R^T$.
    \item Poiščite relacijo $R$, za katero velja $R^T = R$.
    \item Poiščite relacijo $R$, za katero velja $R^T = R^C$. % Relacija na prazni množici.
  \end{enumerate}
\end{vaja}

\begin{vaja}
  Naj bodo $A$, $B$, $C$ množice. Ali obstaja taka relacija $I$ na množici $B$, da velja $I \circ R = R$ za vse relacije $R \subseteq A \times B$ in $S \circ I = S$ za vse relacije $S \subseteq B \times C$?
\end{vaja}

\begin{vaja}
  Naj bo $R$ tranzitivna relacija na množici $A$. Dokažite: če je $R$ antisimetrična, je tudi $R^2$ antisimetrična. Ali velja implikacija v obratno smer?
\end{vaja}

\begin{vaja}
  Naj bodo $R$, $S$ in $V$ relacije na množici $A$. Pokažite, da velja
  \begin{enumerate}
    \item $R \subseteq S \iff R^T \subseteq S^T \iff S^C \subseteq R^C$,
    \item $(R \circ S)^T = S^T \circ R^T$,
    \item $(R \cup S)^T = R^T \cup S^T$,
    \item $R \circ (S \cup V) = (R \circ S) \cup (R \circ V)$,
    \item $R \circ (S \cap V) = (R \circ S) \cap (R \circ V)$.
  \end{enumerate}
\end{vaja}

\begin{vaja}
  Naj bosta $R$ is $S$ simetrični relaciji na $A$. Pokažite, da je relacija $R \circ S$ simetrična natanko tedaj, ko je $R \circ S = S \circ R$.
\end{vaja}

\begin{vaja}
  Relacija $R \subseteq A \times B$ je \emph{funkcijska}, če velja
  %
  \begin{equation*}
    \all{x \in A}{\exactlyone{y}{B}{(x,y) \in R}}.
  \end{equation*}
  %
  Vsaka funkcija $f\colon A \to B$ določa funkcijsko relacijo $R_f
  \subseteq A \times B$,
  %
  \begin{equation*}
    R_f = \set{(x,y) \in A \times B}{f(x) = y},
  \end{equation*}
  %
  ki ji pravimo \emph{graf} funkcije~$f$. Obratno vsaka funkcijska
  relacija $R \subseteq A \times B$ določa funkcijo $f_R\colon A \to B$,
  definirano s predpisom
  %
  \begin{equation*}
    f_R(x) = \text{tisti $y \in B$, za katerega velja $(x,y) \in R$}.
  \end{equation*}
  %
  Za naslednje relacije ugotovite, ali so funkcijske. Če so, katero
  funkcijo določajo?
  \begin{enumerate}
    \item
      $K = \set{(x,y) \in \RR \times \RR}{x = y^2}$.
    \item
      $K = \set{(x,y) \in \RR \times \RR}{x = y^3}$.
    \item
      $K = \set{(x,y) \in \RR_{+} \times \RR_{+}}{x = y^2}$, pri čemer je $\RR_{+}$ množica pozitivnih realnih števil.
    \item
      $I = \set{(x,y) \in \RR_{+} \times \RR_{+}}{x y = 1}$, pri čemer je $\RR_{+}$ množica pozitivnih realnih števil.
    \item
      $R = \set{(m,n) \in \NN \times \NN}{(n = 0 \lor n = 1) \land \some{k \in \NN}{m = 2 k + n}}$.
    \item
      Za dani $a \in A$ definiramo relacijo $R_a = \set{(x,y) \in A \times B}{x = a}$.
      % $R_a$ je funkcijska samo v primeru, da je $A = \set{a}$ in $B = \set{b}$.
    \item
      Za dani $b \in B$ definiramo relacijo $R_b = \set{(x,y) \in A \times B}{y = b}$.
  \end{enumerate}
\end{vaja}

\begin{vaja}
  \
  \begin{enumerate}
    \item
      Dokažite: če sta $R$ in $S$ funkcijski relaciji, je tudi $S \circ R$ funkcijska relacija.
    \item
      Ali se lahko zgodi, da je $S \circ R$ funkcijska relacija, če $S$ in $R$ nista funkcijski relaciji?  
  \end{enumerate}
\end{vaja}

\end{document}