\chapter{Strukture}


Informacija, ki jo posamična množica podaja, je zgolj, katere elemente vsebuje. Izkušnje hitro pokažejo, da ta informacija ni najbolje naravnana za matematično delo. Po eni strani je del te informacije pogosto odveč: tipično si lahko z neko množico pomagamo enako, če njene elemente preimenujemo, tj.~če obravnavamo izomorfno množico. Po drugi strani pa je te informacije premalo: ni dovolj, da vemo, katere elemente imamo na voljo, želimo vedeti tudi, kaj lahko s temi elementi počnemo. Podatek o tem imenujemo \df{struktura} množice.

Vzemimo za primer množico realnih števil~$\RR$. Njene elemente lahko poljubno seštevamo, odštevamo in množimo, tj.~izvajamo določene operacije na njih (seveda imamo še cel kup drugih operacij, vključno z delnimi, kot so deljenje, potenciranje, logaritmiranje\ldots). Strukturo, ki je dana z operacijami, imenujemo \df{algebrska} (ali \df{algebrajska} ali \df{algebraična}).

Množico lahko opremimo tudi z raznimi relacijami, tipično z relacijami urejenosti. Na primer, na $\RR$ imamo relaciji $\leq$ in $<$. To imenujemo \df{struktura urejenosti} (ali \df{urejenostna struktura}).

Realna števila si lahko predstavljamo kot točke na številski premici. Vidimo, da lahko potem računamo razdaljo med njimi. Pravimo, da realna števila tvorijo \df{metrični prostor} oziroma da imajo realna števila \df{metrično strukturo}.

Za realne intervale tudi znamo povedati, kdaj so odprti oz.~zaprti. Kadar imamo pojem odprtosti oz.~zaprtosti, to imenujemo \df{topološka struktura}. Prav tako znamo povedati dolžino intervalov. Kadar imamo pojem velikosti podmnožic, to imenujemo \df{merska struktura}.

Te in še nadaljnje strukture boste podrobneje spoznavali pri raznih matematičnih predmetih, v tej knjigi pa se bomo osredotočili zgolj na nekatere osnovne algebrske in urejenostne strukture.

Tipično velja: več kot imamo strukture na neki množici, bolj uporabna je (še zlasti, kadar se strukture med sabo prepletajo --- na primer, dejstvo, da je seštevanje na $\RR$ monotono, povezuje algebrsko in urejenostno strukturo na $\RR$). Ker imajo realna števila tako bogato strukturo, ni presenetljivo, da jih kar naprej uporabljamo. Za primerjavo: množico vseh permutacij $n$ elementov, ki se imenuje simetrična grupa in označi z $S_n$, uporabljate redkeje (je pa še vedno uporabna, saj premore nekaj operacij --- permutacije lahko sklapljamo in obračamo).

Množico, opremljeno z neko strukturo, imenujemo \df{strukturirana množica}. V tem kontekstu golo množico (brez njene dodatne strukture) imenujemo \df{nosilna množica} (te strukture).

Proučevanje struktur je ena temeljnih matematičnih dejavnosti. Na primer, pri predmetu Algebra spoznavate algebrske strukture, pri Topologiji topološke strukture, pri Analizi metrične in gladke strukture itd.

Za proučevanje strukture pa ne zadostuje opazovati zgolj množic, opremljenih s to strukturo, pač pa tudi preslikave med njimi, ki to strukturo na smiseln način ohranjajo. Tovrstnim preslikavam rečemo \df{homomorfizmi}. Kaj točno to pomeni, bomo spoznali pri konkretnih strukturah v nadaljevanju tega poglavja.

\note{nekje (ne nujno tu) debata, kako strukturirano množico podamo preko njene karakterizacije --- potrebna obstoj in enoličnost do izomorfizma}


\section{Algebrske strukture}

Kot rečeno, algebrska struktura je struktura, dana z operacijami. Operacije, na katere ste navajeni, imajo \df{mestnost}, tj.~koliko podatkov (ki jih imenujemo \df{argumenti} ali \df{operandi}) sprejmejo, da vrnejo rezultat. Na primer, seštevanje vzame dva podatka (seštevanca ali sumanda), ki ju zapišemo na levo in desno stran plusa, da dobimo rezultat (vsoto). Seštevanje je torej dvomestna operacija.

Odštevanje je prav tako dvomestna operacija --- od zmanjševanca odštejemo odštevanec in dobimo razliko. To je dvomestni minus, imamo pa tudi enomestni minus, ki vzame število in vrne njegovo nasprotno število. To sta dve različni operaciji in posledično imate zanju tudi dve različni tipki na kalkulatorju. Dvomestni minus je običajno označen kot $-$, enomestni pa kot ${}^+/_-$\;.

Še en primer enomestne operacije je faktoriela: za vsak $n \in \NN$ lahko naračunamo $n!$, kar je spet naravno število. Primer tromestne operacije je mešani produkt vektorjev v trorazsežnem prostoru: za poljubne tri vektorje je njihov mešani produkt število, katerega absolutna vrednost pove prostornino paralelepipeda, ki ga ti vektorji razpenjajo, predznak pa pove orientacijo tega paralelepipeda.

V splošnem je $n$-mestna operacija na množici $A$ dana kot preslikava $A^n \to A$, vsaj ko gre za operacijo, ki tako vzame kot vrne podatke iz množice $A$ --- taki operaciji rečemo \df{notranja}. Če to ne velja, je operacija \df{zunanja}. Vektorji lepo ponazorijo razliko. Seštevanje vektorjev v prostoru je preslikava $\RR^3 \times \RR^3 \to \RR^3$, torej dvomestna notranja operacija. Množenje vektorjev s skalarji $\RR \times \RR^3 \to \RR^3$ je dvomestna zunanja operacija, kjer enega od argumentov vzamemo iz neke druge množice (v tem primeru iz $\RR$). Skalarno množenje $\RR^3 \times \RR^3 \to \RR$ je prav tako dvomestna zunanja operacija, le da je tokrat rezultat iz druge množice. Prej omenjeni mešani produkt je tromestna zunanja operacija $\RR^3 \times \RR^3 \times \RR^3 \to \RR$.

V definiciji $n$-mestne operacije lahko vzamemo tudi $n = 0$. Ničmestna (notranja) operacija je torej preslikava $\one \to A$, se pravi izbira elementa iz $A$.

Obstajajo še splošnejše vrste operacij (npr.~takšne, ki so odvisne od neskončno argumentov), ampak v tej knjigi se ne bomo ukvarjali z njimi.


\subsection{Magme}

Operacije, s katerimi imamo najpogosteje opravka, so tipično dvomestne. Če želimo obravnavati takšne operacije na splošno, si definiramo strukturo, ki zajema zgolj eno tako operacijo.

\begin{definicija}
	\df{Magma} je množica, opremljena z dvomestno notranjo operacijo.
\end{definicija}

Strukturirano množico običajno zapišemo tako, da znotraj okroglih oklepajev najprej zapišemo simbol za nosilno množico, nato pa naštejemo vse sestavne dele strukture (ločene z vejicami). Če imamo strukturo magme na množici $A$ in dano operacijo označimo z $\oper$, tedaj to magmo zapišemo kot $(A, \oper)$. Če hočemo poudariti, da je $\oper$ dvomestna notranja operacija, lahko še natančneje zapišemo $(A,\ \oper\colon A \times A \to A)$.

Imejmo magmi $(A, \oper)$ in $(B, \soper)$. Za preslikavo $f\colon A \to B$ rečemo, da je \df{homomorfizem magem}, kadar ohranja magemsko strukturo v naslednjem smislu: za vse $x, y \in A$ mora veljati
\[f(x \oper y) = f(x) \soper f(y).\]
Z drugimi besedami, vseeno mora biti, če najprej izvedemo magemsko operacijo in nato izvrednotimo preslikavo ali obratno.

Če imamo magmo $(A, \oper)$, lahko posamične elemente množice $A$ povezujemo z operacijo in na ta način generiramo nove. Na primer, iz $x \in A$ lahko sestavimo računske izraze $x \oper x$, $(x \oper x) \oper x$, $x \oper (x \oper x)$ itd. Če začnemo z večimi elementi, recimo $x, y, z \in A$, lahko dobimo bolj raznotere izraze, npr.~$(x \oper y) \oper z$, $z \oper ((y \oper x) \oper z)$ in tako naprej. Vsi ti izrazi so med sabo različni, njihove vrednosti pa so lahko bodisi enake bodisi različne. Na primer, v magmi $(\NN, +)$ so $2 + 5$, $5 + 2$, $4 + 3$ in $(1 + 2) + (2 + 2)$ različni izrazi, ki pa imajo iste vrednosti.

Magemske izraze smo pisali kot zaporedja znakov, ki so vključevala elemente nosilne množice, simbol za operacijo in oklepaje (slednji so pomembni, saj v splošni magmi operacija ni družilna). Primernejši način podajanja takih izrazov so pa pravzaprav dvojiška drevesa. Vsakemu magemskemu izrazu ustreza neprazno dvojiško drevo, katerega listi so opremljeni z oznakami za elemente nosilne množice.

\note{nekaj primerov magemskih izrazov, podanih tako z dvojiškim drevesom kot z oklepajnim nizom}

Namen teh slikic pa ni zgolj ličen način, kako podati računanje neke operacije, pač pa se zadaj skrivajo vsaj tri temeljne ideje, ki so zelo pomembne za algebrske strukture in ki si jih bomo za začetek ogledali na preprostem primeru magem. Te tri ideje so:
\begin{itemize}
	\item
		prosta struktura,
	\item
		homomorfizem kot preslikava, ki ohranja izraze,
	\item
		podajanje algebrske strukture z generatorji in relacijami.
\end{itemize}

Začnimo s pojmom proste strukture. Če imamo katerokoli množico $A$ (ki jo v tem kontekstu običajno imenujemo \df{baza}), jo lahko razširimo do magme na kanoničen način. Naj $T(A)$ označuje množico vseh magemskih izrazov, ki jih lahko dobimo iz elementov množice $A$, tj.~množico vseh nepraznih dvojiških dreves, katerih listi so opremljeni z elementi množice $A$. Množico $T(A)$ opremimo z naslednjo dvojiško operacijo: če imamo izraza $T_1$ in $T_2$, tvorimo drevo, ki sestoji iz korena, katerega levo poddrevo je $T_1$, desno pa $T_2$. Označimo to dobljeno drevo s $T_1 \tconc T_2$.

Vsak element množice $A$ lahko predstavimo z elementom množice $T(A)$: elementu $x \in A$ pripišemo drevo, ki vsebuje zgolj koren, ki je že kar list in je označen z $x$. Po domače povedano: vsaka vrednost je na trivialen način tudi izraz. To preslikavo $\eta_A\colon A \to T(A)$ imenujemo \df{vložitev baze} oz.~\df{vložitev generatorjev}. Izraz `vložitev' je primeren, saj je ta preslikava očitno injektivna --- izvorni element lahko preberemo z edinega lista v njegovi sliki.

Množico $T(A)$ skupaj z dano operacijo imenujemo \df{prosta magma} nad množico $A$ (razlog za to poimenovanje bo postal jasen kasneje, ko si bomo ogledali podajanje algebrske strukture z generatorji in relacijami). Označimo jo z $F(A) \dfeq (T(A), \tconc)$. Ker lahko $A$ vložimo v $T(A)$, smo v tem smislu dejansko razširili poljubno množico do magme.

Kaj pa se zgodi, če je množica $A$ že opremljena s kakšno magemsko operacijo $\oper$? Tedaj imamo homomorfizem magem $a\colon T(A) \to A$, ki vsakemu računskemu izrazu priredi njegovo vrednost v $A$. Na primer, v primeru magme $(\NN, +)$ slikamo med drugim $4 + ((3 + 1) + ((5 + 0) + 2)) \mapsto 15$. Premisli, da je $a$ v splošnem res homomorfizem magem!

Preslikava $a$ je pomembna, ker zajema vso informacijo o algebrski strukturi na $A$. Z drugimi besedami, enakovredno je podati strukturo $(A, \oper)$ oziroma preslikavo $a$. Če imamo $A$ in $\oper$, lahko podamo $a$ kot zgoraj. Obratno, če imamo $a$, tedaj je $A$ njena kodomena, magemsko operacijo pa rekonstruiramo kot $x \oper y = a\big(\eta_A(x) \tconc \eta_A(y)\big)$. Poanta je sledeča: $x \oper y$ je prav tako računski izraz, tako da lahko z $a$ dobimo njegovo vrednost.

Tudi pojem homomorfizma magem lahko opišemo na alternativen način. Premislimo: če sta $(A, \oper)$ in $(B, \soper)$ magmi ter $f\colon A \to B$ preslikava med njima, tedaj je $f$ homomorfizem magem natanko tedaj, ko ohranja \emph{vse} izraze, ne le tiste oblike $x \oper y$.

Intuitivno je to jasno. Vse računske izraze v magmi generiramo s pomočjo magemske operacije in če se ta ohranja v vsakem kosu izraza, se bo ohranjal celoten izraz. Ponazorimo na konkretnem primeru z izrazom $(u \oper v) \oper (w \oper (x \oper y))$:
\[f\Big((u \oper v) \oper \big(w \oper (x \oper y)\big)\Big) = f(u \oper v) \soper f\big(w \oper (x \oper y)\big) =\]
\[= \big(f(u) \soper f(v)\big) \soper \big(f(w) \soper f(x \oper y)\big) = \big(f(u) \soper f(v)\big) \soper \Big(f(w) \soper \big(f(x) \soper f(y)\big)\Big).\]

Kako pa to trditev natančno formulirati in dokazati? Sredstvo, ki ga potrebujemo, je \df{strukturna indukcija}, ki si jo pa bomo ogledali šele \note{tam in tam}. Prihranimo torej podrobnosti za kasneje in se zaenkrat zanašajmo na intuitivno razumevanje.

Definirajmo preslikavo na izrazih $T(f)\colon T(A) \to T(B)$ tako, da v izrazu magme $(A, \oper)$ vsako vrednost $x \in A$, ki se v izrazu pojavi, spremenimo v vrednost $f(x)$, strukturo izraza pa sicer pustimo enako. Na ta način dobimo izraz magme $(B, \soper)$. S pomočjo $T(f)$ lahko natančneje povemo, kaj pomeni, da ``$f$ ohranja računske izraze''. Če vzamemo izraz, ga ovrednotimo in rezultat preslikamo z $f$, moramo dobiti isto, kot če bi v izrazu vsako posamično vrednost preslikali z $f$ (tj.~uporabili $T(f)$ na izrazu) in ovrednotili dobljeni izraz. Torej, če sta magemski strukturi na $A$ in $B$ podani z $a\colon T(A) \to A$ in $b\colon T(B) \to B$, tedaj mora veljati $f \circ a = b \circ T(f)$. Z drugimi besedami, sledeči diagram komutira.
\[\xymatrix@+2em{
T(A) \ar[d]_a \ar[r]^{T(f)} & T(B) \ar[d]^b \\
A \ar[r]_{f} & B
}\]

Ravno tako, kot je magma $(A, \oper)$ v celoti podana s preslikavo $a$, je tudi homomorfizem $f$ v celoti podan s $T(f)$. Opazimo namreč: če izvrednostimo izraz, ki sestoji zgolj iz ene vrednosti, dobimo taisto vrednost, torej $a \circ \eta_A = \id[A]$. Poračunamo lahko $f = f \circ \id[A] = f \circ a \circ \eta_A = b \circ T(f) \circ \eta_A$.

Opazka $a \circ \eta_A = \id[A]$ pa ima še globlje posledice. Iz nje sledi, da je $a$ surjektivna in da je \note{po naravni razčlenitvi preslikave} množica $A$ izomorfna kvocientu množice $T(A)$. Premislimo, da velja še več: \emph{magma} $(A, \oper)$ je izomorfna kvocientu \emph{proste magme} $F(A)$.

\subsection{Polgrupe, monoidi, grupe}

\subsection{Polkolobarji}
\subsection{Kolobarji}
\subsection{Obsegi}
\section{Strukture urejenosti}
\subsection{Mreže}
\subsection{Boolove mreže}
\section{Kategorije}


%%% Local Variables:
%%% mode: latex
%%% TeX-master: "ucbenik-lmn"
%%% End:
