%% TO NI GLAVNA DATOTEKA
\documentclass[11pt,a4paper,twoside]{book}

%%%%%%%%%%%%%%%%%%%%%%%%%%%%%%%%%%%%%%%%%%%%%%%%%%%%%%%%%
%%%  Imported Packages
%%%%%%%%%%%%%%%%%%%%%%%%%%%%%%%%%%%%%%%%%%%%%%%%%%%%%%%%%
\usepackage[slovene]{babel}
\usepackage[utf8]{inputenc}
\usepackage[T1]{fontenc}

\usepackage{ifthen}
\usepackage{amssymb}
\usepackage{amsmath}
\usepackage{amsthm} % Must come after amsmath!
\usepackage{textgreek}
\usepackage{phonetic}
\usepackage{tablefootnote}
\usepackage{datetime2}

\usepackage{xcolor}
\definecolor{andrejcolor}{rgb}{0.7,0,0.7}
\definecolor{davorincolor}{rgb}{0,0.45,1}
\definecolor{anjacolor}{rgb}{1.0,0.5,0}

\newcommand{\andrej}[1]{{\small\textcolor{andrejcolor}{(#1 \ \mbox{--Andrej})}}}
\newcommand{\davorin}[1]{{\small\textcolor{davorincolor}{(#1 \ \mbox{--Davorin})}}}
\newcommand{\anja}[1]{{\small\textcolor{anjacolor}{(#1 \ \mbox{--Anja})}}}

\definecolor{notecolor}{rgb}{0.6,0.5,0.7}
\newcommand{\note}[1]{{\small\textcolor{notecolor}{(#1)}}}
\newcommand{\alert}[1]{{\small\textcolor{red}{\textbf{#1}}}}

\usepackage{tikz}
\usepackage{tkz-graph}
\usepackage{circuitikz}
\usepackage{xparse}
\usepackage{mathrsfs}
\usepackage{mathabx}
\usepackage{answers}

\usepackage{xypic}

%%% Trenutna verzija, če jo imamo
\IfFileExists{./verzija.tex}{\input{./verzija.tex}}{\newcommand{\OPTversion}{unknown}}

%%% Fonti kot v HoTT book.
\usepackage{mathpazo}
\usepackage[scaled=0.95]{helvet}
\usepackage{courier}
\linespread{1.05} % Palatino looks better with this

%%%%%%%%%%%%%%%%%%%%%%%%%%%%%%%%%%%%%%%%%%%%%%%%%%
%%% Globalne nastavitve
\def\OPTtitle{Logika in množice}

%%%%%%%%%%%%%%%%%%%%%%%%%%%%%%%%%%%%%%%%%%%%%%%%%%%%%%%%%%%%%
%%  Page Style & Margins (A4 page = 210mm x 297mm)

% PAGE GEOMETRY
\usepackage[papersize={210mm,297mm}, % A4
            twoside,
            includehead,
            top=1in, % margina na vrhu strani
            bottom=1in, % margina na dnu strani
            inner=1in, % margina na notranji strani strani
            outer=1in, % margina na zunanji strani strani
            bindingoffset=0pt % dodatna margina na notranji strani
           ]{geometry}

%%%%%%%%%%%%%%%%%%%%%%%%%%%%%%%%%%%%%%%%%%%%%%%%%%%%%%%%%%%%%
% HYPERLINKING AND PDF METADATA
\usepackage[backref=page,
            colorlinks,
            citecolor=linkcolor,
            linkcolor=linkcolor,
            urlcolor=linkcolor,
            unicode,
            pdfauthor={Andrej Bauer, Davorin Lešnik},
            pdftitle={\OPTtitle},
            pdfsubject={matematika},
            pdfkeywords={logika,množice,osnove matematike}]{hyperref}
\renewcommand{\backref}[1]{}
\renewcommand{\backrefalt}[4]{%
   \ifcase #1 %
   (Ni citirano.)
   \or
   (Citirano na strani\ #2.)
   \else
   (Citirano na straneh\ #2.)
   \fi}

\definecolor{linkcolor}{rgb}{0,0,0} % Barva hiperpovezav

%%%%%%%%%%%%%%%%%%%%%%%%%%%%%%%%%%%%%%%%%%%%%%%%%%%%%%%%%%%%%
%%%% Header and footers
%%%%%%%%%%%%%%%%%%%%%%%%%%%%%%%%%%%%%%%%%%%%%%%%%%%%%%%%%%%%%

\usepackage{fancyhdr} % To set headers and footers
\pagestyle{fancyplain}
\setlength{\headheight}{15pt}
\renewcommand{\chaptermark}[1]{\markboth{\textsc{Poglavje \thechapter. #1}}{}}
\renewcommand{\sectionmark}[1]{\markright{\textsc{\thesection\ #1}}}

\lhead[\fancyplain{}{{\thepage}}]%
      {\fancyplain{}{\nouppercase{\rightmark}}}
\rhead[\fancyplain{}{\nouppercase{\leftmark}}]%
      {\fancyplain{}{\thepage}}
\cfoot[\texttt{\footnotesize [delovna verzija {\OPTversion}- \DTMnow]}]{\texttt{\footnotesize [delovna verzija {\OPTversion}- \DTMnow]}}
\lfoot[]{}
\rfoot[]{}

%%%%%%%%%%%%%%%%%%%%%%%%%%%%%%%%%%%%%%%%%%%%%%%%%%%%%%%%%%%%%
%%%%%% Macros
%%%%%%%%%%%%%%%%%%%%%%%%%%%%%%%%%%%%%%%%%%%%%%%%%%%%%%%%%%%%%%%%%%%%%%%%%%%%%%%%%%%%%%%%%%%%%%%%%%%%%%%%%%%%%%%%%%%%%%
%%%  Commands
%%%%%%%%%%%%%%%%%%%%%%%%%%%%%%%%%%%%%%%%%%%%%%%%%%%%%%%%%%%%%%%%%%%%%%%%%%%%%%%%%%%%%%%%%%%%%%%%%%%%%%%%%%%%%%%%%%%%%%


%%%%%%%%%%%%%%%%%%%%%%%%%%%%%%%%%%%%%%%%%%%%%%%%%%%%%%%%%%%%%
%%%  Theorems etc.
%%%%%%%%%%%%%%%%%%%%%%%%%%%%%%%%%%%%%%%%%%%%%%%%%%%%%%%%%%%%%
{
\theoremstyle{theorem}
\newtheorem{izrek}{Izrek}[chapter]
\newtheorem{lema}[izrek]{Lema}
\newtheorem{trditev}[izrek]{Trditev}
\newtheorem{posledica}[izrek]{Posledica}
\newtheorem{pravilo}[izrek]{Pravilo}
}

{
\theoremstyle{definition}
\newtheorem{definicija}[izrek]{Definicija}
\newtheorem{opomba}[izrek]{Opomba}
\newtheorem{primer}[izrek]{Primer}
\newtheorem{zgled}[izrek]{Zgled}
\newtheorem{naloga}[izrek]{Naloga}
}


%%%%%%  Proofs
%%%%%%%%%%%%%%%%%%%%%%%%%%%%%%%%%%%%%%%%%%%%%%%%%%%%%%%%%%%%%
% Za dokaze uporabimo amsmath proof, sicer ne deluje \qedhere.


%%%%%%  Auxiliary
%%%%%%%%%%%%%%%%%%%%%%%%%%%%%%%%%%%%%%%%%%%%%%%%%%%%%%%%%%%%%
\newcommand{\sizedescriptor}[2]
{
\ifthenelse{\equal{#1}{0}}{}{
\ifthenelse{\equal{#1}{1}}{\big}{
\ifthenelse{\equal{#1}{2}}{\Big}{
\ifthenelse{\equal{#1}{3}}{\bigg}{
\ifthenelse{\equal{#1}{4}}{\Bigg}{
#2}}}}}
}

\newcommand{\someref}{{\small\textcolor{blue}{[\textbf{ref.}]}}}
\newcommand{\intermission}{\bigskip\medskip}
\newcommand{\ltc}[1]{$\backslash$\texttt{#1}}  % LaTeX command
\newcommand{\nls}[1]{``\textit{#1}''}  % sentence in a natural language

%%%%%%  Logical Quantifiers, λ- and ι-Expressions
%%%%%%%%%%%%%%%%%%%%%%%%%%%%%%%%%%%%%%%%%%%%%%%%%%%%%%%%%%%%%

\newcommand{\all}[1]{\forall #1 .\,}
\newcommand{\some}[1]{\exists #1 .\,}
\newcommand{\exactlyone}[1]{\exists{!} #1 .\,}
\newcommand{\lam}[1]{\lambda #1 .\,}
\newcommand{\that}[1]{\iota #1 .\,}

%%%%%%  Logic
%%%%%%%%%%%%%%%%%%%%%%%%%%%%%%%%%%%%%%%%%%%%%%%%%%%%%%%%%%%%%
\newcommand{\tvs}{\Omega}  % set of truth values
\newcommand{\true}{\top}  % truth
\newcommand{\false}{\bot}  % falsehood
\newcommand{\etrue}{\boldsymbol{\top}}  % emphasized truth
\newcommand{\efalse}{\boldsymbol{\bot}}  % emphasized falsehood
\newcommand{\impl}{\Rightarrow}  % implication sign
\newcommand{\revimpl}{\Leftarrow}  % reverse implication sign
\newcommand{\lequ}{\Leftrightarrow}  % equivalence sign
\newcommand{\xor}{\mathbin{\veebar}}  % exclusive disjunction sign
\newcommand{\shf}{\mathbin{\uparrow}}  % Sheffer connective
\newcommand{\luk}{\mathbin{\downarrow}}  % Łukasiewicz connective


%%%%%%  Sets
%%%%%%%%%%%%%%%%%%%%%%%%%%%%%%%%%%%%%%%%%%%%%%%%%%%%%%%%%%%%%
%  \set{1, 2, 3}         ->  {1, 2, 3}
%  \set{a \in X}{a < 1}  ->  {a ∈ X | a < 1}
\NewDocumentCommand{\set}
{O{auto} m G{\empty}}
{\sizedescriptor{#1}{\left}\{ {#2} \ifthenelse{\equal{#3}{}}{}{ \; \sizedescriptor{#1}{\middle}| \; {#3}} \sizedescriptor{#1}{\right}\}}
%\newcommand{\vsubset}{\Mapstochar\cap}
%\newcommand{\finseq}[1]{{#1}^*}
\newcommand{\pst}{\mathcal{P}}
\renewcommand{\complement}[1]{{#1}^C}


%%%%%%  Number Sets, Intervals
%%%%%%%%%%%%%%%%%%%%%%%%%%%%%%%%%%%%%%%%%%%%%%%%%%%%%%%%%%%%%
\newcommand{\NN}{\mathbb{N}}
\newcommand{\ZZ}{\mathbb{Z}}
\newcommand{\QQ}{\mathbb{Q}}
\newcommand{\RR}{\mathbb{R}}
\newcommand{\CC}{\mathbb{C}}
\newcommand{\intoo}[3][\RR]{{#1}_{(#2, #3)}}
\newcommand{\intcc}[3][\RR]{{#1}_{[#2, #3]}}
\newcommand{\intoc}[3][\RR]{{#1}_{(#2, #3]}}
\newcommand{\intco}[3][\RR]{{#1}_{[#2, #3)}}


%%%%%%  Maps and Relations
%%%%%%%%%%%%%%%%%%%%%%%%%%%%%%%%%%%%%%%%%%%%%%%%%%%%%%%%%%%%%
\newcommand{\id}[1][]{\mathrm{id}_{#1}}  % identity map
\newcommand{\argbox}{{\;\!\fbox{\phantom{M}}\;\!}}  % box for a function argument
\newcommand{\konst}[1]{\mathrm{k}_{#1}} % constant map
\newcommand{\rstr}[1]{\left.{#1}\right|}  % map restriction
\newcommand{\im}{\mathrm{im}}  % map image
\newcommand{\parto}{\mathrel{\rightharpoonup}}  % partial mapping sign
\NewDocumentCommand{\rel}
{O{\empty} O{\empty}}
{\ifthenelse{\equal{#1}{}}{\mathscr{R}}{{#1} \mathrel{\mathscr{R}} {#2}}}  % a relation
\NewDocumentCommand{\srel}
{O{\empty} O{\empty}}
{\ifthenelse{\equal{#1}{}}{\mathscr{S}}{{#1} \mathrel{\mathscr{S}} {#2}}}  % a second relation
\newcommand{\dom}{\mathrm{dom}}  % domain
\newcommand{\cod}{\mathrm{cod}}  % codomain
\newcommand{\dd}[1]{D_{#1}}  % domain of definition
\newcommand{\rn}[1]{Z_{#1}}  % range
\newcommand{\graph}[1]{\Gamma_{#1}}  % graph of a (partial) function
\NewDocumentCommand{\img}  % image
{O{\empty} m G{\empty}}
{{#2}_*\ifthenelse{\equal{#3}{}}{}{\!\sizedescriptor{#1}{\left}( {#3} \sizedescriptor{#1}{\right})}}
\NewDocumentCommand{\pim}  % preimage
{O{\empty} m G{\empty}}
{{#2}^*\ifthenelse{\equal{#3}{}}{}{\!\sizedescriptor{#1}{\left}( {#3} \sizedescriptor{#1}{\right})}}
\newcommand{\ec}[2][]{[\:\!{#2}\:\!]_{#1}}  % equivalence class
\newcommand{\transposed}[1]{\widehat{#1}}


%%%%%%  Projections and Injections
%%%%%%%%%%%%%%%%%%%%%%%%%%%%%%%%%%%%%%%%%%%%%%%%%%%%%%%%%%%%%
\NewDocumentCommand{\fst}
{O{\empty} O{\empty}}
{\pi_1^{{#1}\ifthenelse{\equal{#2}{}}{}{,}{#2}}}
\NewDocumentCommand{\snd}
{O{\empty} O{\empty}}
{\pi_2^{{#1}\ifthenelse{\equal{#2}{}}{}{,}{#2}}}
\NewDocumentCommand{\inl}
{O{\empty} O{\empty}}
{\iota_1^{{#1}\ifthenelse{\equal{#2}{}}{}{,}{#2}}}
\NewDocumentCommand{\inr}
{O{\empty} O{\empty}}
{\iota_2^{{#1}\ifthenelse{\equal{#2}{}}{}{,}{#2}}}


%%%%%%  Categories
%%%%%%%%%%%%%%%%%%%%%%%%%%%%%%%%%%%%%%%%%%%%%%%%%%%%%%%%%%%%%
\newcommand{\ct}[1]{\mathbf{#1}}
\newcommand{\mnoz}{\ct{Mno\check{z}}}
\newcommand{\pkol}{\ct{PKol}}  % category of semirings
\newcommand{\upkol}{\pkol_1}  % category of unital semirings
\newcommand{\kol}{\ct{Kol}}  % category of rings
\newcommand{\ukol}{\kol_1}  % category of unital rings


%%%%%%  Exercises and Solutions
%%%%%%%%%%%%%%%%%%%%%%%%%%%%%%%%%%%%%%%%%%%%%%%%%%%%%%%%%%%%%
\Newassociation{resitev}{Resitev}{resitve}
\renewcommand{\Resitevlabel}[1]{\emph{Re\v{s}itev~#1}}
{
\theoremstyle{definition}
\newtheorem{vaja}{Vaja}[chapter]
}


%%%%%%  Misc.
%%%%%%%%%%%%%%%%%%%%%%%%%%%%%%%%%%%%%%%%%%%%%%%%%%%%%%%%%%%%%
\renewcommand{\divides}{\,|\,}
% Načeloma bi morala biti navpična črta v \divides obdana z \mathrel, ampak to vodi do prevelikih presledkov.
\newcommand{\df}[1]{\emph{\textbf{#1}}}  % defined notion
\newcommand{\oper}{\mathop{\circledast}\nolimits}  % symbol for a generic operation
\newcommand{\soper}{\mathop{\boxasterisk}\nolimits}  % symbol for a second generic operation
\newcommand{\tconc}{\mathop{\bullet}\nolimits}  % symbol for binary tree concatenation
\newcommand{\ism}{\cong}  % isomorphic
\newcommand{\inv}[1]{#1^{-1}} % inverz preslikave
\newcommand{\equ}{\sim}  % equivalent
\newcommand{\dfeq}{\mathrel{\mathop:}=}  % definitional equality
\newcommand{\revdfeq}{=\mathrel{\mathop:}}  % reverse definitional equality
\newcommand{\isdefined}[1]{{#1}\!\downarrow}  % given value is defined
\newcommand{\kleq}{\simeq}  % Kleene equality
\newcommand{\claim}[3]{{#1} \;\colon\; \frac{#2}{#3}}  % claim, divided on context, assumptions, conclusions
\newcommand{\one}{\mathtt{\mathbf{1}}}  % generic singleton
\newcommand{\unit}{\mathord{()}}  % element in a generic singleton
\newcommand{\nul}{\mathtt{N}}  % null map
\newcommand{\suc}{\mathtt{S}}  % successor
\newcommand{\prd}{\mathtt{P}}  % predecessor
\newcommand{\tprd}{\tilde{\prd}}  % predecessor as a total function
\newcommand{\monus}{\mathbin{\vphantom{+}\text{\mathsurround=0pt \ooalign{\noalign{\kern-.35ex}\hidewidth$\smash{\cdot}$\hidewidth\cr\noalign{\kern.35ex}$-$\cr}}}}
% Definicija za monus pobrana s TeX Stack Exchange
\newcommand{\wf}{\prec}  % well-founded order
\NewDocumentEnvironment{implproof}  % proof of an implication
{O{\empty} G{\empty} O{=>} G{\empty}}
{
\begin{description}
\item[\quad$\sizedescriptor{#1}{\left}({#2}
\ifthenelse{\equal{#3}{=>}}{\impl}{
\ifthenelse{\equal{#3}{<=}}{\revimpl}{
\ifthenelse{\equal{#3}{->}}{\rightarrow}{
\ifthenelse{\equal{#3}{<-}}{\leftarrow}{
#3
}}}} {#4}\sizedescriptor{#1}{\right})$]\ \vspace{0.3em}\\
}
{
\end{description}
}


%%%%%%%%%%%%%%%%%%%%%%%%%%%%%%%%%%%%%%%%%%%%%%%%%%%%%%%%%%%%%%%%%%%%%%%%%%%%%%%%%%%%%%%%%%%%%%%%%%%%%%%%%%%%%%%%%%%%%%

%%% Local Variables:
%%% mode: latex
%%% TeX-master: "ucbenik-lmn"
%%% End:


\begin{document}

%--------------------------------------------------------------------
%--------------------------------------------------------------------
% TITLE PAGE

% če se spremeni naslov, je treba spremeniti tudi zgoraj v paketu hyperref
\title{\OPTtitle\\\texttt{\OPTversion}}
\author{Andrej Bauer \and Davorin Lešnik}
\maketitle

%--------------------------------------------------------------------
%--------------------------------------------------------------------
% Foreword

\chapter*{Predgovor}%\addcontentsline{toc}{chapter}{\numberline{}Predgovor}

%--------------------------------------------------------------------
%--------------------------------------------------------------------
% TOC

\tableofcontents
% \listoftables % To se meni zdi nepotrebno, zakaj se to daje v knjige? (Andrej)

%--------------------------------------------------------------------
%--------------------------------------------------------------------
% BODY

\Opensolutionfile{resitve}

\input{matematicno-izrazanje.tex}
\chapter{Množice in preslikave}
\label{cha:mnozice-in-preslikave}


Temeljni gradniki sodobne matematike so \df{množice}, ki so skupki ali zbirke matematičnih
objektov, lahko spet množice. Vsaka množica sestoji iz \df{elementov} in je z njimi
natančno določena. Kadar je $a$ element množice $M$, to zapišemo $a \in M$.

Ideja množice kot poljubne zbirke elementov je zavajajoče preprosta, kar so na lastni koži
izkusili matematiki na prelomu iz 19.~v 20.~stoletje. Takrat so že vedeli, da so množice zelo
uporabne in da lahko iz njih tvorimo razne vrste matematičnih objektov. A znameniti
matematik in filozof Bertrand Russell je odkril paradoks, ki se imenuje po njem, in gre
takole. Naj bo~$R$ množica vseh množic, ki niso element same sebe. Ali $R$ je element~$R$?
Če je $R$ element $R$, potem iz definicije $R$ sledi, da $R$ ni element $R$. In če $R$ ni
element $R$, spet iz definicije $R$ sledi, da $R$ je element $R$. Torej $R$ hkrati je in
ni svoj element, kar je protislovje! Russellov paradoks ste morda že spoznali v
priljubljeni različici, ki govori o vaškem brivcu, ki brije vse vaščane, ki ne brijejo
samih sebe.

Russellov paradoks je povzročil pravo krizo v temeljih matematike. Ker so bile množice
nepogrešljivo orodje, jih niso hoteli kar zavreči, po drugi strani pa je bilo treba
preprečiti Russellov in druge paradokse, ki so jih še odkrili. Bertrand Russell je
predlagal rešitev, ki jo je poimenoval \df{teorija tipov}. Russellova teorija tipov je
pomembno vplivala na nadaljni razvoj temeljev matematike, sodobna teorija tipov pa je
pomembno orodje v računalništvu. Tako kot množice so bili tipi skupki elementov, a so
tvorili neskončno hierarhijo, v kateri so bili elementi tipa vedno iz nižjega nivoja
hierarhije kot tip, ki so mu pripadali. Za potrebe večine matematike zadostuje že
preprostejša dvoslojna hierarhija množic in \df{razredov}. Množice smejo biti elementi
množic in razredov, razredi pa ne. Russellov paradoks izgine, ker je $R$ razred vseh
tistih množic, ki niso same svoj element. Vprašanje, ali je $R$ element samega sebe, tako
postane nesmiselno, saj $R$ ni množica. A zaenkrat odložimo podrobnejšo obravnavo razredov
in se raje posvetimo osnovnima pojmoma, množica in preslikava.

V splošni razpravi o množicah, ki bi presegala meje matematične vede, bi se opirali na
zgodovinski in družbeni kontekst, jezikovni izvor in rabo besed `množica', `skupek' in
`zbirka', kognitivno analizo, eksperimente, filozofijo itn. Vsi ti vidiki so za matematike
izjemo koristni, saj iz takih ``pred-matematičnih'' obravnav črpamo sveže zamisli in
matematiko naredimo zares uporabno. Ko pa delujemo znotraj matematike, zunanje vplive
odmislimo in se zanašamo le še na pravila logičnega sklepanja in matematične zakone, da ne
prihaja do nejasnosti in dvomljivih sklepov.

Kot matematiki lahko ustvarimo takšen ali drugačen pojem množice in pri tem imamo popolno
svobodo. Se množica lahko spreminja ali vedno vsebuje iste elemente? Je pomemben vrsti red
elementov v množici? Sme množica biti element same sebe? Ali morajo biti elementi množice
izračunljivi? To so vprašanja, ki nimajo enoznačnega odgovora. In res je znanih več med
seboj nezdružljivih zvrsti teorije množic, ki matematično opredeljujejo različne vidike
običajnega razumevanja besede `množica'. Mi bomo spoznali ``standardno'' teorijo množic,
ki jo uporablja velika večina matematikov.


\section{Načelo ekstenzionalnosti}
\label{sec:nacelo-ekstenzionalnosti}

Zamisel, da je množica natančno določena s svojimi elementi, izrazimo z matematičnim
zakonom, ki mu pravimo \df{načelo ekstenzionalnosti}:

\begin{pravilo}[Ekstenzionalnost množic]
  Množici sta enaki, če vsebujeta iste elemente.
\end{pravilo}

Kaj pravzaprav pomeni, da je to ``pravilo'', ``matematični zakon'' ali ``načelo''? So ga
razglasili v parlementu, je to zakon narave, ali morda dogma, ki jo je razglasil profesor
na predavanjih? Bodo tisti, ki načela ekstenzionalnosti ne spoštujejo, deležni Lešnikove
masti? Ne. Matematični zakoni so \emph{dogovori}, nekakšna pravila matematične igre. V
zgodovinskem razvoju matematike so se uveljavili tisti dogovori, ki so bili uporabni v
naravoslovju in tehniki, ali pa so v njih matematiki videli notranjo lepoto in lastno
uporabno vrednost.

Pravkar smo se dogovorili, da bomo obravnavali matematične objekte množice, ki vsebujejo
elemente in da zanje velja načelo ekstenzionalnosti. Namesto besed `množica' in `element'
bi lahko izbrali tudi kaki drugi besedi, denimo `zbor' in `član', ali celo `morje' in
`riba', s čimer se matematična vsebina pojmov ne bi čisto nič spremenila, čeprav ne gre
preveč izzivati svojih stanovskih kolegic in kolegov. Strukturo, lastnosti in povezave med
matematičnimi objekti namreč določajo dogovorjeni matematični zakoni in ne besede, s
katerimi jih poimenujemo.

Še enkrat poudarimo, da ima vsakdo, še posebej pa mladi um, popolno svobodo matematičnega
ustvarjanja. Želite razmišljati o drugačnih množicah, ki ne zadoščajo načelou
ekstenzionalnsti? Ali pa o številih, ki zadoščajo zakonu $x + x = 0$? O geometriji, v
kateri skozi točko lahko potegnemo dve vzporednici k dani premici? Kar dajte! Pri tem vas
le prosimo, celo zahtevamo, da razmišljate temeljito, vztrajno in globoko, da ste iskreni
do sebe in ostalih ter da svoje zamisli in spoznanja predstavite na matematikom razumljiv
način.

Vrnimo se k našim množicam. Načelo ekstenzionalnosti nam pove, da lahko množico podamo
tako, da natančno opredelimo njene elemente. A to ne pomeni, da množica obstaja, brž ko jo
lahko natančno opredelimo! To je pot, ki vodi naravnost do Russelovega paradoksa, saj so
elementi paradoksalne množice~$R$ natančno opredeljeni. Potrebujemo dodatna pravila, ki
določajo dopustne \df{konstrukcije množic}. Izbrati jih moramo previdno, da se izognemo
težavam.

\section{Končne množice}
\label{sec:koncne-mnozice}

Posebej preprosta konstrukcija množic združi končen nabor matematičnih objektov v množico.
Na primer, če so $a$, $b$ in $c$ matematični objekti, potem lahko tvorimo množico
%
\begin{equation*}
  \set{a, b, c}
\end{equation*}
%
katere objekti so natanko $a$, $b$ in $c$. To pomeni, da za vsak matematični objekt~$x$
velja
%
\begin{equation*}
  \text{$x \in \set{a, b, c}$, če in samo če $x = a$ ali $x = b$ ali $x = c$.}
\end{equation*}
%
Fraza ``če in samo če'' tu pomeni, da velja dvoje:
%
\begin{enumerate}
\item Če $x = a$ ali $x = b$ ali $x = c$, potem $x \in \set{a, b, c}$.
\item Če $x \in \set{a, b, c}$, potem $x = a$ ali $x = b$ ali $x = c$.
\end{enumerate}
%
Tako nam na primer prva trditev zagotavlja $1+1 \in \set{1, 2, 3}$, ker velja
vsaj ena od možnosti: $1 + 1 = 1$ ali $1 + 1 = 2$ ali $1 + 1 = 3$. Iz druge trditve sledi, da
$5 \in \set{1, 2, 3}$ ne velja, ker ne velja nobena od možnosti: $5 = 1$ ali $5 = 2$ ali
$5 = 3$.

Splošna konstrukcija končnih množic poteka takole.

\begin{pravilo}
  \label{pravilo:koncna-mnozica}
  Za vse objekte $a$, $b$, \dots, $z$ je $\set{a, b, \ldots, z}$ množica, katere elementi
  so natanko objekti $a$, $b$, \dots, $z$.
\end{pravilo}

Za trenutek ustavimo tok misli in opozorimo, da zapis s tropičjem `$\ldots$' ni dovolj
natančen, saj dopušča dvoumnosti. Denimo, so elementi množice
%
\begin{equation*}
  \set{3, 5, 7, \ldots, 31},
\end{equation*}
%
liha števila med $3$ in $31$, ali samo praštevila? Zapis res ni dovolj natančen. Kljub
temu tak zapis v praksi uporabljamo, ker v praksi bralec večinoma pravilno ugane, kaj je
bilo mišljeno, saj imamo ljudje zelo podobne sposobnosti prepoznavanja vzorcev. Z
matematičnega vidika pa to ni dopustno, saj lahko tropičje \emph{vedno} razumemo na več
načinov. (Ne verjamete? Naslednji člen v zaporedju $1, 2, 3, \ldots$ je seveda~$5$, ker je
naslednji člen vsota prejšnjih dveh, kot v Fibonaccijevem zaporedju.)

Kot smo že omenili, želimo pojem množice, pri kateri vrstni red elementov ni pomemben.
Torej bi morali biti množici $\set{1, 2}$ in $\set{2, 1}$ enaki. Pa je to res? Velja ena
od treh možnosti:
%
\begin{enumerate}
\item Iz načela ekstenzionalnosti in konstrukcije množic $\set{1, 2}$ in $\set{2, 1}$ sledi, da sta enaki.
\item Iz načela ekstenzionalnosti in konstrukcije množic $\set{1, 2}$ in $\set{2, 1}$ sledi, da nista enaki.
\item Načelo ekstenzionalnosti in konstrukcije množic $\set{1, 2}$ in $\set{2, 1}$ ne določajo, ali sta enaki.
\end{enumerate}
%
V prvem primeru bi želeli dokazati enakost. V drugem primeru smo v zagati, saj smo se
dogovorili za matematična pravila, ki imajo neželene posledice. V tretjem primeru moramo
dodati še kakšne nove zakone o množicah. Na srečo obvelja prva možnost.

\begin{trditev}
  Množici $\set{1, 2}$ in $\set{2, 1}$ sta enaki.
\end{trditev}

\begin{proof}
  Dokaz, ki ga bomo zapisali je izjemno podroben in ga v praksi matematik ne bi zapisal,
  saj je z njegovim branjem več dela, kot če bi naredili sami. Ker pa želimo pokazati, da
  tudi najbolj trivialna dejstva lahko dokažemo, ga zapišimo.

  Izhajati smemo izključno iz naslednji dejstev:
  %
  \begin{itemize}
  \item načelo ekstenzionalnosti,
  \item $x \in \set{1, 2}$, če in samo če $x = 1$ ali $x = 2$,
  \item $x \in \set{2, 1}$, če in samo če $x = 2$ ali $x = 1$.
  \end{itemize}
  %
  Najprej uporabimo načelo ekstenzionalnosti, ki zagotavlja, da sta $\set{1, 2}$ in
  $\set{2, 1}$ enaki, če imata iste elemente. Dokažimo torej, da imata iste elemente. To
  naredimo v dveh korakih:
  %
  \begin{enumerate}
  \item Dokažimo, da za vsak element $\set{1, 2}$ dokažemo, da je element $\set{2, 1}$.
    Naj bo $x \in \set{1, 2}$. Iz definicije množice $\set{1, 2}$
    sledi, da je $x = 1$ ali $x = 2$. Obravnavamo dva podprimera:
    %
    \begin{enumerate}
    \item Primer $x = 1$: iz $x = 1$ sledi, da je $x = 2$ ali $x = 1$, zato je $x \in \set{2, 1}$.
    \item Primer $x = 2$: iz $x = 2$ sledi, da je $x = 2$ ali $x = 1$, zato je $x \in \set{2, 1}$.
    \end{enumerate}
    %
  \item Dokažimo, da za vsak element $\set{2, 1}$ dokažemo, da je element $\set{1, 2}$.

    Ta korak je povsem podoben prvemu, le da je treba povsod zamenjati~$1$ in~$2$.
    Matematik bi zato na tem mestu zapisal, da je drugi korak podoben prevemu in dokaz
    zaključil. A tega tokrat ne bomo storili in bomo zapisali popoln dokaz.

    Naj bo $x \in \set{2, 1}$. Iz definicije množice $\set{2, 1}$ sledi, da je $x = 2$ ali
    $x = 1$. Obravnavamo dva primera:
    %
    \begin{enumerate}
    \item Primer $x = 2$: iz $x = 2$ sledi, da je $x = 1$ ali $x = 2$, zato je $x \in \set{1, 2}$.
    \item Primer $x = 1$: iz $x = 1$ sledi, da je $x = 1$ ali $x = 2$, zato je $x \in \set{1, 2}$. \qedhere
    \end{enumerate}
    %
  \end{enumerate}
\end{proof}

Mimogrede, črn kvadratek označuje konec dokaza. Imenuje se tudi ``Halmos'' po matematiku
Paulu Halmosu, ki ga je prvi uporabljal. S podobnim razmislekom, ki ga prepuščamo za vajo,
lahko dokažemo, da ni pomembno, ali se element pojavi enkrat ali večkrat.

\begin{naloga}
  Podrobno dokažite, da sta množici $\set{1, 1, 2}$ in $\set{1, 2}$ enaki.
\end{naloga}

V prejšnji nalogi smo zapisali $\set{1, 1, 2}$. Pa je to sploh dovoljeno?
Pravilo~\ref{pravilo:koncna-mnozica} pravi, da lahko iz objektov $a, b, c, \ldots, z$
tvorimo končno množico $\set{a, b, \ldots, z}$. Nikjer ne piše, da smeta biti $a$ in $b$
enaka, zato je upravičeno vprašanje, ali je dovoljeno za $a$ in $b$ vzeti~$1$. V
matematiki vse razumemo dobesedno. V pravilu~\ref{pravilo:koncna-mnozica} piše ``Za vse
objekte'', torej imamo povsem proste roke. Povedano z drugimi besedami, množico
$\set{1, 1, 2}$ smemo tvoriti, ker nikjer ne piše, da morajo biti elementi različni.

V zvezi s pravilom~\ref{pravilo:koncna-mnozica} se pojavljajo še drugi dvomi. Ali smemo
tvoriti množico, ki ima več elementov, kot je črk abecede? Ali bi bilo pravilo še vedno
isto, če bi namesto ``$a, b, \ldots, z$'' zapisali ``$a, b, \ldots, j$''? Ali smemo
tvoriti množico z nič elementi? Če namreč vstavimo nič elementov, se pravilo glasi ``Za
vse objekte je $\set{\,}$ množica, katere elementi so natanko objekti,'' kar je vsaj
nenavadno. Iz nesrečnega tropičja se res ne vidi, kaj je in kaj ni dovoljeno. Če poškilite
v razdelek~\ref{sec:aksiomi-teorije-mnozic}, kjer so našteti ``uradni'' aksiomih teorije
množic, tam pravila o končnih množicah ne boste našli, saj sledi iz treh bolj osnovnih
pravil.

\begin{pravilo}
  \label{pravilo:prazna-mnozica}
  \df{Prazna množica} $\emptyset$ je množica, ki nima elementov.
\end{pravilo}

\begin{pravilo}
  \label{pravilo:neurejeni-dvojec}
  Za vsak $x$ in $y$ je \df{(neurejeni) par} ali \df{dvojec} $\set{x, y}$ množica, katere
  elementa sta natanko $x$ in $y$.
\end{pravilo}

\begin{pravilo}
  \label{pravilo:unija}
  Za vsaki množici $A$ in $B$ je \df{unija $A \cup B$} množica, ki ima za elemente
  natanko vse objekte, ki so element $A$ ali element $B$.
\end{pravilo}

V pravilu~\ref{pravilo:neurejeni-dvojec} smo besedo ``neurejeni'' zapisali v oklepaju, kar
pomeni, da beseda pravzaprav ni pombembna in bi jo lahko tudi izpustili. Se pravi, da
``neurejeni dvojec'' in ``dvojec'' pomenita isto. V primeru nejasnosti raje uporabimo
daljšo obliko.

Tri nova pravila skupaj nadomestijo pravilo~\ref{pravilo:koncna-mnozica} in odstranijo
marsikateri dvom o uporabi. Prvo pravilo pojasni, da lahko tvorimo množico brez elementov.
Poleg oznake $\emptyset$ je za prazno množico smiselno uporabiti tudi zapis $\set{\,}$.

Drugo pravilo pove, kako lahko tvorimo množico z dvema elementoma, pa tudi z enim.
Spomnimo se, pravila je treba brati dobesedno: za $x$ in $y$ bi lahko vzeli dvakrat isti
objekt~$z$ in tvorili množico $\set{z, z}$, ki ima natanko elementa $z$ in $z$. To je
pravzaprav množica z enim samim elementom $z$, zato ji pravimo tudi \df{enojec} in jo
zapišemo~$\set{z}$.

Tretje pravilo nam omogoča, da tvorimo večje množice. Denimo, množico z elementi $a$, $b$,
$c$ lahko tvorimo kot unijo
%
\begin{equation*}
  \set{a, b} \cup \set{c}.
\end{equation*}
%
To ni edini način, enako množico lahko dobimo na več načinov:
%
\begin{equation*}
  (\set{a} \cup \set{b}) \cup \set{c}
  \quad\text{ali}\quad
  \set{b} \cup \set{c, a}
  \quad\text{ali}\quad
  \set{a,c,a} \cup \set{b,c}
  \quad\text{itn.}
\end{equation*}
%
Seveda bi morali dokazati, da so vse te množice enake, a tega ne bomo storili.

Pogosto nam bo prišlo prav, da bomo imeli pri roki množico z enim elementom, pri čemer nam
bo vseeno, kaj ta element je. V ta namen postavimo pravilo, ki zagotavlja obstoj množice z
enim elementom.

\begin{pravilo}
  \label{pravilo:enojec}
  \df{Standardni enojec} je množica~$\one$, katere edini element je~$\unit$.
\end{pravilo}

Morda se zdi nenavadno, da množico označimo s številom, a ta občutek bo hitro izginil, ko
bomo računali z množicami. Pravaprav bi lahko prazno množico označili z nič $\mathbf{0}$,
in nekateri matematiki to dejansko počnejo.

Edini element množice $\one$ smo označili z nenavadnim zapisom $\unit$. Na tem mestu ne
bomo pojasnili, zakaj pišemo tako, radovedneži pa lahko pogledajo v
razdelek~\ref{sec:algebra-mnozic}. Mimogrede, seveda velja $\one = \set{\unit}$.

Pravilo~\ref{pravilo:enojec} ni nujno potrebno, saj lahko tvorimo veliko različnih enojcev
kar sami $\set{\emptyset}$, $\set{42}$, $\set{\set{\emptyset}}$ itn. Ali je kateri od njih
``prvi med enakimi'' in bi ga lahko uporabljali kot ``standardni'' enojec? Ker je odgovor
v veliki meri stvar osebnega mnenja, je bolje, da razglasimo pravilo, ki ustoliči
standardni enojec. S prazno množico nimamo podobnih težav, saj je ena sama.

% \subsection{Druge množice}

% \andrej{To ne paše sem, ker bi bilo tu dosti bolj naravno nadaljevati s preslikavami.
%  To bomo prestavili na mesto, kjer bo dejansko prišlo prav.}

% Množice, s katerimi v matematiki delamo, tipično vsebujejo števila, ali pa so vsaj na tak ali drugačen način izpeljane iz številskih množic. Spomnimo se standardnih oznak najpogosteje uporabljanih številskih množic.
% \begin{center}
% \begin{tabular}{|cc|}
% \hline
% \textbf{Množica} & \textbf{Oznaka} \\
% \hline
% množica naravnih števil & $\NN$ \\
% množica celih števil & $\ZZ$ \\
% množica racionalnih števil & $\QQ$ \\
% množica realnih števil & $\RR$ \\
% množica kompleksnih števil & $\CC$ \\
% \hline
% \end{tabular}
% \end{center}

% Nekateri $0$ vzamejo za naravno število, nekateri ne. To je v celoti stvar dogovora, kaj pomeni pojem ``naravno število''. Za nas bo prišlo bolj prav, če ničlo štejemo kot element množice naravnih števil, torej $\NN = \set{0, 1, 2, 3, \ldots}$.

\section{Preslikave}

Temelj matematike ne tvorijo le množice, ampak tudi drugi matematični pojmi. Prvi izmed
njih je \df{preslikava}, oziroma s tujko \df{funkcija}.\footnote{Nekateri uporabljajo
  izraz ``funkcija'' samo za tiste preslikave, ki slikajo v realna ali kompleksna števila,
  vendar to navado izpodriva računalništvo, saj funkcije v programskih jezikih nimajo
  omejitev. Dandanes večina matematikov besedo ``funkcija'' obravnava kot sopomenko besede
  ``preslikava'' in tako jo bomo uporabljali tudi mi.} V srednji šoli ste že spoznali
nekatere preslikave, kot so na primer linearne preslikave, trigonometrijske funkcije,
logaritem itd. Nas pa ne bodo zanimale posamezne preslikave, ali posebne lastnosti
preslikav, ampak preslikave na splošno.

Vsaka preslikava ima tri sestavne dele: \df{domeno} ali \df{začetno množico},
\df{kodomeno} ali \df{ciljno množico} in \df{predpis}. Domeni se pogosto reče tudi
\df{definicijsko območje}. Če govorimo o preslikavi, ki ima domeno~$X$ in kodomeno~$Y$, to
ponazorimo s puščico med $X$ in $Y$, takole
%
\begin{equation*}
  \xymatrix{
    {X} \ar[r] &
    {Y}
  }
\end{equation*}
%
Če želimo preslikavo poimenovati, na primer $f$, zapišemo
%
\begin{equation*}
  \xymatrix{
   {f : X} \ar[r] &
    {Y}
  }
  \qquad\text{ali}\qquad
  \xymatrix{
   {X} \ar[r]^{f} &
   {Y}
  }
\end{equation*}
%
Pravimo, da je \df{$f$ preslikava iz $X$ v $Y$}. Zapis nad puščico je prikladen, kadar
imamo opravka z večimi preslikavami, ki jih predstavimo z diagramom. Na primer,
%
\begin{equation*}
  \xymatrix{
    {X} \ar[r] &
    {Y} \ar[r]^{f} &
    {Z}  &
    {W} \ar[l]_{g}
  }
\end{equation*}
%
nam pove, da imamo opravka z (neimenovano) preslikavo iz $X$ v $Y$, s preslikavo $f$ iz
$Y$ v $Z$ in s preslikavo $g$ is $W$ v $Z$. Diagrami so lahko še precej bolj zapleteni.

Tretji del preslikave je predpis, ki določa, kako elemente domene preslikamo v elemente
kodomene. Kaj pravzaprav to pomeni? Možnih je več odgovorov. V srednji šoli predpis
enačimo z matematično formulo, ki spremenljivko preslika v vrednost, na primer $x$ slika v
$2 \sin(x + \pi/4)$. S simboli to zapišemo
%
\begin{equation*}
  x \mapsto 2 \sin(x + \pi/4).
\end{equation*}
%
in preberemo ``$x$ se slika v dvakrat sinus od $x$ plus pi četrtin.''
%
Matematiki smo natančni, zato ne mešamo uporabe puščic $\to$ in $\mapsto$. Navadna puščica
se uporablja pri oznaki domene in kodomene, repata pa v predpisu. V računalništvu besedo
`predpis' razumemo kot `programska koda' in o preslikavah razmišljajo kar kot o
algoritmih --- tudi to je eden od možnih pogledov na preslikave.

V teoriji množic razumemo besedo `predpis' kot kakršnokoli prirejanje med elementi množic
domene~$X$ in kodomene~$Y$, mora pa veljati:
%
\begin{itemize}
\item \df{celovitost}: vsakemu elementu iz $X$ je prirejen vsaj en element iz $Y$,
\item \df{enoličnost}: če sta elementu $x$ prirejena $y \in Y$ in $z \in Y$, potem $y = z$.
\end{itemize}

Za vsako množico~$A$ je \df{identiteta} na~$A$ preslikava
%
\begin{equation*}
  \id[A] : A \to A
\end{equation*}
%
ki poljubnemu elementu $x \in A$ priredi~$x$. To je celovito prirejanje, saj vsak
$x \in A$ ima prirejeni element, namreč kar $x$, je pa tudi enolično: če sta $y_1$ in
$y_2$ prirejena $x \in A$, potem sta oba enaka~$x$ in zato enaka drug drugemu.

Za vsaki množici $A$ in $B$ ter $b \in B$ \df{konstantna preslikava}
%
\begin{equation*}
  \konst{b} : A \to B
\end{equation*}
%
priredu vsakemu elementu iz~$A$ element~$b$. Sami premislite, da je tako prirejanje
celovito in enolično.

\subsection{Funkcijski predpisi}
\label{sec:funkcijski-predpisi}

Predpise lahko podamo na različne načine, najbolj pogost pa je \df{funkcijski predpis}, ki
se mu še posebej posvetimo in se ob njem naučimo nekaj natančnosti. Funkcijski predpis ima
obliko
%
\begin{equation*}
  x \mapsto \cdots,
\end{equation*}
%
ki smo jo že videli maloprej. Na desni, lahko namesto $\cdots$ zapišemo izraz, v katerem
se sme pojaviti simbol~$x$, denimo
%
\begin{equation*}
  x \mapsto 1 + x^2.
\end{equation*}
%
S funkcijskip predpisom zapišemo identiteto in konstantno preslikavo takole:
%
\begin{align*}
  \id[A] &: A \to A
  &
  \konst{b} &: A \to B
  \\
  \id[A] &: x \mapsto x
  &
  \konst{b} &: x \mapsto b.
\end{align*}

Ni nujno, da se~$x$ pojavi, denimo $x \mapsto 42$ vsakemu elementu iz domene priredi
število $42$. V funkcijskem predpisu se smejo pojaviti tudi drugi simboli, ki jim
pravimo \df{parametri}. Tako je
%
\begin{equation*}
  x \mapsto a \cdot x + b
\end{equation*}
%
funkcijski predpis s parametroma $a$ in $b$, ki elementu $x$ priredi element $a \cdot x + b$.

Spremenljivka $x$ nima v naprej določene vrednosti, pač pa kaže, kam lahko vstavimo
elemente domene. Pravimo, da je $x$ \df{vezana spremenljivka}, kar pomeni, da je veljavna
le v funkcijskem predpisu, nanj je vezana, in da ni pomembno, s katerim simbolom jo
označimo. Tako sta funkcijska predpisa
%
\begin{equation*}
  x \mapsto 1 + x^2
  \qquad\text{in}\qquad
  a \mapsto 1 + a^2
\end{equation*}
%
enaka in lahko bi celo pisali $\Box \mapsto 1 + \Box^2$ ali
$\heartsuit \mapsto 1 + \heartsuit^2$.

V funkcijskem predpisu mora na levi stati en sam simbol, ki na desni kaže, kam je treba
vstaviti element iz domene. Tako
%
\begin{equation*}
  \sin(x) \mapsto \cos(2 x),
  \qquad
  3 + 2 \mapsto 5
  \qquad\text{in}\qquad
  \sin(x) \mapsto 2 \cdot \sin(x)
\end{equation*}
%
\emph{niso} veljavni funkcijski predpisi.

Seveda dopuščamo možnost, da se vezana spremenljivka pojavi enkrat, večkrat ali sploh ne.
Funkcijska predpisa
%
%
\begin{equation*}
  x \mapsto 42
  \qquad\text{in}\qquad
  x \mapsto x \cdot \sin(x)
\end{equation*}
%
sta torej veljavna.

Če želimo preslikavo z danim funkcijskim predpisom poimenovati, na primer $f$, zapišemo
%
\begin{equation*}
  f : x \mapsto 1 + x^2.
\end{equation*}
%
To preberemo ``$f$ slika $x$ v ena plus $x$ na kvadrat.'' Običajna sta tudi zapisa
%
\begin{equation*}
  f(x) = 1 + x^2
  \qquad\text{in}\qquad
  f(x) \dfeq 1 + x^2.
\end{equation*}
%
Funkcijske predpise je podrobno prvi preučeval Alonzo Church,\footnote{Alonzo Church
  (1903--1995) je bil ameriški matematik in logik, ki je pomembno prispeval k razvoju
  logike in teoretičnega računalništva. Njegov študent, Dana Stewarta Scott, je imel
  študenta Marka Petkovška in Andreja Bauerja, slednji pa je imel študenta Davorina
  Lešnika.} ki je uporabljal zapis
%
\begin{equation*}
  \lambda x \,.\, 1 + x^2
\end{equation*}
%
in teorijo funkcijskih predpisov poimenoval \df{$\lambda$-račun}. V logiki se je njegov
zapis obdržal in se uveljavil tudi v programski jezikih:
%
\begin{itemize}
\item v Pythonu pišemo \verb|lambda x : 1+x**2|,
\item v Haskellu pišemo \verb|\x -> 1+x**2| in
\item v OCamlu pišemo \verb|fun x => 1+x*x|.
\end{itemize}
%
Predvsem v programiranju funkcijskim predpisom pravijo tudi \df{anonimne} ali \df{brezimne
  preslikave}.

Nekateri starejši zapisi funkcijskih predpisov so slabi, a jih ljudje vztrajno
uporabljajo. Opozorimo le na en slab zapis, ki povzroča precej preglavic, ne da bi se
matematiki tega zares zavedali. Funkcijski predpis mora določati vezano spremenljivko,
sicer ne vemo, kako vstaviti vrednosti, a na žalost jo matematiki pogosto izpustijo skupaj
z $\mapsto$, da ostane samo izraz na desni.
%
Težava je v tem, da se lahko v funkcijskem predpisu pojavi več kot en simbol. Če vam na primer povem, da imam v mislih funkcijski predpis
%
\begin{equation*}
  a \cdot x + b
\end{equation*}
%
boste vsi mislili, da je mišljeno $x \mapsto a \cdot x + b$. A pravzprav bi lahko bilo
tudi $a \mapsto a \cdot x + b$ ali $b \mapsto a \cdot x + b$ ali celo
$t \mapsto a \cdot x + b$! Namreč, nič ni narobe s funkcijskim predpisom, v katerem se
pojavijo dodatni simboli.

Morda pa lahko vezano spremenljivko in $\mapsto$ brez škode izpustimo, če v izrazu nastopa
samo en simbol, denimo $1 + x^2?$
%
A spet bi zabredli v težave. Je $42$ število ali funkcijski predpis $x \mapsto 42$? Je
$1 + x^2$ funkcijski predpis $x \mapsto 1 + x^2$ ali $a \mapsto 1 + x^2$?

Velikokrat površno rečemo, da funkcijski predpis podaja preslikavo. To ni res, saj smo že
prej povedali, da ima vsaka preslikava tri sestavne dele: domeno, kodomeno in prirejanje.
Res, če ne poznamo domene, ne moremo preveriti, ali je funkcijski predpis celovit. Denimo,
funkcijski predpis
%
\begin{equation*}
  x \mapsto \frac{x}{x^2 - 2}
\end{equation*}
%
ni celovit, če je domena množica realnih števil, in je celovit, če je domena množica
racionalnih števil. Tudi kodomeno moramo poznati, sicer ne moremo določiti nekaterih
lastnosti preslikave, kot je na primer surjektivnost, glej
razdelek~\ref{razdelek:injektivnost-in-surjektivnost}.



\subsection{Ostali načini podajanja preslikav}
\label{sec:ostali-predpisi}

Funkcijski predpisi niso edini način za podajanje prirejanja, zato omenimo še nekatere
druge.

Preslikavo s končno domeno lahko podamo s tabelo, na primer:
%
\begin{center}
  $f : \set{1, 2, 3, 5} \to \set{10, 20, 30}$

  \medskip

  \begin{tabular}{|c|c|} \hline
    1 & 10 \\ \hline
    2 & 10 \\ \hline
    3 & 20 \\ \hline
    5 & 10 \\ \hline
  \end{tabular}
\end{center}
%
To seveda pomeni, da $f$ elementu $1$ priredi $10$, $2$ priredi $10$, $3$ priredi $20$ in $5$
priredi $10$. Tabelo lahko predstavimo na različne načine, lahko kar naštejemo vsa prirejanja:
%
\begin{align*}
  f(1) &= 10 \\
  f(2) &= 10 \\
  f(3) &= 20 \\
  f(5) &= 10.
\end{align*}
%
Tudi
%
\begin{align*}
  1 &\mapsto 10 \\
  2 &\mapsto 10 \\
  3 &\mapsto 20 \\
  5 &\mapsto 10.
\end{align*}
%
je še vedno le tabela, ki prikazuje prirejanje. Ne sme nas motiti dejstvo, da smo
$\mapsto$ uporabili za naštevanje prirejanj, namesto za funkcijski prdpis.

Preslikava je lahko določena tudi z opisom računskega postopka, pravimo mu \df{algoritem},
s pomočjo katerega izračunamo vrednost preslikave pri danem argumentu. Paziti moramo, da je
opis postopka res natančen in nedvoumen, lahko ga kar zapišemo kot program. Teoretični
računalničar bi pripomnil, da je treba pri tem izbrati programski jezik, ki ima ustrezno
matematično definicijo.

Preslikave lahko podamo tudi tako, da opišemo pogoje, pri katerih je element kodomene
prirejen elementu domene. Na primer, preslikavo $f : \NN \to \ZZ$ bi lahko definirali z
zahtevo, da naravnemu številu $n \in \NN$ priredimo celo število $k \in \ZZ$, kadar velja
%
\begin{equation*}
  k^2 \leq n < (k+1)^2.
\end{equation*}
%
To prirejanje je veljavno, če je celovito in enolično, česar ne bomo preverjali, lahko pa
poskusite sami. Nekaj prirejanj $f$ prikazuje naslednja razpredelnica:
%
\begin{align*}
0 &\mapsto 0   &   4 &\mapsto 2   &    8  &\mapsto 2   &   12 &\mapsto 3 \\
1 &\mapsto 1   &   5 &\mapsto 2   &    9  &\mapsto 3   &   13 &\mapsto 3 \\
2 &\mapsto 1   &   6 &\mapsto 2   &    10 &\mapsto 3   &   14 &\mapsto 3 \\
3 &\mapsto 1   &   7 &\mapsto 2   &    11 &\mapsto 3   &   15 &\mapsto 3
\end{align*}
%
Ali znate z besedami opisati preslikavo~$f$?

V splošnem je lahko preslikava podana s precej zapleteno konstrukcijo, ki zahteva veliko
preverjanja in dokazovanja. Osnovne načine podajanja preslikav bomo spoznali skupaj s
konstrukcijami množic.


\subsection{Aplikacija in substitucija}
\label{sec:aplikacija-in-subsitucija}

Do sedaj smo se ukvarjali s tem, kako preslikavo podamo, zdaj pa se vprašajmo, kako lahko
preslikavo uporabimo. Če je $f : X \to Y$ preslikava iz $X$ v $Y$ in je $x \in X$, potem
lahko \df{$f$ uporabimo na $x$} in dobimo \df{vrednost} preslikave~$f$ pri
\df{argumentu}~$x$, to je tisti edini element $Y$, ki ga~$f$ priredi~$x$. Vrednost $f$
pri~$x$ zapišemo
%
\begin{equation*}
  f(x)
  \qquad\text{ali}\qquad
  f\,x
\end{equation*}
%
in preberemo ``$f$ od $x$'' ali ``$f$ pri $x$''. Izraz $f(x)$, oziroma $f\,x$, se imenuje
\df{aplikacija}. Večinoma se uporablja zapis z oklepaji, a ne vedno: navajeni smo pisati
$\ln 2$ in $\sin \alpha$ namesto $\ln(2)$ in $\sin(\alpha)$. Oklepaje izpuščamo tudi v
nekaterih programskih jezikih in občasno v algebri.

V analizi je uveljavljen še en zapis za aplikacijo, ki se uporablja za zaporedja. Namreč,
zaporedje ni nič drugega kot preslikava $a : \NN \to \RR$ iz naravnih v realna števila.
Aplikacijo $a(n)$, ki označuje $n$-ti člen zaporedja, ponavadi pišemo~$a_n$, torej
argument podpišemo.

Preslikavo lahko uporabimo na argumentu tudi, če je nismo poimenovali. Na primer,
preslikavo $\RR \to \RR$, podano s funkcijskim predpisom
%
\begin{equation*}
  x \mapsto 1 + x^2
\end{equation*}
%
uporabimo na argumentu~$3$:
%
\begin{equation*}
  (x \mapsto 1 + x^2)(3).
\end{equation*}
%
Se vam zdi tak zapis nenavaden? Verjetno, a pomislite, zakaj je tako: ker običajno
preslikave poimenujemo in se nanje vedno sklicujemo z njihovim imenom. Prav nobenega
razloga ni, da ne bi s funkcijskimi predpisi delali tako, kot s števili, vektorji in
ostalimi matematičnimi objekti, na katere smo že navajeni. Računalničarji radi rečejo, da
je treba tudi preslikave obravnavati kot ``enakopravne državljane''. Prav imajo, zato
bomo vadili uporabo funkcijskih predpisov ter z njimi delali, kot da niso nič posebnega,
saj niso!

Kako pravzaprav določimo vrednost funkcije pri danem argumentu? To je odvisno od tega,
kako je podano prirejanje. Če imamo tabelarični prikaz, poiščemo argument v levem stolpcu
in pogledamo v desni stolpec. Če je preslikava podana s funkcijskim predpisom, argument
vstavimo v predpis. Na primer, če je $f : \RR \to \RR$ podana s funkcijskim predpisom
%
\begin{equation*}
  f(x) = 1 + x^2,
\end{equation*}
%
potem je vrednost $f(3)$ enaka $1 + 3^2$, kar je seveda enako~$10$, a to zahteva dodaten
račun, ki nas v tem trenutku ne zanima. Pravimo, da smo simbol~$x$ \df{zamenjali} ali
\df{substituirali} s~$3$, oziroma da smo~$3$ \df{vstavili} v~$f$ namesto~$x$. Seveda lahko
vstavimo argument neposredno v funkcijski predpis, zato je aplikacija
%
\begin{equation*}
  (x \mapsto 1 + x^2)(3)
\end{equation*}
%
seveda spet enaka $1 + 3^2$.

Preslikavo smemo uporabiti na poljubnem elementu domene, ki je lahko zapisan na bolj ali
manj zapleten način, pri čemer gre še vedno samo za zamenjavo. Na primer, v zgornjo
preslikavo~$f$ lahko vstavimo $3 + 4$ in dobimo $1 + (3 + 4)^2$ ali pa za neki $u \in \RR$
vstavimo $u + 2$ in dobimo $1 + (u + 2)^2$. V razdelku~\ref{sec:eksponent} bomo spoznali
še dodatna pravila za vstavljanje izrazov, ki se vrtijo okoli vezanih spremenljivk.


\subsection{Načelo ekstenzionalnosti preslikav}

Kot smo že omenili, je možih več pogledov na preslikave. Ali je pomembno, kako učinkovito
računamo vrednosti preslikave? Vsekakor, ampak ali naj to pomeni, da sta preslikavi
različni, če imata enake vrednosti, a je ena podana z učinkovitim pravilom in druga z
neučinkovitim? V matematiki je odgovor nikalen.

\begin{pravilo}[Ekstenzionalnost preslikav]
  Preslikavi sta enaki, če imata enaki domeni in kodomeni ter imata za vse argumente
  enaki vrednosti.
\end{pravilo}

Natančneje, če sta $f : A \to B$ in $g : C \to D$ preslikavi in velja $A = C$, $B = D$ ter
za vsak $x \in A$ velja $f(x) = g(x)$, tedaj velja $f = g$.

Takoj opozorimo na razliko med
%
\begin{equation*}
  f(x) = g(x)
  \qquad\text{in}\qquad
  f = g
\end{equation*}
%
saj bi marsikdo trdil, da med njima ni razlike. Levi izraz pravi, da sta $f(x)$ in $g(x)$
enaka elementa množice $C$, desni pa da sta $f$ in~$g$ enaki preslikavi iz $A$ v $B$. Na
sploh je treba razlikovati med $f$ in $f(x)$, saj to nikakor nista enaka objekta: prvi je
preslikava, drugi pa vrednost te preslikave pri~$x$. Verjetno nihče ne bi trdil, da je
preslikava $\cos$ isto kot $\cos \frac{\pi}{4}$, ali ne? Isti razmislek veleva, da
$\cos x$ ni isto kot $\cos$, če tudi si mislimo, da je $x$ poljuben. Zmeda izhaja iz
neprimernega zapisa preslikav. Če bi že od malih nog pravilno uporabljali funkcijske
predpise, bi seveda vedeli, da načelo ekstenzionalnosti za preslikave zagotavlja enakost
~$\cos$ in $x \mapsto \cos x$, oba pa sta različna od $\cos x$, ki sploh ni preslikava,
ampak neko realno število. Čeprav je število $\cos x$ odvisno od parametra~$x$, je še
vedno le število.

V bran tradicionalnemu zapisu pa moramo vseeno povedati, da se lahko \emph{dogovorimo} za
nekoliko napačen zapis, če to ne povzroča zmede. S tem se izognemu preveč birokratskemu
pisanju nebistvenih podrobnosti in lahko bistveno izboljšamo komunikacijo in razumevanje
med izkušenimi matematiki. A začetnikom priporočamo, da v dobrobit boljšega razumevanja
snovi vsaj na začetku študija raje vztrajajo pri doslednem zapisu.

Vrnimo se še k načelu ekstenzionalnosti preslikav. Ali ni pravzaprav očitno, da sta
preslikavi enaki, če imata enaki domeni, kodomeni in vrednosti? Morda res, a to ni razlog,
da tega ne bi eksplicitno zapisali. Vsak matematik vam ve povedati kako zgodbo o tem,
kako se je v dokazu skrivala napako ravno tam, kjer je bilo nekaj ``očitno''. Poleg tega
pa si lahko predstavljamo razmere, v katerih je smiselno razlikovati med dvema
preslikavama, ki imata vedno enake vrednosti, denimo v programiranju, kjer je učinkovitost
zelo pomembna.

%% STAR MATERIAL OD DAVORINA. Preveriti, kaj od tega je treba dati v besedilo, in kam.

% Množice ne obstajajo ločene ena od druge pač pa so med sabo povezane s
% \df{preslikavami} oziroma s tujko \df{funkcijami}.  Posamična preslikava slika elemente ene
% množice po določenem predpisu v elemente druge množice.

% Če je $f$ preslikava, ki slika iz množice $X$ v množico $Y$, to zapišemo
% %
% \begin{equation*}
%   f : X \to Y.
% \end{equation*}
% %
% Rečemo, da je množica~$X$ \df{začetna množica} ali \df{domena} preslikave~$f$, množica~$Y$
% pa je \df{ciljna množica} ali \df{kodomena} preslikave $f$.


% Običaj je, da predpis preslikave podamo s pomočjo spremenljivke, tipično z oznako $x$. Na primer, če je $f$ preslikava kvadriranja, njen predpis zapišemo kot
% \[f(x) = x^2.\]
% Na tem mestu je potrebno poudariti več reči.
% \begin{itemize}
% \item
% Velikokrat površno rečemo, da zgornji predpis podaja preslikavo. To ni povsem res --- to je zgolj predpis preslikave. Za to, da preslikavo v celoti podamo, je potrebno navesti tri stvari: poleg predpisa še domeno in kodomeno. Vse to je del informacije o preslikavi.

% To se jasno pokaže, če začnemo razmišljati o lastnostih preslikav. Se še spomnite iz srednje šole, kaj pomeni, da je preslikava surjektivna? (Bomo ponovili v razdelku~\ref{razdelek:injektivnost-in-surjektivnost}.) Če vzamemo, da preslikava $f$ zadošča zgornjemu predpisu in jo obravnavamo kot preslikavo $f\colon \RR \to \RR$, ni surjektivna, če jo obravnavamo recimo kot preslikavo $f\colon \RR_{\geq 0} \to \RR_{\geq 0}$, pa je.
% \item
% Za spremenljivko $x$ velja isto, kot smo razpravljali že v prejšnjem razdelku pri lastnostih elementov množic: spremenljivka $x$ nima vnaprej določene vrednosti, pač pa predstavlja mesto, kamor lahko vstavimo poljubno vrednost. Seveda je potem vseeno, če vzamemo kakšno drugo črko ali čisto drug simbol: $f(y) = y^2$ določa isti predpis kot $f(x) = x^2$; prav tako $f(\heartsuit) = \heartsuit^2$. Se pravi, tudi v tem primeru gre za nemo spremenljivko. Če si torej izberemo neko vrednost, jo lahko vstavimo na mesto spremenljivke in izračunamo vrednost dobljenega izraza, npr.~$f(3) = 3^2 = 9$ oziroma $f(2\pi) = (2\pi)^2 = 4\pi^2$. Predstavljajte si, da je spremenljivka pravzaprav škatlica, kamor lahko vstavite vrednost, torej
% \[f(\argbox) = \argbox^2.\]
% \item
% Alternativen način zapisa $f(x) = x^2$ je
% \[f\colon x \mapsto x^2.\]
% Pazimo: navadna puščica $\to$ podaja domeno in kodomeno, kot razloženo zgoraj. Repata puščica $\mapsto$ pa za posamičen element domene pove, v kateri element kodomene se preslika.

% Zapis z repato puščico je še posebej uporaben, kadar želimo podati preslikavo, ne da bi nam bilo potrebno izbrati ime zanjo. Na primer, realno funkcijo kvadriranja lahko v celoti podamo takole:
% \begin{align*}
% \RR &\to \RR \\
% x &\mapsto x^2
% \end{align*}
% (prva vrstica pove domeno in kodomeno, druga pa predpis). Tako podanim preslikavam potem rečemo \df{brezimne preslikave} (s tujko \df{anonimne funkcije}). Kasneje (v razdelku~\ref{razdelek:brezimne-preslikave}) bomo spoznali bolj strnjen zapis takih preslikav, ki je še posebej primeren za izvajanje operacij med preslikavami; takrat bomo takšno funkcijo zapisali kot $\lam{x \in \RR} x^2$.
% \end{itemize}

% \note{Sklop (kompozicija, kompozitum) preslikav. Identiteta kot enota za sklapljanje. Razčlenitev (dekompozicija, faktorizacija) preslikav.}

% \davorin{Definirati moramo tudi oznako $\set{f(x)}{x \in X}$, kar je druge vrste oznaka kot prej definirana $\set{x \in X}{\phi(x)}$. Se gremo primerjavo s Pythonom (razlika med \texttt{\{f(x) for x in X\}} in \texttt{\{x if phi(x)\}})? Smo matematični hipsterji in uvedemo oznako $\{f(x) \,|\, x \in X \,|\, \phi(x)\}$, ki ustreza \texttt{\{f(x) for x in X if phi(x)\}}, kar bi tudi prišlo prav?}

% Zaenkrat smo imeli primere, ko je bil prepis preslikave dan z eno samo spremenljivko, npr.~$f(x) = x^2$. Zelo pogoste so pa tudi \df{preslikave več spremenljivk}, npr.~$f(x, y) = x^2 + y^2$. Že osnovne računske operacije so take --- na primer, pri seštevanju vzamemo \emph{dva} podatka in vrnemo rezultat (vsoto).

% V takem primeru je smiselno reči: domena preslikave sestoji iz \df{dvojic} ali \df{parov} števil. Pri seštevanju je to, katero število je prvo, katero pa drugo, sicer nepomembno, pri kakšni drugi operaciji (npr.~že odštevanju), pa je, zato posebej zahtevajmo: gre za \df{urejene dvojice} (\df{pare}). Urejeno dvojico elementov $a$ in $b$ (v tem vrstem redu) po dogovoru zapišemo kot $(a, b)$. Vrednosti $a$ in $b$ imenujemo \df{komponenti} tega para; natančneje, $a$ je \df{prva komponenta}, $b$ pa \df{druga komponenta}.

% Če imamo dve množici $A$ in $B$, tedaj množico vseh urejenih dvojic, katerih prva komponenta je element iz $A$, druga komponenta pa element iz $B$, označimo $A \times B$ in imenujemo \df{zmnožek} ali \df{produkt} množic $A$ in $B$. Glede na to, da obstaja mnogo operacij, ki se imenujejo ``produkt'' (poznate že vsaj produkt števil, produkt števila z vektorjem, skalarni produkt vektorjev in vektorski produkt vektorjev, obstaja pa jih še precej več), je koristno produkt množic posebej poimenovati, da ga ločimo od drugih: zanj se je uveljavil izraz \df{kartezični produkt} (izhaja iz imena Cartesius, tj.~latinske različice priimka Renéja Descarta\footnote{René Descartes (1596 -- 1650) je bil francoski filozof, matematik in znanstvenik.}).

% Seštevanje potemtakem lahko razumemo kot preslikavo $+\colon \RR \times \RR \to \RR$. V tem smislu še vedno gre za preslikavo, ki dan vhodni podatek preslika v neki rezultat, le da je vhodni podatek dvojica števil, ne pa zgolj eno število. Kadar imamo produkt več enakih faktorjev, ga lahko (kot običajno) zapišemo v obliki potence; pisali bi lahko tudi $+\colon \RR^2 \to \RR$.

% Seveda nismo omejeni na preslikave samo ene ali dveh spremenljivk. Nič nam ne preprečuje definirati recimo $f(x, y, z) = 2x + y - 3z$. Smiselna domena te preslikave setoji iz \df{urejenih trojic} števil. V splošnem, če jemljemo elemente iz množic $A$, $B$, $C$, tedaj se množica vseh takih trojic označi z $A \times B \times C$. Prejšnji predpis določa potem preslikavo $f\colon \RR \times \RR \times \RR \to \RR$ (oziroma krajše $f\colon \RR^3 \to \RR$).

% Spremenljivk je lahko še več; poleg dvojic in trojic tako dobimo še četverice, peterice, šesterice\ldots V splošnem takšna končna zaporedja elementov imenujemo \df{urejene večterice}. Tudi število spremenljivk je lahko označeno s črko; na primer, preslikava, ki računa povprečje $n$ števil (kjer $n \in \NN_{\geq 1}$), je dana kot
% \begin{align*}
% \RR^n &\to \RR \\
% (x_1, x_2, \ldots, x_n) &\mapsto \frac{x_1 + x_2 + \ldots + x_n}{n}
% \end{align*}
% (če hočemo poudariti, da imajo naše večterice natanko $n$ komponent, jih imenujemo $n$-terice). Nadlega pri tem je sicer spet dvoumnost tropičja. Deloma jo je možno odpraviti tako, da celotno večterico označimo z eno spremenljivko. Pogosta izbira zapisa je $f(\mathbf{x})$ ali $f(\vec{x})$ (razlog za to je, da lahko večterico vidimo kot vektor).

% Marsikdaj želimo delati ne samo z eno preslikavo, pač pa s celo množico preslikav naenkrat. Zato uvedemo: množica vseh preslikav, ki slikajo iz $X$ v $Y$, se označi kot $Y^X$; temu se reče \df{eksponent} množic $X$ in $Y$ (\note{na primernem mestu kasneje} bomo razložili, od kod ta oznaka).

% \begin{zgled}
% Množico vseh preslikav, ki realna števila slikajo nazaj v realna števila, označimo z $\RR^\RR$. Če nas zanimajo realne preslikave, ki so definirana samo na intervalu $\intoo{-1}{1}$, opazujemo množico $\RR^{\intoo{-1}{1}}$. Definiramo lahko preslikavo
% \begin{align*}
% \RR^{\intoo{-1}{1}} &\to \RR \\
% f &\mapsto f(0),
% \end{align*}
% ki preslikavam priredi njihovo vrednost v točki $0$. Ta preslikava torej ima za argumente (tj.~vnose) celotne preslikave in ne števila! Sama po sebi je element množice $\RR^{\RR^{\intoo{-1}{1}}}$.
% \end{zgled}

% \begin{zgled}
% Za poljubne množice $A$, $B$, $C$ lahko definiramo sledečo preslikavo, katere argumenti so pari preslikav.
% \begin{align*}
% B^A \times C^B &\to C^A \\
% (f, g) &\mapsto g \circ f
% \end{align*}
% \end{zgled}


% \davorin{Glede na to, da gre za slovenski učbenik, dajem izrazu `preslikava' prednost pred izrazom `funkcija'. Seveda pa sem pojasnil tudi slednji izraz (v prvem poglavju).}

% \note{Uvod. Definicijsko območje in zaloga vrednosti \davorin{morda dodamo kot možno ime za zalogo vrednosti še prevod angleške besede `range', se pravi `razpon'?}. Zožitve (tako domene kot kodomene); oznake za to so $\rstr{f}_A$, $\rstr{f}^B$, $\rstr{f}_A^B$. Izvrednotenje (evalvacija) preslikave (če ne bomo tega pojasnili že pri eksponentih množic).}


\section{Zmnožek}
\label{sec:zmnozek}

Množice lahko \df{tvorimo} ali \df{konstruiramo} iz drugih množic na različne načine. V
tem poglavju bomo spoznali tri osnovne konstrukcije, ostale pa kasneje, ko bomo že nekaj
vedeli o logiki. Najprej obravnavajmo zmnožek ali kartezični produkt.

Takoj se zastavi vprašanje, kako sploh opisati novo konstrukcijo množic. Načelo
ekstenzionalnosti pove, da je množica opredeljena s svojimi elementi. Torej moramo
pojasniti, kaj so elementi nove množice, se pravi, kako jih vpeljemo, kaj lahko z njimi
počnemo in kakšne so njihove zakonitosti. Natančneje, novo konstrukcijo množic
določajo naslednja pravila:
%
\begin{enumerate}
\item pravilo \df{tvorbe}, ki vpelje novo množico,
\item pravila \df{vpeljave} elementov, ki podajo operacije, s katerimi gradimo elemente,
\item pravila \df{uporabe}, ki podajo opreacije, s katerimi razgradimo ali uporabimo elemente,
\item \df{enačbe}, ki opredeljujejo zakonitosti, ki veljajo za operacije vpeljave in uporabe.
\end{enumerate}
%
Najbolje je, da si postopek ogledamo na primeru.

\begin{pravilo}[Tvorba zmnožka]
  \label{pravilo:zmnozek-tvorba}
  Za vsaki množici $A$ in $B$ je $A \times B$ množica, ki se imenuje \df{zmnožek} ali
  \df{kartezični produkt} $A$ in $B$.
\end{pravilo}

\noindent
%
Pravilo tvorbe pove, da lahko tvorimo novo množico $A \times B$, ne pove pa, kakšne
elemente ima. To je vsebina naslednjih dveh pravil, ki povesta, kako sestavimo in
razstavimo elemente zmnožka.

\begin{pravilo}[Vpeljava urejenih parov]
  \label{pravilo:zmnozek-vpeljava}
  %
  Za vse $a \in A$ in $b \in B$ je $(a, b) \in A \times B$. Element $(a, b)$ imenujemo
  \df{urejeni par}.
\end{pravilo}

\begin{pravilo}[Uporaba urejenih parov]
  \label{pravilo:zmnozek-uporaba}
    %
  Za vsak $p \in A \times B$ je $\fst(p) \in A$ \df{prva projekcija} in $\snd(p) \in B$
  \df{druga projekcija} elementa~$p$.
\end{pravilo}

Nazadnje podamo še enačbe.

\begin{pravilo}[Računsko pravilo za urejene pare]
  \label{pravilo:zmnozek-racunanje}
  Za vse $a \in A$, $b \in B$ velja $\fst(a, b) = a$ in $\snd(a, b) = b$.
\end{pravilo}

\begin{pravilo}[Ekstenzionalnost urejenih parov]
  \label{pravilo:zmnozek-ekstenzionalnost}
  Za vse $p, q \in A \times B$ velja: če $\fst(p) = \fst(q)$ in $\snd(p) = \snd(q)$,
  potem $p = q$.
\end{pravilo}

\noindent
%
Računsko pravilo se tako imenuje, ker lahko z njim poenostavljamo izraze, drugo pa je
načelo ekstenzionalnosti, ker pravi, da je urejeni par določen s prvo in drugo projekcijo.

Kadar imamo opravka z večimi zmnožki, na primer $A \times B$ in $C \times D$, bi lahko
prišlo do zmede glede projekcij. Takrat jih opremimo še z dodatnimi oznakami množic, da
razločimo projekciji $\fst[A][B] : A \times B \to A$ in $\fst[C][D] : C \times D \to C$,
in podobno za~$\snd$.

Malo bolj naivna konstrukcija zmnožka bi se glasila takole: kartezični produkt
$A \times B$ je množica vseh urejenih parov $(a, b)$, kjer je $a \in A$ in $b \in B$. A
taka konstrukcija ni popolna, saj ne pove, kaj lahko z urejenim parom počnemo. Kako naj
vemo, da iz $(a, b)$ lahko izluščimo $a$ in $b$, in kako preverimo, ali sta dva urejena
para enaka? Če takih zadev ne določimo, bi lahko kdo mislil, da je urejeni par kaka druga
operacija, denimo seštevanje, unija, ali kdovekaj.

Dejstvo, da je vsak element zmnožka množic urejen par, in to celo na en sam način, lahko
dokažemo.

\begin{trditev}
  Naj bosta $A$ in $B$ množici. Za vsak element $p \in A \times B$ obstaja natanko en
  $a \in A$ in natanko en $b \in B$, da velja $p = (a, b)$.
\end{trditev}

\begin{proof}
  Naj bosta $A$ in $B$ množici in $p \in A \times B$. Najprej pokažimo, da $p$ res je enak
  nekemu urejenemu paru, namreč
  %
  \begin{equation*}
    p = (\fst(p), \snd(p)).
  \end{equation*}
  %
  Uporabimo načelo ekstenzionalnosti za pare, ki nam zagotavlja to enačbo, če dokažemo
  %
  \begin{equation*}
    \fst(p) = \fst(\fst(p), \snd(p))
    \qquad\text{in}\qquad
    \snd(p) = \snd(\fst(p), \snd(p)).
  \end{equation*}
  %
  Ti dve enačbi pa veljata, ker sta primerka računskih pravil za pare.

  Preveriti moramo še, da je $(\fst(p), \snd(p))$ edini urejeni par, ki je enak~$p$.
  Povedano z drugimi besedami, dokazati moramo: če je $p = (a, b)$ za neki $a \in A$ in
  $b \in B$, potem velja $a = \fst(p)$ in $b = \snd(p)$. Pa denimo, da bi za neki
  $a \in A$ in $B \in B$ veljalo $p = (a,b)$. Tedaj bi lahko uporabili računska pravila za
  pare in dobili
  %
  \begin{equation*}
    \fst(p) = \fst(a, b) = a
    \qquad\text{in}\qquad
    \snd(p) = \snd(a, b) = b,
  \end{equation*}
  %
  kar smo želeli dokazati.
\end{proof}

Trditev je prikladna, ko želimo podati funkcijsko pravilo za preslikavo, katere domena je
zmnožek množic. Primer take preslikave je
%
\begin{gather*}
  \RR \times \RR \to \RR \\
  p \mapsto \fst(p) + \snd(p)^2 \cdot \fst(p).
\end{gather*}
%
Ta zapis je precej nepregleden, a sledili smo navodilu, da mora stati na levi strani
funkcijskega predpisa simbol. Prejšnja trditev nam zagotavlja, da lahko vsak element
$\RR \times \RR$ na en sam način izrazimo kot urejeni par $(x, y)$, in zato ne bo nič
narobe, če zapišemo ta isti funkcijski predpis bolj pregledno tako, da upoštevamo, da
je $p$ enak $(x, y)$ za enolično določena $x$ in $y$:
%
\begin{gather*}
  \RR \times \RR \to \RR \\
  (x, y) \mapsto x + y^2 \cdot x.
\end{gather*}
%
Če bi funkcijo poimenovali, denimo $f$, bi dobili običajni zapis:
%
\begin{gather*}
  f : \RR \times \RR \to \RR \\
  f(x, y) = x + y^2 \cdot x.
\end{gather*}
%
Za tako preslikavo pravimo, da je ``funkcija dveh spremenljivk'', ker si mislimo, da smo
podali argumenta $x$ in $y$ ločeno drug od drugega. Tu pravzaprav vidimo, da bi lahko
rekli tudi, da je funkcija dveh spremenljivk pravzaprav običajna funkcija, katere
arugmenti so urejeni pari.

Poleg zmnožka dveh množic bi lahko tvorili tudi zmnožek treh ali več množic. Pravila bodo
podobna kot za zmnožek dveh množic, le da bi namesto urejenih parov tvorili \df{urejene
  večterice} in da bi imeli več projekcij. Za vsako projekcijo bi zapisali eno računsko
pravilo, princip ekstenzionalnosti pa bi bil tudi podoben tistemu za urejene pare.
Podorobnosti prepustimo za vajo.


\section{Vsota}
\label{sec:vsota}

Spoznali smo že unijo $A \cup B$ množic $A$ in $B$, ki vsebuje tiste elemente, ki so v $A$
ali v $B$. Če imata $A$ in $B$ skupne elemente, bodo ti v uniji seveda nastopili samo
enkrat. V skranjem primeru dobimo $A \cup A = A$. Včasih pa želimo združiti množici tako,
da ne pride do prekrivanja. Taka konstrukcija je \df{vsota} $A + B$ množic $A$ in $B$.
Prekrivanje preprečimo tako, da elemente, ki jih je prispevala~$A$ označimo z eno oznako,
tiste, ki jih je prispevala~$B$, pa z drugo.

\begin{pravilo}[Vsota]
  \label{vsota:tvorba}
  Za vsaki množici $A$ in $B$ je $A + B$ množica, ki se imenuje \df{vsota} ali
  \df{koprodukt} množic $A$ in $B$.
\end{pravilo}

\begin{pravilo}[Vpeljava elementov vsote]
  \label{vsota:vpeljava}
  Za vsaki množici $A$ in $B$ velja:
  %
  \begin{enumerate}
  \item za vsak $a \in A$ je $\inl(a) \in A + B$,
  \item za vsak $b \in B$ je $\inr(b) \in A + B$.
  \end{enumerate}
\end{pravilo}

S pravilom vpeljave smo pojasnili, da uporabljamo oznaki $\inl$ in $\inr$, prvo za
elemente iz~$A$ in drugo za elemente iz~$B$. Oznakama pravimo tudi
\df{injekciji}\footnote{Pravzaprav niti ni pomembno, kako poimenujemo oznaki, da sta le
  različni. V funkcijskem programiranju, kjer poznamo vsote podatkovnih tipov, programer
  sam določi, kakšne oznake bo uporabljal za injekcije.} in sta preslikavi
%
\begin{equation*}
  \iota_1 : A \to A + B
  \qquad\text{and}\qquad
  \iota_2 : B \to A + B.
\end{equation*}
%
Kadar imamo opravka z večimi vsotami, na primer $A + B$ in $C + D$, bi lahko prišlo do
zmede glede oznak. Takrat injekcije opremimo še z dodatnimi oznakami množic, da razločimo
injekciji $\inl[A][B] : A \to A + B$ in $\inl[C][D] : C \to C + D$, in podobno za~$\inr$.

Potrebujemo še pravili za uporabo in enakost elementov vsote, ki ju združimo v eno samo
pravilo.

\begin{pravilo}
  \label{vsota:uporaba}
  Za vsaki množici $A$ in $B$ in za vsak $u \in A + B$, bodisi obstaja natanko en
  $a \in A$, da je $u = \inl(a)$, bodisi obstaja natanko en $b \in B$, da je
  $u = \inr(b)$.
\end{pravilo}

Fraza ``bodisi \dots bodisi'' pomeni, da je vsak element $u \in A + B$ enak $\inl(a)$
za natanko en $a \in A$ ali $\inr(b)$ za natanko en $b \in B$, ne more pa se zgoditi oboje
hkrati ali nič od tega. Torej $\inl(a) = \inr(b)$ ne drži in celo v primeru, ko je
$A = B$ in $a = b$, je $\inl(a) \neq \inr(b)$. S tem smo v $A + B$ res ločili elemente $A$
od elementov $B$.
%
Fraza ``natanko en'' pove, da iz $u = \inl(a_1)$ in $u = \inl(a_2)$ sledi $a_1 = a_2$.
Povedano drugače, če velja $\inl(a_1) = \inl(a_2)$, potem je $a_1 = a_2$. Podobno iz
$\inr(b_1) = \inr(b_2)$ sledi $b_1 = b_2$.
%
Podajmo prepost primer, ki verjetno marsikaj pojasni:
%
\begin{equation*}
  \set{a, b, c} + \set{a, d, e} =
  \set{\inl(a), \inl(b), \inl(c), \inr(a), \inr(d), \inr(e)}.
\end{equation*}

Kako definiramo preslikavo $A + B \to C$? Ker je vsak element domene $A + B$ bodisi
$\inl(a)$ za neki $a \in A$ bodisi $\inr(b)$ za neki $b \in B$, \emph{obravnavamo oba
  primera}. Tako funkcijski zapis za preslikavo $A + B \to C$ zapišemo kot
%
\begin{equation*}
  u \mapsto
  \begin{cases}
    \cdots a \cdots & \text{če $u = \inl(a)$,}\\
    \cdots b \cdots & \text{če $u = \inr(b)$,}
  \end{cases}
\end{equation*}
%
kjer smemo v $\cdots a \cdots$ zapisati izraz, ki vsebuje simbol~$a$, in v
$\cdots b \cdots$ izraz, ki vsebuje simbol~$b$. Ker je tak zapis nekoliko neroden, se
dogovorimo, da ga lahko zapišemo tudi s \emph{večdelnim} funkcijskim predpisom:
%
\begin{align*}
  \inl(a) &\mapsto \cdots a \cdots, \\
  \inr(b) &\mapsto \cdots b \cdots.
\end{align*}
%
Če želimo preslikavo poimenovati, zapišemo
%
\begin{align*}
  f &: A + B \to C, \\
  f(\inl(a)) &= \cdots a \cdots \\
  f(\inr(b)) &= \cdots b \cdots.
\end{align*}
%
Vsi ti zapisi res določajo celovito in enolično prirejanje, saj nam pravila za vsoto
zagotavljajo, da vedno obvelja natanko en primer. Na sploh lahko podamo funkcijski zapis z
večimi primeri, če le pazimo, da obravnavamo vse možnosti, in da se le-te ne prekrivajo.
Na primer, predpis
%
\begin{align*}
  (A + B) \times C &\to B + A \\
  (\inl[A][B](a), c) &\mapsto \inr[B][A](a) \\
  (\inr[A][B](b), c) &\mapsto \inl[B][A](b)
\end{align*}
%
je celovit in enoličen, medtem ko predpis
%
\begin{align*}
  (A \times A) + B &\to A \\
  \inl(a_1, a_2) &\mapsto a_2
\end{align*}
%
ni veljaven, ker ni celovit, saj manjka primer $\inr(b) \mapsto \cdots$.

Poleg vsote dveh množic bi lahko tvorili zmnožek treh ali več množic. Pravila bi bila
podobna, le da bi imeli več injekcij in več primerov.

\section{Eksponent}
\label{sec:eksponent}

Denimo, da sta $A$ in $B$ množici. Tedaj lahko obravnavamo preslikave
%
\begin{equation*}
  A \to B
\end{equation*}
%
z domeno $A$ in kodomeno $B$. Ali vse take preslikave tvorijo množico? Russellov paradoks
nas je izučil, da moramo pazljivo postaviti pravila za konstrukcije množic, nato pa jih
strogo držati. Pravila, ki smo jih podali do sedaj, ne zagotavljajo knostrukcije množic
vseh preslikav iz $A$ v $B$. Potrebujemo novo pravilo.

\begin{pravilo}[Eksponent]
  Za vsaki množici $A$ in $B$ ima \df{eksponent} ali \df{eksponentna množica $B^A$} za
  elemente natanko vse preslikave iz~$A$ v~$B$.
\end{pravilo}

Potemtakem je zapis $f : A \to B$ enakovreden zapisu $f \in B^A$.

Pravila, ki opredeljujejo elemente množice $B^A$ smo že spoznali. Pravilo vpeljave pravi,
da je preslikava podana z domeno, kodomeno ter celovitim in enoličnim prirejanjem med
njima. Pravilo uporabe je kar aplikacija: če je $f \in B^A$ in $a \in A$, lahko tvorimo
$f(a) \in B$. Tudi računsko pravilo za preslikave smo že spoznali, saj je to kar pravilo
zamenjave: funkcijski predpis uporabimo na argumentu tako, da vezano spremenljivko v
predpisu zamenjamo z argumentom. In ekstenzionalnost preslikav pove, kdaj sta dve
preslikavi enaki.

Preslikavi, ki sprejme kot argument preslikavo, pravimo \df{funkcional} ali \df{preslikava
  višjega reda}. Primer take preslikave je \df{kompozicija}:
%
\begin{align*}
  {\circ} &: C^B \times B^A \to C^A \\
  {\circ} &: (g, f) \mapsto (x \mapsto g(f(x))).
\end{align*}
%
Pišemo jo kot operacijo, torej $g \circ f$ namesto ${\circ}(g, f)$. V zgornjem zapisu smo
uporabili eksponente, a v tem primeru je bolj pregleden diagram:
%
\begin{equation*}
  \xymatrix{
    {A}
    \ar[r]^{f}
    \ar@/_2ex/[rr]_{g \circ f}
    &
    {B}
    \ar[r]^{g}
    &
    {C}
  }
\end{equation*}
%
Zakaj smo $\circ$ definirali tako, da kompozicijo $f$ in $g$ pišemo $g \circ f$ namesto
$f \circ g$? Ker si je mnogo lažje zapomniti računsko pravilo
%
\begin{equation*}
  (g \circ f)(x) = g(f(x)),
\end{equation*}
%
ki velja z našo definicijo, kot pa $(f \circ g)(x) = g(f(x))$, kar bi veljalo, če bi
zamenjali vlogi~$f$ in~$g$.

\begin{trditev}
  \parbox{0pt}{}
  %
  \begin{enumerate}
  \item Identiteta je nevtralna za kompozicijo: $\id[B] \circ f = f = f \circ \id[A]$.
  \item Kompozicija je asociativna: $(h \circ g) \circ f = h \circ (g \circ f)$.
  \end{enumerate}
\end{trditev}

\begin{proof}
  Trditev je zapisana pomanjkljivo, saj ne piše, kaj so $A$, $B$,$ f$ in~$g$. Avtorja
  trditve bi lahko vprašali, kaj je hotel povedati, a je bolje, da poskusimo to razvozlati
  sami, ker je to odlična vaja iz razumevanja matematičnih besedil.

  Takoj vidimo, da je $A$ množica, sicer zapis $\id[A]$ ne bi bil smiselen, in podobno je
  tudi $B$ množica. Simboli $f$, $g$ in $h$ zagotovo označujejo preslikave, saj nastopajo
  v kompoziciji. Kaj pa njihove domene in kodomene? Preslikava $f$ mora imeti domeno $A$,
  sicer ne bi bilo dovoljeno komponirati $f \circ \id[A]$, in mora imeti kodomeno $B$,
  sicer ne bi bilo dovoljeno komponirati $\id[B] \circ f$. Ostaneta še domeni in kodomeni
  preslikav $g$ in~$h$. Kompozicija $g \circ f$ kaže, da mora biti domena $g$ enaka
  kodomeni~$f$, torej $B$. Kompozicija $h \circ g$ pa pove, da je kodomena $C$ enaka
  domeni $h$. Če vse to zložimo v diagram, dobimo
  %
  \begin{equation*}
    \xymatrix{
      {A} \ar[r]^{f}
      &
      {B} \ar[r]^{g}
      &
      {\text{?}} \ar[r]^{h}
      &
      {\text{?}}
    }
  \end{equation*}
  %
  Trditev moramo razumeti tako, da bo čim bolj splošna in smiselna. Torej bomo za neznani
  množici vzeli kar poljubni množici $C$ in $D$:
  %
  \begin{equation*}
    \xymatrix{
      {A} \ar[r]^{f}
      &
      {B} \ar[r]^{g}
      &
      {C} \ar[r]^{h}
      &
      {D}
    }
  \end{equation*}
  %
  Preverimo, ali smo trditev pravilno razumeli. Ko vstavimo podrobnosti, se prvi del
  glasi: ``Za vse množice $A$ in $B$ ter preslikavo $f : A \to B$ velja
  $\id[B] \circ f = f = f \circ \id[A]$.''
  %
  Ker je to smiselna izjava, jo dokažimo. Enakost preslikav se dokaže z ekstenzionalnostjo
  preslikav, torej preverimo, ali imajo $\id[B] \circ f$, $f$ in $f \circ \id[A]$ enako
  vrednost za poljuben $x \in A$:
  %
  \begin{align*}
    (\id[B] \circ f)(x) &= \id[B] (f (x)) = f (x), \\
    f (x) &= f(x), \\
    (f \circ \id[A])(x) &= f (\id[A](x)) = f(x).\\
  \end{align*}
  %
  Zapišimo podrobno še drugi del: ``Za vse množice $A$, $B$, $C$ in $D$ ter preslikave
  $f : A \to B$, $g : B \to C$ in $h : C \to D$ velja
  $(h \circ g) \circ f = h \circ (g \circ f)$. To spet dokažemo tako, da uporabimo levo in
  desno stran enačbe na poljubnem $x \in A$:
  %
  \begin{align*}
    ((h \circ g) \circ f)(x) &= (h \circ g)(f(x)) = h(g(f(x))) \\
    (h \circ (g \circ f))(x) &= h((g \circ f)(x)) = h(g(f(x))). \qedhere
  \end{align*}
\end{proof}


Kompozicijo smo zapisali z \emph{vgnezdenim} funkcijskim predpisom, ki argumentu priredi
preslikavo, ki je spet podana s funkcijskim predpisom. V splošnem je vgnezdeni funkcijski
predpis oblike
%
\begin{align*}
  A &\mapsto C^B \\
  a &\mapsto (b \mapsto \cdots),
\end{align*}
%
kjer se lahko v $\cdots$ pojavita~$a$ in~$b$. Na tak zapis se je treba navaditi, a je zelo
prikladen, še posebej v funkcijskem programiranju. V matematiki ni zelo pogost, a mi se ga
ne bomo bali.

Pri računanju s preslikavami višjega reda včasih hkrati obravnavamo več funkicjskih
predpisov in lahko pride do zmede, če za vse uporabimo isto vezano spremenljivko. Na
primer, kompozitum preslikav
%
\begin{equation*}
  \begin{aligned}
    \RR &\to \RR \\
    x &\mapsto x^2 - 4
  \end{aligned}
  %
  \qquad\text{in}\qquad
  %
  \begin{aligned}
    \RR &\to \RR \\
    x &\mapsto 2 - x
  \end{aligned}
\end{equation*}
%
bi lahko izračunali takole:
%
\begin{align*}
  (x \mapsto x^2  - 4) \circ (x \mapsto 2 - x)
  &= (x \mapsto (x \mapsto x^2 - 4) ((x \mapsto 2 - x) x)) \\
  &= (x \mapsto (x \mapsto x^2 - 4) (2 - x)) \\
  &= (x \mapsto (2 - x)^2 - 4) \\
  &= (x \mapsto x^2 - 4 x).
\end{align*}
%
Tu imamo tri pojavitve $x$, ki bi jih morali ločiti, ker vsaka nastopa kot vezana
spremenljivka v svojem funkcijskem predpisu. Še posebej nejasen je računski korak
$(x \mapsto (x \mapsto x^2 - 4) (2 - x)) = (x \mapsto (2 - x)^2 - 4)$, ko vezano
spremenljivko~$x$ v funkcijskem predpisu zamenjamo z izrazom $2 - x$, ki tudi vsebuje~$x$.
To sta dva različna $x$-a! Spomnimo se, da lahko vezane spremenljivke vedno preminujemo.
Ponovimo račun, a tokrat tako, da imajo različni funkcijski predpisi različne vezane
spremenljivke. Kompozitum
%
\begin{equation*}
  \begin{aligned}
    \RR &\to \RR \\
    y &\mapsto y^2 - 4
  \end{aligned}
  %
  \qquad\text{in}\qquad
  %
  \begin{aligned}
    \RR &\to \RR \\
    z &\mapsto 2 - z
  \end{aligned}
\end{equation*}
%
izračunamo takole:
%
\begin{align*}
  (y \mapsto y^2  - 4) \circ (z \mapsto 2 - z)
  &= (x \mapsto (y \mapsto y^2 - 4) ((z \mapsto 2 - z) x)) \\
  &= (x \mapsto (y \mapsto y^2 - 4) (2 - x)) \\
  &= (x \mapsto (2 - x)^2 - 4) \\
  &= (x \mapsto x^2 - 4 x).
\end{align*}
%
To je dosti bolj pregledno. Da ne bo prihajalo do zapletov z vezanimi spremenljivkami, se
dogovorimo: \emph{kadar imamo opravka z večimi vezanimi spremenljivkami, jih vedno
  preimenujemo tako, da so med seboj različne.}

Funkcionale srečamo v analizi in funkcijskem programiranju. Limita zaporedja je
funkcional, ker sprejme kot argument zaporedje realnih števil, se pravi element $\RR^\NN$,
in mu priredi realno število. Odvod je funkcional, ki sprejme element $\RR^\RR$ in mu
priredi element $\RR^\RR$. Če smo povsem natančni, limita kot preslikava $\RR^\NN \to \RR$
ni celovit funkcional, ker nekatera zaporedja ne konvergirajo. Prav tako odvod kot
preslikave $\RR^\RR \to \RR^\RR$ ni celovit, ker nekatere preslikave niso odvedljive.
Preslikavam, ki niso celovite, pravimo \emph{delne} in o njih bomo več povedali v
razdelku~\ref{delne-preslikve}.

\section{Izomorfizem množic}
\label{sec:izomorfizem-mnozic}

Ko otrok prvič spozna pojem števila, je ta zanimiv sam po sebi. Z vnemo šteje do sto in se
rad pogovarja se o tem, koliko je en miljon. Sčasoma se radovednost osredotoči na
aritmetične operacije in, če ima mladenič ali mladenka v sebi matematično žilico, na
\emph{zakonitosti} števil: množenje z~$1$ nima učinka, vrstni red seštevanja ni pomemben
itd. Ali tudi operacijam na množicah, ki smo jih spoznali do sedaj, vladajo kakšne
podobne zakonitosti?

Za števili $a$ in $b$ velja $a \cdot b = b \cdot a$. Nekaj podobnega velja tudi za množici
$A$ in $B$ in njuna zmnožka $A \times B$ in $B \times A$. V splošnem sicer nista enaka, a
sta v nekem smislu enakovredna, ker lahko par $(x, y) \in A \times B$ pretvorimo v par
$(y, x) \in B \times A$ in obratno. Ta razmislek vodi do pojma izomorfizma.

% Pojasnilo: izomorfnost $A \ism B$ je struktura, ki je naravno podana z dvema
% preslikavama $A \to B$ in $B \to A$ ter dvema enačbama med njima. Zato tu zapišemo
% definicijo, ki hkrati uvede vse te pojme.

\begin{definicija}
  Množici $A$ in $B$ sta \df{izomorfni} in pišemo $A \ism B$, kadar obstajata preslikavi
  %
  \begin{equation*}
    f : A \to B
    \qquad\text{in}\qquad
    g : B \to A,
  \end{equation*}
  %
  za kateri velja
  %
  \begin{equation*}
    g \circ f = \id[A]
    \qquad\text{in}\qquad
    f \circ g = \id[B].
  \end{equation*}
  %
  Pravimo, da je~$f$ \df{izomorfizem} med~$A$ in~$B$ in da je~$g$ \df{inverz} ali
  \df{obrat}~$f$.
\end{definicija}

Preverimo, da velja $A \times B \cong B \times A$ za poljubni množici $A$ in $B$. To
storimo tako, da zapišemo preslikavi med zmnožkoma in preverimo, da tvorita
izomorfizem:\footnote{Držimo se pravila, da nikoli ne uporabimo iste vezane spremenljivke
  dvakrat, zato pravilo za $f$ zapišemo z $x$ in $y$ in pravilo za $g$ z $v$ in $u$.
  Marsikdo bi oba funkcisjka predpisa zapisal z $x$ in $y$, torej
  $f : (x, y) \mapsto (y, x)$ in $g : (y, x) \mapsto (x, y)$. To zmede nekatere študente,
  ker mislijo, da ``sta je $x$ v definiciji $f$ isti kot v definiciji $g$'', karkoli že
  naj bi to pomenilo. Poudarimo še enkat: vezana spremenljivka v funkcijskem predpisu nima
  nikakršne zveze z nobeno drugo pojavitvijo iste spremenljivke kje druge.}
%
\begin{align*}
  f &: A \times B \to B \times A &
  g &: B \times A \to A \times B \\
  f &: (x, y) \mapsto (y, x) &
  g &: (v, u) \mapsto (u, v).
\end{align*}
%
Treba je preveriti, da velja $g \circ f = \id[A \times B]$ in
$f \circ g = \id[B \times A]$. To naredimo z uporabo ekstenzionalnosti preslikav, ki pravi
da $g \circ f = \id[A]$ velja, če velja $(g \circ f)(a,b) = \id[A](a,b)$ za vse $a \in A$
in $b \in B$, in podobno za $f \circ g$. Obravnavajmo torej poljubna $a \in A$ in
$b \in B$ in izračunajmo:
%
\begin{equation*}
  (g \circ f)(a, b) =
  g (f (a, b)) = g (b, a) = (a, b).
\end{equation*}
%
Na podoben način preverimo $f \circ g = \id[B \times A]$.

\begin{zgled}\label{zgled:logaritmiranje-je-obratno-od-eksponenciranja}
  Primere izmorfizmov poznamo že iz srednje šole. Naj bo $\RR$ množica vseh realnih števil
  (glej razdelek~\ref{sec:realna-stevila}) in $\RR_{>0}$ množica vseh pozitivnih realnih
  števil. Tedaj logaritem in eksponentna funkcija,
  %
  \begin{equation*}
    \log : \RR_{>0} \to \RR
    \qquad\text{in}\qquad
    \exp : \RR \to \RR_{>0}
  \end{equation*}
  %
  tvorita izomorfizem, saj za $x \in \RR$ velja $\log (\exp x) = x$ in za $y \in \RR_{>0}$
  velja $\exp (\log y) = y$. Eksponentna funkcija seštevanje slika v množenje:
  $\exp 0 = 1$ in $\exp (x + y) = \exp x \cdot \exp y$, zato ni samo izomorfizem med
  množicama, ampak celo izomorfizem med grupama $(\RR, {+}, 0)$ in
  $(\RR_{>0}, {\cdot}, 1)$.

  Če ne veste, kaj je grupa in izomorfizem grup, nikar ne obupavajte. Vsak matematik se v
  vsakdanjem delu nenehno srečuje z neznanimi pojmi. Veste, da je znameniti profesor
  France Križanič\footnote{France Križanič (1928--2002), slovenski matematik} v enega od
  svojih učbenikov zapisal, da naj tisti, ki mu je branje dokazov odveč, ravna tako kot Du
  Fu:\footnote{Du Fu (712--770 pr.~n.~š), kitajski pesnik}
  %
  \begin{center}
    \begin{tabular}{l}
      Ko berem knjige,\\
      z vinom se krepčam\\
      in znak preskočim,\\
      če ga ne poznam.
    \end{tabular}
  \end{center}
\end{zgled}

\begin{naloga}
  Odkorakajte v knjižnico, izposodite si knjigo profesorja Križaniča in jo preberite.
\end{naloga}

Dokažimo nekaj osnovih lastnosti izmorfnosti in izomorizmov. Tokrat ne bomo zapisali
podrobnih dokazov. Za vajo jih dopolnite do tolikšnih podrobnosti, da boste sami sebe
prepričali, da trditve držijo.

\begin{trditev}
  Če je $f : A \to B$ izomorfizem med množicama $A$ in $B$ ter sta preslikavi
  $g : B \to A$ in $h : B \to A$ obe obrata~$f$, potem je $g = h$.
\end{trditev}

\begin{proof}
  Ker je $g$ obrat $f$, velja
  %
  \begin{equation*}
    g \circ f = \id[A]
    \qquad\text{in}\qquad
    f \circ g = \id[B],
  \end{equation*}
  %
  in ker je $h$ obrat $f$, velja
  %
  \begin{equation*}
    h \circ f = \id[A]
    \qquad\text{in}\qquad
    f \circ h = \id[B].
  \end{equation*}
  %
  Dokazati moramo, da iz teh štirih predpostavk sledi $g = h$, kar storimo z naslednjim
  računom:
  %
  \begin{align*}
    g
    &= \id[A] \circ g \tag{kompozicija z $\id[A]$ nima učinka} \\
    &= (h \circ f) \circ g \tag{predpostavka $h \circ f = \id[A]$} \\
    &= h \circ (f \circ g) \tag{kompozicija je asociativna} \\
    &= h \circ \id[B] \tag{predpostavka $f \circ g = \id[B]$} \\
    &= h. \tag{kompozicija z $\id[B]$ nima učinka}
  \end{align*}
\end{proof}

Če je $f : A \to B$ izomorfizem, potem ima natanko en obrat, ki ga označimo $\inv{f}$. Če
$f$ ni izomorfizem, zapis $f^{-1}$ ni veljaven izraz.

Oznaka za obrat je nekoliko nerodna, ker se prekriva z zapisom za obratno vrednost
števila: če je $x \in \RR$ neničelno realno število, potem je $\inv{x}$ tisto realno
število, za katerega velja $x \cdot \inv{x} = 1$. Torej moramo paziti: če je
$f : \RR \to \RR$ izomorfizem in $x \in \RR$, je $\inv{(f(x))}$ obrat števila $f(x)$,
medtem ko je $\inv{f}(x)$ število, ki ga dobimo, ko obrat preslikave $f$ uporabimo na~$x$.
Sami premislite, kaj je $\inv{(\inv{f}(x))}$.

\begin{naloga}
  Podajte primer izomorfizma $f : \RR \to \RR$ in števila $x \in \RR$, da velja
  $\inv{f}(x) = \inv{(f(x))}$. Nato podajte še primer, ko velja
  $\inv{f}(x) \neq \inv{(f(x))}$.
\end{naloga}

\begin{naloga}
  Ozrimo se še enkrat na dokaz prejšnje trditve. Ali smo uporabili vse štiri predpostavke?
  Zapišite \emph{bolj splošno trditev}, se pravi tako, ki navede samo tiste predpostavke,
  ki jih res potrebujemo v dokazu.
\end{naloga}

\begin{trditev}
  Za vse izmorfizme $f : A \to B$ in $g : B \to C$ velja
  %
  \begin{equation*}
    \inv{(\inv{f})} = f
    \qquad\text{in}\qquad
    \inv{(g \circ f)} = \inv{f} \circ \inv{g}.
  \end{equation*}
\end{trditev}

\begin{proof}
  Dokaz prepuščamo za vajo. Pozor, v desni enakosti se je zamenjal vrstni red $f$ in $g$!
  Nadalje opazimo še to: zapisali smo $\inv{(\inv{f})}$ in $\inv{(g \circ f)}$, ne da bi
  predhodno preverili, ali sta $\inv{f}$ in $g \circ f$ izomorfizma. Torej morate v dokazu
  najprej preveriti, da je sta $\inv{f}$ in $g \circ f$ izomorfizma, če sta $f$ in $g$
  izomorfizma.
\end{proof}

\begin{trditev}
  Za vse množice $A$, $B$ in $C$ velja:
  %
  \begin{enumerate}
  \item $A \ism A$,
  \item če $A \ism B$, potem $B \ism A$,
  \item če $A \ism B$ in $B \ism C$, potem $A \ism C$.
  \end{enumerate}
\end{trditev}

\begin{proof}
  \parbox{0pt}{}
  %
  \begin{enumerate}
  \item $\id[A]$ je izomorfizem iz $A$ v $A$, ki je sam svoj obrat,
  \item če je $f : A \to B$ izomorfizem iz $A$ v $B$, potem je $\inv{f}$ izomorfizem iz
    $B$ v $A$ in $f$,
  \item če je $f : A \to B$ izomorfizem iz $A$ v $B$ in $g : B \to C$ izomorfizem iz $B$ v
    $C$, potem je $g \circ f$ izomorfizem iz $A \to C$. \qedhere
  \end{enumerate}
\end{proof}

\begin{trditev}
  Preslikava ima največ en inverz.
\end{trditev}

\begin{naloga}
  Pogosto rečemo, da sta seštevanje in odštevanje obratni operaciji. Strogo vzeto, ti dve
  operaciji nista obratni kot preslikavi, saj obe slikata (recimo, da ju gledamo na
  realnih številih) $\RR \times \RR \to \RR$, tj.~ne slikata v nasprotnih smereh. Ugotovi,
  v kakšnem smislu točno sta seštevanje in odštevanje obratni, tj.~kateri dve preslikavi
  sta pravzaprav druga drugi obratni.
\end{naloga}

% TODO Izomorfnost je kongruenca za produkt, vsoto in eksponent.

\section{Algebra množic}
\label{sec:algebra-mnozic}

Kot že veste, seštevanje, množenje in potenciranje števil zadoščajo naslednjim
algebrajskim zakonom:
%
\begin{align*}
  a + 0 &= a                   &     a \cdot 1 &= a \\
  a + b &= b + a               &     a \cdot b &= b \cdot a \\
  a + (b + c) &= (a + b) + c   &     a \cdot (b \cdot c) &= (a \cdot b) \cdot c \\[1ex]
  0 \cdot a &= 0                           &   1^a &= 1 \\
  (a + b) \cdot c &= a \cdot c + b \cdot c &   (a \cdot b)^c &= a^c \cdot b^c \\[1ex]
  a^0 &= 1                     &   a^1 &= a \\
  a^{b + c} &= a^b \cdot a^c   &   a^{b \cdot c} &= (a^b)^c \\[1ex]
  0^a &= 0 \quad \text{če $a \neq 0$.}
\end{align*}
%
Že prej smo opazili, da je zakon $a \cdot b = b \cdot a$ podoben izomorfizmu
$A \times B \ism B \times A$. Kaj pa ostali zakoni?

\begin{izrek}
  \label{izrek:algebra-mnozic}
  Za vse množice $A$, $B$ in $C$ velja:
  %
  \begin{align*}
    A + \emptyset &\ism A                   &     A \times \one &\ism A \\
    A + B &\ism B + A               &     A \times B &\ism B \times A \\
    A + (B + C) &\ism (A + B) + C   &     A \times (B \times C) &\ism (A \times B) \times C \\[1ex]
    \emptyset \times A &\ism \emptyset                           &   \one^A &\ism \one \\
    (A + B) \times C &\ism A \times C + B \times C &   (A \times B)^C &\ism A^C \times B^C \\[1ex]
    A^\emptyset &\ism \one                     &   A^\one &\ism A \\
    A^{B + C} &\ism A^B \times A^C   &   A^{B \times C} &\ism (A^B)^C \\[1ex]
    \emptyset^A &\ism \emptyset \quad \text{če $A \neq \emptyset$.}
  \end{align*}
\end{izrek}

Izrek ni sam sebi namen, ampak je v njem nauk: \emph{z množicami lahko računamo}, tako kot
s števili. Preostanek razdelka je posvečen dokazu izreka.

\subsubsection{Asociativnost}
\label{sec:asociativnost}

Za ogrevanje dokažimo asociativnost zmnožkov,
$A \times (B \times C) \ism (A \times B) \times C$. Splošni element
$A \times (B \times C)$ je urejeni par oblike $(x, (y, z))$, kjer je $x \in A$, $y \in B$
in $z \in C$, med tem ko je splošni element $(A \times B) \times C$ oblike $((u, v), w)$,
kjer je $u \in A$, $v \in B$ in $w \in C$. Izomorfizmov ni težko zapisati:
%
\begin{align*}
  f &:  A \times (B \times C) \to (A \times B) \times C &
  g &: (A \times B) \times C \to A \times (B \times C) \\
  f &: (x, (y, z)) \mapsto ((x, y), z) &
  g &: ((u, v), w) \mapsto (u, (v, w)).
\end{align*}
%
Preverimo, da je $g$ obrat $f$. Za vse $x \in A$, $y \in B$ in $z \in C$ velja:
%
\begin{equation*}
  g(f(x, (y, z))) = g((x, y), z) = (x, (y, z))
\end{equation*}
%
in za vse $u \in A$, $v \in B$ in $w \in C$ velja
%
\begin{equation*}
  f(g((u, v), w)) = f(u, (v, w)) = ((u, v), w).
\end{equation*}
%
Tudi asociativnost vsote, $A + (B + C) \cong (A + B) + C$ ni nič bolj zapletena, le da
imamo opravka z injekcijami in obravnavanjem primerov. Najprej zapišimo izomorfizma s
popolnoma natančnim zapisom, kjer vse injekcije opremimo z oznakami množic:
%
\begin{align*}
  f &:  A + (B + C) \to (A + B) + C &
  g &: (A + B) + C \to A + (B + C) \\
  f &: \inl[A][B+C](x)             \mapsto \inl[A+B][C](\inl[A][B](x)) &
  g &: \inl[A+B][C](\inl[A][B](u)) \mapsto \inl[A][B+C](u)\\
  f &: \inr[A][B+C](\inl[B][C](y)) \mapsto \inl[A+B][C](\inr[A][B](y)) &
  g &: \inl[A+B][C](\inr[A][B](v)) \mapsto \inr[A][B+C](\inl[B][C](v))  \\
  f &: \inr[A][B+C](\inr[B][C](z)) \mapsto \inr[A+B][C](z) &
  g &: \inr[A+B][C](w)              \mapsto \inr[A][B+C](\inr[B][C](w))
\end{align*}
%
Isti zapis brez oznak množic je precej bolj čitljiv:
%
\begin{align*}
  f &:  A + (B + C) \to (A + B) + C &
  g &: (A + B) + C \to A + (B + C) \\
  f &: \inl(x)             \mapsto \inl(\inl(x)) &
  g &: \inl(\inl(u)) \mapsto \inl(u)\\
  f &: \inr(\inl(y)) \mapsto \inl(\inr(y)) &
  g &: \inl(\inr(v)) \mapsto \inr(\inl(v))  \\
  f &: \inr(\inr(z)) \mapsto \inr(z) &
  g &: \inr(w)              \mapsto \inr(\inr(w))
\end{align*}
%
Ali vidite, zakaj matematiki cenimo kratek in pregleden zapis? Preveč podrobnosti lahko
zakrije bistvo ideje. Preverjanje, da je $g$ obrat $f$, prepustimo tistim, ki radi veliko
pišejo.

\subsubsection{Preslikave in enojec}
\label{sec:preslikave-enojec}

Preslikavi
%
\begin{align*}
  f &: A \times \one \to A &
  g &: A \to A \times \one \\
  f &: (x, u) \mapsto a &
  g &: y \mapsto (y, \unit)
\end{align*}
%
tvorita izomorfizem $A \times \one \ism A$, saj za vsak $a \in A$ in $t \in \one$ velja,
upoštevaje da so vsi elementi~$\one$ enaki~$\unit$,
%
\begin{equation*}
  g(f(a, t)) = g(a) = (a, \unit) = (a, t)
  \qquad\text{in}\qquad
  f(g(a)) = f(a, t) = a.
\end{equation*}
%
Lahko bi rekli, da je $\one$ nevtralni element za zmnožek \emph{do izomorfizma natančno},
s čimer povemo, da ne velja \emph{enakost} $A \times \one = A$, ampak le
\emph{izomorfizem} $A \times \one \ism A$. Na tem mestu lahko tudi pojasnimo nenavadni
zapis edinega elementa~$\one$. Elementi zmnožka dveh množic so urejene dvojice, zmnožka
treh množic urejene trojice itd. Zmnožek nič množic je nevtralni element za množenje,
torej so njegovi elementi urejen ničterice, oziroma urejena ničterica~$\unit$, ker je ena
sama.

Izomorfizma $A^\one \ism A$ ni težko zapisati:
%
\begin{align*}
  f &: A^\one \to A &
  g &: A \to A^\one \\
  f &: h \mapsto h(\unit) &
  g &: x \mapsto (y \mapsto x)
\end{align*}
%
Preverimo, da je $g$ inverz $f$. Za vsak $x \in A$ velja
%
\begin{equation*}
  f(g(x)) = f(y \mapsto x) = x,
\end{equation*}
%
zato je $f \circ g = \id[A]$. Za vsak $h \in A^\one$ velja
%
\begin{equation*}
  g(f(h)) = g(h(\unit)) = (y \mapsto h(\unit)).
\end{equation*}
%
Ali sta $h$ in $y \mapsto h(\unit)$ enaki preslikavi? Kot vsakič, uporabimo
ekstenzionalnost preslikav, le da je tokrat še posebej preprosta: preslikavi z
domeno~$\one$ sta enaki, če imata enako vrednost pri argumentu $\unit$, saj je to edini
element~$\one$. Torej je $h = (y \mapsto h(\unit))$, saj velja
%
\begin{equation*}
  (y \mapsto h(\unit))(\unit) = h(\unit).
\end{equation*}
%
Izomorfnost $A$ in $A^\one$ pravzaprav pove nekaj zanimivega: preslikave $\one \to A$
lahko obravnavamo kot elemente~$A$ in obratno.


\subsubsection{Preslikave in prazna množica}
\label{sec:presl-prazna-mnozica}

Lotimo se izomorfizmov, v katere je vpletena prazna množica. Tu se ne moremo več zanašati
le na prirojen občutek za logiko, saj s prazno množico nimamo vsakdanjih izkušenj, oziroma
jo obravnavamo kot posebnost. Kako bi odgovorili na vprašanje, ali so vsi elementi prazne
množice praštevila? Pravilni odgovor je ``da''. In hkrati so vsi elementi prazne množice
sestavljena števila. Zakaj je to res bomo spoznali v
razdelku~\ref{sec:logika-prazna-mnozica}, ko bomo podrobno obravnavali pravila sklepanja.
Zaenkrat si zapomnimo, da je pravilna vsaka izjava ``za vse elemente prazne množice velja
\dots''. Pravimo, da je taka izjava \emph{na prazno izpolnjena}

Začnimo z vprašanjem, ali lahko tvorimo kako preslikavo $\emptyset \to A$. Najprej
ugotovimo, da so vse preslikave $\emptyset \to A$ enake. Res, za $f, g : \emptyset \to A$
velja $f = g$ natanko tedaj, ko za vse $x \in \emptyset$ velja $f(x) = g(x)$. A ravnokar
smo povedali, da je vsaka izjava oblike ``za vse $x \in \emptyset$ \dots'' veljavna. Pa
imamo kako preslikavo $\emptyset \to A$? Odgovor je pritrdilen, če lahko podamo kako
celovito in enolično prirejanje med elementi~$\emptyset$ in~$A$. Ker sta celovitost in
enoličnost spet izavi oblike ``za vse $x \in \emptyset$ \dots'', sta na prazno izpolnjena,
zato bo zadoščalo kakršnokoli prirejanje, denimo: nobenemu elementu ne priredimo nobenega
elementa. S tem smo utemeljili naslednjo trditev.

\begin{trditev}
  Za vsako množico $A$ obstaja natanko ena preslikava $\emptyset \to A$.
\end{trditev}

Edini preslikavi $\emptyset \to A$ pravimo \df{prazna preslikava}. S tem smo utemeljili
$A^\emptyset \ism \one$, saj izomorfizem prazni preslikavi priredu
$\unit$, njegov obrat pa priredi $\unit$ prazno preslikavo.

\subsubsection{Izomorfizmi in eksponenti}
\label{sec:izomorfizmi-in-eksponenti}

Nazadnje se posvetimo še zakonu $A^{B \times C} \ism (A^B)^C$.
Preverimo, da preslikavi\footnote{Saj ste se že naučili grške črke, ali ne?}
%
\begin{align*}
  \Lambda &: A^{B \times C} \to (A^B)^C
  &
  \Theta &: (A^B)^C \to A^{B \times C}
  \\
  \Lambda &: f \mapsto (c \mapsto (b \mapsto f(b, c)))
  &
  \Theta &: g \mapsto ((b, c) \mapsto g(c)(b))
\end{align*}
%
tvorita izomorfizem. Za vse $f \in A^{B \times C}$, $x \in B$ in $y \in C$ velja
%
\begin{align*}
  \Theta(\Lambda(f))(x, y)
  &= ((b, c) \mapsto \Lambda(f)(c)(b)) (x, y)  \\
  &= \Lambda(f)(y)(x) \\
  &= (c \mapsto (b \mapsto f(b, c)))(y)(x) \\
  &= (b \mapsto f(b, y)(x) \\
  &= f(x, y),
\end{align*}
%
zato je $\Theta(\Lambda(f)) = f$. Prav tako za vse $g \in (A^B)^C$ in $x \in B$ in
$y \in C$ velja
%
\begin{align*}
  \Lambda(\Theta(g))(y)(x)
  &= (c \mapsto (b \mapsto \Theta(g)(b, c)))(y)(x) \\
  &= (b \mapsto \Theta(g)(b, y))(x) \\
  &= \Theta(g)(x, y) \\
  &= ((b, c) \mapsto g(c)(b)) (x, y) \\
  &= g(y)(x)
\end{align*}
%
in zato $\Lambda(\Theta(g)) = g$.
%
Preslikavi $\Lambda(f)$ pravimo \df{transpozicija} preslikave~$f$, in prav tako preslikavi
$\Theta(g)$ pravimo transpozicija preslikave~$g$.

Izomorfizem $A^{B \times C} \ism (A^B)^C$ je zanimiv, ker pove, da lahko preslikavo dveh
argumentov vedno prevedemo na preslikavo enega argumenta. Natančneje, če je
$f : B \times C \to A$ preslikava dveh argumentov, je njena transpozicija
$\Lambda(f) : C \to A^B$ preslikava enega argumenta, njena vrednost pa je preslikava, ki
pričakuje še en argument. To dejstvo se s pridom izkorišča v funkcijskem programiranju:
namesto, da bi definirali preslikavo $f : B \times C \to A$, ki sprejme urejeni par
$(b, c)$ in vrne vrednost $f(b,c)$, raje definiramo enakovredno preslikavo
$\tilde{f} : B \to C \to A$, ki sprejme $b$ in vrne preslikavo $\tilde{f}(b)$, ta pa
sprejme še $c$ in vrne vrednost $\tilde{f}(b)(c)$.


% TODO Potence A^n in binomski izrek.

% TODO zmnozek, vsota in eksponent zo kongruenca za izomorfizme

% Pri asociativnosti produkta obravnavamo $A_1 \times A_2 \times \cdots \times A_n$ in
% enojec kot produkt nič množic. Podobno za vsote.

% Tu je treba pojasniti, zakaj pišemo $\unit$ za element $\one$.


% \section{Kar je že Davorin napisal}

% Interval realnih števil podamo s krajiščema intervala v oklepajih --- okrogli oklepaji ( ) označujejo odprtost intervala (krajišče ni vključeno v interval), oglati oklepaji [ ] pa zaprtost (krajišče je vključeno). Tako se npr.~interval realnih števil od $0$ do $1$, ki ne vsebuje krajišč, označi z $(0, 1)$, če jih vsebuje, pa z $[0, 1]$.

% Včasih pridejo prav tudi intervali na drugih množicah kot $\RR$. Zato se dogovorimo, da bomo intervale označevali tako, da podamo množico, ob kateri v indeksu zapišemo krajišči v oklepajih, npr.~$\intco[\NN]{1}{5} = \set{1, 2, 3, 4}$. Realna intervala iz prejšnjega odstavka tako zapišemo kot $\intoo{0}{1}$ in $\intcc{0}{1}$.

% Če interval v katero smer gre v nedogled, preprosto zapišemo množico z ustreznim simbolom za urejenost in krajiščem v indeksu. Na primer, $\RR_{> 0}$ označuje množico pozitivnih realnih števil, $\RR_{\geq 0}$ pa množico nenegativnih realnih števil.

% Primerjave med elementi, kot npr.~pravkar podani $>$ in $\geq$, imenujemo \df{relacije} (podrobneje jih bomo spoznali v poglavju~\ref{poglavje:relacije}). Zgornji zapis bomo uporabljali tudi za druge vrste relacij, ne samo za relacije urejenosti. Na primer, množico vseh neničelnih realnih števil zapišemo kot $\RR_{\neq 0}$.

% \davorin{To bi vsaj bil moj predlog. Na ta način se izognemo dvoumnostim (kar je namen). Na primer, kaj pomeni $\forall\, a > 0$? Če zapišemo $\forall\, a \in \NN_{> 0}$ ali $\forall\, a \in \RR_{> 0}$, je jasno. Razlog, da matematiki ``goljufajo'' in pridejo skozi brez tega, je (napol dogovorjena in ponotranjena, ampak arbitrarna) izbira črk; vsak izkušen matematik ve, da $\forall\, \epsilon > 0$ pomeni $\forall\, \epsilon \in \RR_{> 0}$. Dodaten problem je, da kasneje uporabljamo urejene pare, ki jih vsi na naši fakulteti pišejo z okroglimi oklepaji. Poskusimo se izogniti zmedi, ali $(a, b)$ pomeni urejeni par ali odprti interval. Če se ne strinjate, popravite in pustite komentar.}

% Če imamo dan neki element in neko množico, potem pripadnost tega elementa tej množici izrazimo s simbolom $\in$. Na primer, da je štiri naravno število, zapišemo $4 \in \NN$ (beri: ``štiri pripada množici naravnih števil'').

% Elementi množic lahko zadoščajo raznim lastnostim. Na primer, recimo, da $\phi$ označuje lastnost ``biti manj od pet''; to potem zapišemo
% \[\phi(x) \ = \ \ x < 5.\]
% V tem primeru $x$ imenujemo \df{spremenljivka}, saj ne gre za točno določeno vrednost, pač pa predstavlja splošno število (recimo, da se dogovorimo, da s $\phi$ označujemo lastnost na realnih številih).

% Tovrstne lastnosti nam omogočajo, da iz neke množice odberemo elemente z dano lastnostjo in na ta način dobimo novo množico, ki je podmnožica prejšnje. Množico vseh realnih števil, ki so manjša od pet, zapišemo na naslednji način.
% \[\set{x \in \RR}{x < 5}\]
% Seveda, ker je primerjava s števili zelo pogosta lastnost, je uporabno, če uvedemo krajše oznake, ki povejo isto; že prej smo se dogovorili, da tako množico označimo z $\RR_{< 5}$. Za povsem splošne lastnosti pa ne bomo imeli vnaprej dogovorjenih oznak, zato je dobro, da poznamo splošni zapis. Torej, če je $X$ poljubna množica in $\phi$ poljubna lastnost njenih elementov, tedaj podmnožico, ki vsebuje točno tiste elemente množice $X$, ki zadoščajo lastnosti $\phi$, označimo takole.
% \[\set[1]{x \in X}{\phi(x)}\]

% Pri tem se zavedajmo: ni pomembno, da spremenljivko označimo ravno z $x$. Zapis
% \[\set[1]{y \in X}{\phi(y)}\]
% še vedno označuje isto množico. V vsakem primeru gre za množico vseh elementov iz $X$ z lastnostjo $\phi$. Pravzaprav sploh ni nujno, da uporabimo črko; poslužimo se lahko kateregakoli simbola (ki mu nismo predtem že predpisali določenega pomena). Taisto množico lahko zapišemo tudi $\set{\heartsuit \in X}{\phi(\heartsuit)}$.

% Kadar imamo spremenljivko, ki jo lahko preimenujemo, ne da bi spremenili pomen izraza, jo imenujemo \df{nema spremenljivka}. Takšne primere že dobro poznate; na primer, integral $\int_0^1 x^2 \,dx$ se ne spremeni, če preimenujete spremenljivko in zapišete $\int_0^1 y^2 \,dy$.

% \begin{zgled}
% Kako bi zapisali množico vseh sodih naravnih števil? Spomnimo se, da je število sodo, kadar je deljivo z $2$. Za $n \in \NN$ to zapišemo takole: $2 \divides n$ (beri: ``dve deli $n$''). Množica sodih naravnih števil se potem zapiše kot
% \[\set[1]{n \in \NN}{2 \divides n}.\]
% \end{zgled}


\section{Vaje}

\begin{vaja}
Kaj veste povedati o množici~$A$, če zanjo velja, da so vsi njeni elementi enaki?
\begin{resitev}
Množica~$A$ ima kvečjemu en element, tj.~množica~$A$ je bodisi prazna bodisi enojec. Tudi: množica~$A$ je podmnožica kakega enojca oz.~edina preslikava $A \to \one$ je injektivna.
\end{resitev}
\end{vaja}

\begin{vaja}
  Načelo ekstenzionalnosti preslikav bi lahko zapisali tudi takole:
  %
  \begin{quote}
    Preslikavi $f : A \to B$ in $g : C \to D$ sta enaki, če velja $A = C$, $B = D$ in za
    vse $x_1, x_2 \in A$ velja, da iz $x_1 = x_2$ sledi $f(x_1) = g(x_2)$.
  \end{quote}
  %
  Dokažite, da je ta različica enakovredna običajnem načelu ekstenzionalnosti.
\end{vaja}

\begin{vaja}
  Zapišite pravila za zmnožek treh množic. Nato premislite še, kako bi podali pravila za
  zmnožek $n$ množic, kjer je~$n$ naravno število.
\end{vaja}

\begin{vaja}
  Naštejte vse elemente množice $\one + \one + \one$.
\end{vaja}

\begin{vaja}
  Preveri tiste izomorfnosti iz izreka~\ref{izrek:algebra-mnozic}, ki jih v
  razdelku~\ref{sec:algebra-mnozic} nismo utemeljili.
\end{vaja}

%%% Local Variables:
%%% mode: latex
%%% TeX-master: "ucbenik-lmn"
%%% End:


% Pod to vrstico so poglavja, ki še niso pripravljena
% za javno dostopno verzijo
\ifOPTincludeall
\chapter{Logika in pravila sklepanja}
\label{chap:logika}


\textbf{Opomba:} To poglavje je del učbenika v nastajanju in ni povsem v skladu s predavanji. Kljub temu ga vključujem v te zapiske, ker vsebuje precej koristnih nasvetov in misli.

%%%%%%%%%%%%%%%%%%%%%%%%%%%%%%%%%%%%%%%%%%%%%%%%%%%%%%%%%%%%%%%%%%%%%%
\section{Kaj je matematični dokaz?}
\label{sec:kaj-je-dokaz}

V srednji šoli se dijaki pri matematiki učijo, \emph{kako} se kaj
izračuna. Na univerzi imajo študentje matematike pred seboj
zahtevnejšo nalogo: poleg \emph{kako} morajo vedeti tudi \emph{zakaj}.
Od njih se pričakuje, da bodo računske postopke znali tudi utemeljiti,
ne pa samo slediti pravilom, ki jih je predpisal učitelj. Razumeti
morajo dokaze znamenitih izrekov in sami poiskati dokaze preprostih
izjav. Da bi se lažje spopadli s temi novimi nalogami, bomo prvi del
predmeta Logika in množice posvetili matematični infrastrukturi:
izjavam, pra\-vi\-lom sklepanja in dokazom. Učili se bomo, kako pišemo
dokaze, kako jih analiziramo in kako jih sami poiščemo.

Osrednji pojem matematične aktivnosti je \emph{dokaz}. Namen dokaza je
s pomočjo točno določenih in vnaprej dogovorjenih \emph{pravil
  sklepanja} utemeljiti neko matematično \emph{izjavo}. Načeloma mora
dokaz vsebovati vse podrobnosti in natanko opisati posamezne korake
sklepanja, ki privedejo do želene matematične izjave. Ker bi bili taki
dokazi zelo dolgi in bi vsebovali nezanimive podrobnosti, matematiki
običajno predstavijo samo oris ali glavno zamisel dokaza. Izkušenemu
matematiku to zadošča, saj zna oris sam dopolniti do pravega dokaza.
Matematik začetnik seveda potrebuje več podrobnosti. Poglejmo si
primer.

\begin{izrek}
  \label{izr:n3-n-deljivo-3}
  Za vsako naravno število $n$ je $n^3 - n$ deljivo s~$3$.
\end{izrek}

\noindent
Po kratkem premisleku bi izkušeni matematik zapisal:

\begin{quote}
  \begin{dokaz}
    Očitno.
  \end{dokaz}
\end{quote}

\noindent
To ni dokaz, izkušeni matematik nam le dopoveduje, da je (zanj) izrek
zelo lahek in da nima smisla izgubljati časa s pisanjem dokaza.
Začetnik, ki težko razume že sam izrek, bo ob takem ">dokazu"< seveda
zgrožen. Verjetno bo najprej preizkusil izrek na nekaj primerih:
%
\begin{align*}
  1^3 - 1 &= 0,\\
  2^3 - 2 &= 8 - 2 = 6,\\
  3^3 - 3 &= 27 - 4 = 24,\\
  4^3 - 4 &= 64 - 4 = 60.
\end{align*}
%
Res dobivamo večkratnike števila~$3$. Ali smo izrek s tem dokazali? Seveda ne!
Preizkusili smo le štiri primere, ostane pa jih še neskončno mnogo. Kdor
misli, da lahko iz nekaj primerov sklepa na splošno veljavnost, naj v poduk
vzame naslednjo nalogo.

\begin{naloga}
  Ali je $n^2 - n + 41$ praštevilo za vsako naravno število~$n$?
\end{naloga}

\noindent
Ko izkušenega matematika prosimo, da naj nam vsaj pojasni idejo dokaza,
zapiše:

\begin{quote}
  \begin{dokaz}
    Število $n^3 - n$ je zmnožek treh zaporednih naravnih števil.
  \end{dokaz}
\end{quote}

\noindent
To še vedno ni dokaz, ampak samo namig. Starejši študenti matematike pa
bi iz namiga morali znati sestaviti naslednji dokaz:

\begin{quote}
  \begin{dokaz}
    Ker je $n^3 - n = (n-1) \cdot n \cdot (n+1)$, je $n^3 - n$ zmnožek
    treh zaporednih naravnih števil, od katerih je eno deljivo s~$3$,
    torej je tudi $n^3 - n$ deljivo s~$3$.
  \end{dokaz}
\end{quote}

\noindent
Mimogrede, znak {\;\qedsign\;} označuje konec dokaza. Čeprav bi bila
večina matematikov s tem dokazom zadovoljna, bi morali za popoln dokaz
preveriti še nekaj podrobnosti:
%
\begin{enumerate}
\item Ali res velja $n^3 - n = (n-1) \cdot n \cdot (n+1)$?
\item Ali je res, da je izmed treh zaporednih naravnih števil eno
  vedno deljivo s~$3$?
\item Ali je res, da je zmnožek treh števil deljiv s~$3$, če je eno od
  števil deljivo s~$3$?
\end{enumerate}
%
S srednješolskim znanjem algebre ugotovimo, da je odgovor na prvo
vprašanje pritrdilen. Tudi odgovora na drugo in tretje vprašanje sta
očitno pritrdilna, mar ne? To pa ne pomeni, da ju ni treba dokazati.
Nasprotno, zgodovina matematike nas uči, da moramo prav ">očitne"<
izjave še posebej skrbno preveriti.

\begin{naloga}
  Kakšno je tvoje mnenje o resničnosti naslednjih izjav? Vprašaj
  starejše kolege, asistente in učitelje, kaj menijo oni. Ali znajo
  svoje mnenje utemeljiti z dokazi?
  %
  \begin{enumerate}
  \item Sodih števil je manj kot naravnih števil.
  \item Kroglo je mogoče razdeliti na pet delov tako, da lahko iz njih
    sestavimo dve krogli, ki sta enako veliki kot prvotna krogla.
  \item Sklenjena krivulja v ravnini, ki ne seka same sebe, razdeli
    ravnino na dve območji, eno omejeno in eno neomejeno.
  \item S krivuljo ne moremo prekriti notranjosti kvadrata.
  \item Če ravnino razdelimo na tri območja, potem zagotovo obstaja
    točka, ki je dvomeja in ni tromeja med območji.
  \end{enumerate}
\end{naloga}

\noindent
%
Vrnimo s k izreku~\ref{izr:n3-n-deljivo-3}. Če dokaz zapišemo preveč
podrobno, postane dolgočasen in ne\-ra\-zumljiv:

\begin{quote}
  \begin{dokaz}
    Naj bo $n$ poljubno naravno število. Tedaj velja
    %
    \begin{align*}
      n^3 - n
      &= n \cdot n^2 - n \cdot 1 \\
      &= n \cdot (n^2 - 1) \\
      &= n \cdot ((n + 1) \cdot (n - 1)) \\
      &= n \cdot ((n - 1) \cdot (n + 1)) \\
      &= (n \cdot (n - 1)) \cdot (n + 1) \\
      &= (n - 1) \cdot n \cdot (n + 1).
    \end{align*}
    %
    Vidimo, da je $n^3 - n$ zmnožek treh zaporednih naravnih števil.
    Dokažimo, da je eno od njih deljivo s~$3$. Število $n$ lahko
    enolično zapišemo kot $n = 3 k + r$, kjer je $k$ naravno število
    in $r = 0$, $r = 1$ ali $r = 2$. Obravavajmo tri primere:
    %
    \begin{itemize}
    \item če je $r = 0$, je $n = 3 k$, zato je $n$ deljiv s~$3$,
    \item če je $r = 1$, je $n - 1 = (3 k + 1) - 1 = 3 k + (1 - 1) = 3
      k + 0 = 3 k$, zato je $n-1$ deljiv s~$3$,
    \item če je $r = 2$, je $n + 1 = (3 k + 2) + 1 = 3 k + (2 + 1) = 3
      k + 3 = 3 k + 3 \cdot 1 = 3 (k +1)$, zato je $n+1$ deljiv s~$3$.
    \end{itemize}
    %
    Vemo torej, da je $n - 1$, $n$ ali $n + 1$ deljiv s~$3$.
    Obravnavamo tri primere:
    %
    \begin{itemize}
    \item Če je $n - 1$ deljiv s~$3$, tedaj  obstaja naravno število
      $k$, da je $n - 1 = 3 k$. V tem primeru je $(n - 1) n (n + 1) =
      (3 k) n (n + 1) = 3 (k n (n + 1))$, zato je $(n - 1) n (n + 1)$
      deljivo s~$3$.
    \item Če je $n$ deljiv s~$3$, tedaj obstaja naravno število $k$,
      da je $n = 3 k$. V tem primeru je $(n - 1) n (n + 1) = (n - 1)
      (3 k) n (n + 1) = (3 k) (n - 1) (n + 1) = 3 (k (n - 1) (n +
      1))$, zato je $(n - 1) n (n + 1)$ deljivo s~$3$.
    \item Če je $n + 1$ deljiv s~$3$, tedaj obstaja naravno število
      $k$, da je $n + 1 = 3 k$. V tem primeru je $(n - 1) n (n + 1) =
      (n - 1) n (3 k) = (n - 1) (3 k) n = (3 k) (n - 1) n = 3 (k (n -
      1) n)$, zato je $(n - 1) n (n + 1)$ deljivo s~$3$.
    \end{itemize}
    %
    V vsakem primeru je $(n - 1) n (n + 1)$ deljivo s~$3$. Ker smo
    dokazali, da je $n^3 = n = (n - 1) n (n + 1)$, je tudi $n^3 - n$
    deljivo s~$3$.
  \end{dokaz}
\end{quote}

\begin{naloga}
  S kolegi se igraj naslednjo igro.\footnote{%
    Igranje odsvetujemo zunaj prostorov Fakultete za matematiko in fiziko.}
  Prvi igralec v zgornjem dokazu poišče korak, ki ga je treba še dodatno
  utemeljiti. Drugi igralec ga utemelji. Nato prvi igralec poišče nov korak,
  ki ga je treba še dodatno utemeljiti in igra se ponovi. Zgubi tisti, ki se prvi naveliča igrati. Ali lahko igra traja neskončno dolgo?
\end{naloga}

Matematični dokaz ima dvojno vlogo. Po eni strani je utemeljitev matematične
izjave, zato mora biti čim bolj podroben. V idealnem primeru bi bil dokaz
zapisan tako, da bi lahko njegovo pravilnost preverili mehansko, z
računalnikom. Po drugi strani je dokaz sredstvo za komunikacijo idej med
matematiki, zato mora vsebovati ravno pravo mero podrobnosti. Mera pa je
odvisna od tega, komu je dokaz namenjen. Te socialne komponente se študenti
učijo skozi prakso v toku študija. Dokazu kot povsem matematičnemu pojmu pa se
bomo posvetili prav pri predmetu Logika in množice. Pojasnili bomo, kaj je
dokaz kot matematični konstrukt in kako ga zapišemo tako podrobno, da je res
mehansko preverljiv. Naučili se bomo tudi nekaj preprostih tehnik iskanja
dokazov, ki pa še zdaleč ne bodo zadostovale za reševanje zares zanimivih
matematičnih problemov, ki zahtevajo poglobljeno znanje, vztrajnost, kanček
talenta in nekaj sreče.


%%%%%%%%%%%%%%%%%%%%%%%%%%%%%%%%%%%%%%%%%%%%%%%%%%%%%%%%%%%%%%%%%%%%%%

\section{Simbolni zapis matematičnih izjav}
\label{sec:simbolni-zapis-izjav}

Matematična \textbf{izjava} je smiselno besedilo, ki izraža kako lastnost ali
razmerje med matematičnimi objekti (števili, liki, funkcijami, množicami
itn.). Primeri matematičnih izjav:
%
\begin{itemize}
\item $2 + 2 = 5$.
\item Točke $P$, $Q$ in $R$ so kolinearne.
\item Enačba $x^2 + 1 = 0$ nima realnih rešitev.
\item $a > 5$.
\item $\phi \lor \psi \lthen (\lnot \phi \lthen \psi)$.
\end{itemize}
%
Vidimo, da je lahko izjava resnična, neresnična, ali pa je resničnost
izjave \emph{odvisna} od vrednosti spremenljivk, ki nastopajo v njej.
Primeri besedila, ki niso matematične izjave:
%
\begin{itemize}
\item Ali je $2 + 2 = 5$?
\item Za vsak $x > 5$.
\item Študenti bi morali znati reševati diferencialne enačbe.
\item Od nekdaj lepe so Ljubljanke slovele, al lepše od Urške bilo ni nobene.
\item $\phi \lor ) \psi \lthen \psi$.
\end{itemize}
%
Matematične izjave običajno pišemo kombinirano v naravnem jeziku in z
matematični simboli, saj so tako najlažje razumljive ljudem. Za
potrebe matematične logike pa izjave pišemo \emph{samo} z
matematičnimi simboli. Tako zapisani izjavi pravimo \textbf{logična
  formula}. V ta namen moramo nadomestiti osnovne gradnike izjav, kot
so ">in"<, ">ali"< in ">za vsak"<, z \textbf{logičnimi operacijami}.
Le-te delimo na tri sklope. V prvi sklop sodita \textbf{logični
  konstanti}:
%
\begin{itemize}
\item resnica $\top$,
\item neresnica $\bot$.
\end{itemize}
%
V računalništvu resnico $\top$ pogosto označimo z $1$ ali \texttt{True} in
neresnico $\bot$ z $0$ ali \texttt{False}. Naslednji sklop so \textbf{logični
vezniki}, s katerimi sestavljamo nove izjave iz že danih:
%
\begin{itemize}
\item konjunkcija $\phi \land \psi$, beremo ">$\phi$ in $\psi$"<,
\item disjunkcija $\phi \lor \psi$, beremo ">$\phi$ ali $\psi$"<,
\item implikacija $\phi \lthen \psi$, beremo ">če $\phi$ potem $\psi$"<,
\item ekvivalenca $\phi \liff \psi$, beremo ">$\phi$ če, in samo če, $\psi$"< ali pa ">$\phi$ natanko tedaj, kadar~$\psi$"<,
\item negacija $\lnot \phi$, beremo ">ne $\phi$"<,
\end{itemize}
%
V tretji sklop sodita \textbf{logična kvantifikatorja}:
%
\begin{itemize}
\item univerzalni kvantifikator $\all{x \in S} \phi$, beremo ">za vse $x$
  iz $S$ velja $\phi$"<,
\item eksistenčni kvantifikator $\some{x \in S} \phi$, beremo ">obstaja
  $x$ v $S$, da velja $\phi$"<,
\end{itemize}
%
Pri tem je $S$ množica, razred\footnote{V poglavju~\ref{chap:mnozice}
  bomo spoznali razliko med množicami in razredi, zaenkrat si $S$
  predstavljamo kot množico.} ali tip spremenljivke~$x$. V praksi se
uporablja več inačic zapisa za kvantifikatorje, kot so ">$\forall x : S
.\, \phi$"<, ">$\forall x \in S : \phi$"< in ">$(\forall x \in S)
\phi$"<. Srečamo tudi zapis ">$\phi, \forall x \in S$"<, ki pa ga
odsvetujemo.

\textbf{Neomejena kvantifikatorja} $\all{x} \phi$ in
$\some{x} \phi$ se uporabljata, kadar je vnaprej znana množica $S$,
po kateri teče spremenljivka~$x$. V matematičnem besedilu je običajno
razvidna iz spremnega besedila, včasih pa je treba upoštevati
ustaljene navade: $n$ je naravno ali celo število, $x$ realno, $f$ je
funkcija ipd.

V uporabi so nekatere ustaljene okrajšave:
%
\begin{xalignat*}{3}
  &\some{x,y \in S} \phi,&
  &\text{pomeni}&
  &\some{x}{S} (\some{y}{S} \phi),\\
  %
  &\all{x \in S,y \in T} \phi,&
  &\text{pomeni}&
  &\all{x \in S}(\all{y \in T} \phi),\\
  %
  &\phi \liff \psi \liff \rho \liff \sigma&
  &\text{pomeni}&
  &(\phi \liff \psi) \land (\psi \liff \rho) \land (\rho \liff \sigma),\\
  %
  &f(x) = g(x) = h(x) = i(x)&
  &\text{pomeni}&
  &f(x) = g(x) \land g(x) = h(x) \land h(x) = i(x),\\
  &a \leq b < c \leq d&
  &\text{pomeni}&
  &a \leq b \land b < c \land c \leq d.
\end{xalignat*}
%
Nekatere okrajšave odsvetujemo. V nizu neenakosti naj gredo vse
primerjave v isto smer. Torej ne pišemo $a \leq b < c \geq d$, ker se
zlahka zmotimo in mislimo, da velja $a \geq d$. To bi morali zapisati
ločeno kot $a \leq b < c$ in $c \geq d$. Prav tako ne nizamo neenakosti,
saj premnogi iz $f(x) \neq g(x) \neq h(x)$ ">sklepajo"< $f(x) \neq
h(x)$, čeprav neenakost \emph{ni} tranzitivna relacija. Zapis $f(x) =
g(x) \neq h(x) = i(x)$ je v redu, saj ena sama neenakost ne povzroči
težav.

\begin{naloga}
  Zapiši $f(x) = g(x) \neq h(x) = i(x)$ brez okrajšav.
\end{naloga}

Povejmo še nekaj o pisanju oklepajev. Oklepaji povedo, katera
operacija ima prednost. Kadar manjkajo, moramo poznati dogovorjeno
\textbf{prioriteto} operacij. Na primer, ker ima množenje višjo
prioriteto kot seštevanje, je $5 \cdot 3 + 8$ enako $(5 \cdot 3) + 8$
in ne $5 \cdot (3 + 8)$. Tudi logične operacije imajo svoje
prioritete, ki pa niso tako splošno znane kot prioritete aritmetičnih
operacij. Zato bodite pazljivi, ko berete tuje besedilo.

Mi bomo privzeli naslednje prioritete logičnih operacij:
%
\begin{itemize}
\item negacija $\lnot$ ima prednost pred
\item konjunkcijo $\land$, ki ima prednost pred
\item disjunkcijo $\lor$, ki ima prednost pred
\item implikacijo $\lthen$, ki ima prednost pred
\item kvantifikatorjema $\forall$ in $\exists$.
\end{itemize}
%
Na primer:
%
\begin{itemize}
\item $\lnot \phi \lor \psi$ je isto kot $(\lnot \phi) \lor \psi$,
\item $\lnot \lnot \phi \lthen \phi$ je isto kot $(\lnot(\lnot\phi))
  \lthen \phi$,
\item $\phi \lor \psi \land \rho$ je isto kot $\phi \lor (\psi \land \rho)$,
\item $\phi \land \psi \lthen \phi \lor \psi$ je isto kot $(\phi
  \land \psi) \lthen (\phi \lor \psi)$,
\item $\all{x \in S} \phi \lthen \psi$ je isto kot $\all{x \in S} (\phi
    \lthen \psi)$,
\item $\some{x \in S} \phi \land \psi$ je isto kot $\some{x \in S} (\phi
    \land \psi)$.
\end{itemize}

V aritmetiki poznamo poleg prioritete operacij tudi \textbf{levo} in
\textbf{desno asociranost}. Denimo, seštevanje je levo asocirano,
ker beremo $5 + 3 + 7$ kot $(5 + 3) + 7$, saj najprej izračunamo $5 +
3$ in nato $8 + 7$. Pri seštevanju to sicer ni pomembno in bi lahko
seštevali tudi $3 + 7$ in nato $5 + 10$. Drugače je z odštevanjem,
kjer $5 - 3 - 7$ pomeni $(5 - 3) - 7$ in ne $5 - (3 - 7)$. Tudi za
logične operacije velja dogovor o njihovi asociranosti. Konjunkcija in
disjunkcija sta levo asocirani:
% 
\begin{align*}
  \phi \land \psi \land \rho
  &\qquad\text{pomeni}\qquad
  (\phi \land \psi) \land \rho,\\
  \phi \lor \psi \lor \rho
  &\qquad\text{pomeni}\qquad
  (\phi \lor \psi) \lor \rho.
\end{align*}
%
Za disjunkcijo in konjunkcijo sicer ni pomembno, kako postavimo
oklepaje, ker sta obe možnosti med seboj ekvivalentni, vendar je prav,
da natančno določimo, katera od njiju je mišljena. V logiki je
implikacija desno asocirana:
%
\begin{equation*}
  \phi \lthen \psi \lthen \rho
  \qquad\text{pomeni}\qquad
  \phi \lthen (\psi \lthen \rho).
\end{equation*}
%
Tu \emph{ni} vseeno, kako postavimo oklepaje, saj $\phi \lthen (\psi
\lthen \rho)$ in $(\phi \lthen \psi) \lthen \rho$ v splošnem nista
ekvivalentna. Vendar pozor! Ko matematiki, ki niso logiki, v
matematičnem besedilu zapišejo
%
\begin{equation*}
  \phi \lthen \psi \lthen \rho,
\end{equation*}
%
s tem skoraj vedno mislijo
%
\begin{equation*}
  (\phi \lthen \psi) \land (\psi \lthen \rho).
\end{equation*}
%
Zakaj? Zato ker je to priročen zapis, ki nakazuje zaporedje sklepov
">iz $\phi$ sledi $\psi$ in nato iz $\psi$ sledi $\rho$"<, še posebej,
če je zapisan v več vrsticah. Recimo, za nenegativni števili $x$ in
$y$ bi takole zapisali utemeljitev neenakosti med aritmetično in
geometrijsko sredino:
%
\begin{align*}
  & (x - y)^2 \geq 0 \lthen \\
  & x^2 - 2 x y + y^2 \geq 0 \lthen
  \tag{razstavimo}\\
  & x^2 + 2 x y + y^2 \geq 4 x y \lthen
  \tag{prištejemo $4 x y$}\\
  & (x + y)^2 \geq 4 x y \lthen
  \tag{faktoriziramo}\\
  & \frac{(x + y)^2}{4} \geq x y \lthen
  \tag{delimo s $4$}\\
  & \frac{x+y}{2} \geq \sqrt{x y}.
  \tag{korenimo}
\end{align*}
%ANDREJ: meni je Gordon rekel, da utemeljitve sledijo sklepu, torej so
% eno vrstico niže.
%
Matematiki radi celo spustijo znak $\lthen$ in preprosto vsak
naslednji sklep napišejo v svojo vrstico. Ker torej velja tak ustaljen
način pisanja zaporedja sklepov, je varneje pisati $\phi \lthen (\psi
\lthen \rho)$ brez oklepajev, da ne povzročamo zmede.
%ANDREJ: zadnjega stavka ne razumem.

%%%%%%%%%%%%%%%%%%%%%%%%%%%%%%%%%%%%%%%%%%%%%%%%%%%%%%%%%%%%%%%%%%%%%%

\section{Kako beremo in pišemo simbolni zapis}
\label{sec:simbolni-zapis}

Izjave, zapisane v simbolni obliki, ni težko prebrati. Na primer,
%
\begin{equation*}
  \all{x, y \in \RR}
    x^2 = 4 \land y^2 = 4 \lthen x = y,
\end{equation*}
%
preberemo:
%
\begin{quote}
  ">Za vse realne $x$ in $y$, če je $x^2$ enako $4$ in $y^2$ enako
  $4$, potem je $x$ enako $y$."<
\end{quote}
%
Več izkušenj pa je potrebnih, da \emph{razumemo} matematični pomen
take izjave, v tem primeru:
%
\begin{quote}
  ">Enačba $x^2 = 4$ ima največ eno realno rešitev."<
\end{quote}
%
Začetnik potrebuje nekaj vaje, da se navadi brati simbolni zapis. Tudi
prevod v obratno smer, iz besedila v simbolno obliko, ni enostaven,
zato povejmo, kako se prevede nekatere standardne fraze.

\subsubsection{">$\phi$ je zadosten pogoj za $\psi$."<}

To pomeni, da zadošča dokazati $\phi$ zato, da dokažemo $\psi$, ali v
simbolni obliki
%
\begin{equation*}
  \phi \lthen \psi.
\end{equation*}

\subsubsection{">$\phi$ je potreben pogoj za $\psi$."<}

To pomeni, da $\psi$ ne more veljati, ne da bi veljal~$\phi$. Z drugimi
besedami, če velja $\psi$, potem velja tudi $\phi$, kar se v simbolni obliki
zapiše
%
\begin{equation*}
  \psi \lthen \phi.
\end{equation*}

\subsubsection{">$\phi$ je zadosten in potreben pogoj za $\psi$."<}

To je kombinacija prejšnjih dveh primerov, ki trdi, da iz $\phi$ sledi
$\psi$ in iz $\psi$ sledi $\phi$, kar pa je ekvivalenca:
%
\begin{equation*}
  \phi \liff \psi.
\end{equation*}

\begin{naloga}
  Je ">$n$ je sod in $n > 2$"< \textbf{potreben} ali \textbf{zadosten}
  pogoj za ">$n$ ni praštevilo"<?
\end{naloga}


\subsubsection{">Naslednje izjave so ekvivalentne: $\phi$, $\psi$, $\rho$ in $\sigma$."<}

To pomeni, da sta vsaki dve izmed danih izjav ekvivalentni, se pravi
%
\begin{equation*}
  (\phi \liff \psi) \land (\phi \liff \rho) \land (\phi \liff \sigma) \land (\psi \liff \rho)
  \land (\psi \liff \sigma) \land (\rho \liff \sigma).
\end{equation*}
%
Ker je ekvivalenca tranzitivna relacija, ni treba obravnavati vseh
kombinacij, zadostujejo že tri, ki dane izjave ">povežejo"< med seboj:
%
\begin{equation*}
  (\phi \liff \psi) \land (\psi \liff \rho) \land (\rho \liff \sigma).
\end{equation*}
%
To pišemo krajše kar kot
%
\begin{equation*}
  \phi \liff \psi \liff \rho \liff \sigma,
\end{equation*}
%
čeprav je formalno gledano tako zapis nepravilen. V
razdelku~\ref{sec:ekvivalenca} bomo spoznali, kako se tako zaporedje
ekvivalenc dokaže s ciklom implikacij $\phi \lthen \psi \lthen \rho
\lthen \sigma \lthen \phi$.

\begin{naloga}
  Podaj konkretne primere izjav $\phi$, $\psi$ in $\rho$, iz katerih
  je razvidno, da izjava $(\phi \liff \psi) \land (\psi \liff \rho)$
  \emph{ni} ekvivalentna niti $(\phi \liff \psi) \liff \rho$ niti
  $\phi \liff (\psi \liff \rho)$.
\end{naloga}



\subsubsection{">Za vsak $x$ is $S$, za katerega velja $\phi$, velja tudi
  $\psi$."<}

To lahko preberemo tudi kot ">Za vsak $x$ iz $S$, če zanj velja $\phi$,
potem velja $\psi$,"< kar je v simbolni obliki
%
\begin{equation*}
  \all{x \in S} \phi \lthen \psi.
\end{equation*}
%
Tudi izjave oblike ">vsi $\phi$-ji so $\psi$-ji"< so te oblike, denimo ">vsa
od dva večja praštevila so liha"< zapišemo
%
\begin{equation*}
  \all{n \in \NN} n > 2 \land \text{$n$ je praštevilo} \lthen \text{$n$ je lih}.
\end{equation*}

\begin{naloga}
  V simbolni obliki zapiši ">$n$ je lih"< in ">$n$ je praštevilo"<.
  Namig: $n$ je lih, kadar obstaja naravno število $k$, za katerega
  velja $n = 2 k + 1$, in $n$ je praštevilo, kadar \emph{ni} zmnožek
  dveh naravnih števil, ki sta obe večji od~$1$.
\end{naloga}


\subsubsection{">Enačba $f(x) = g(x)$ nima realne rešitve."<}

To lahko povemo takole: ni res, da obstaja $x \in \RR$, za katerega bi
veljalo $f(x) = g(x)$. S simboli zapišemo
%
\begin{equation*}
  \lnot \some{x \in \RR} f(x) = g(x).
\end{equation*}
%
Opozoriti velja, da iz same enačbe ne moremo vedno sklepati, kaj je
neznanka. V enačbi $a x^2 + b x + c = 0$ bi za neznanko lahko načeloma
imeli katerokoli od štirih spremenljivk $a$, $b$, $c$ in $x$, ali pa
kar vse. Večina matematikov bi sicer uganila, da je najverjetneje
neznanka $x$, vendar se v splošnem ne moremo zanašati na običaje in
uganjevanje, ampak moramo točno povedati, kateri simboli so
\textbf{neznanke} in kateri \textbf{parametri}.

\begin{naloga}
  Zapiši v simbolni obliki: ">Sistem enačb
  %
  \begin{align*}
    a_1 x + b_1 y &= c_1,\\
    a_2 x + b_2 y &= c_2
  \end{align*}
  %
  nima pozitivnih realnih rešitev $x, y$."<
\end{naloga}

\begin{naloga}
  Zapiši v simbolni obliki:
  \begin{enumerate}
  \item ">Enačba $f(x) = g(x)$ ima največ eno realno rešitev."<
  \item ">Enačba $f(x) = g(x)$ ima več kot eno realno rešitev."<
  \item ">Enačba $f(x) = g(x)$ ima natanko dve realni rešitvi."<
  \end{enumerate}
\end{naloga}


\subsubsection{">Brez izgube za splošnost."<}

V matematičnih besedilih najdemo frazo ">brez izgube za splošnost"<
kot v naslednjem primeru.

\begin{izrek}
  \label{izrek:abc-vsota-razlik-soda}
  Za vsa cela števila $a$, $b$ in $c$ je $|a-b|+|b-c|+|c-a|$ sodo
  število.
\end{izrek}

\begin{dokaz}
  Brez izgube za splošnost smemo predpostaviti $a \geq b \geq c$.
  Tedaj velja
  %
  \begin{equation*}
    |a-b| + |b-c| + |c-a| = (a - b) + (b - c) - (c - a) = 2 (a - c),
  \end{equation*}
  %
  kar je sodo število.
\end{dokaz}

Fraza ">brez izgube za splošnost"< nakazuje, da dokaz obravnava le eno
od večih možnosti. Načeloma bi morali obravnavati tudi ostale
možnosti, ki pa jih je pisec dokaza opustil, ker so bodisi zelo lahke
bodisi zelo podobne tisti, ki jo dokaz obravnava. Za začetnika je
najtežje dognati, katere so preostale možnosti in zakaj se je pisec
dokaza pravzaprav odločil zanje. Avtor zgornjega dokaza je verjetno
opazil, da števila $a$, $b$ in $c$ v izrazu $|a-b|+|b-c|+|c-a|$
nastopajo \emph{simetrično}: če jih premešamo, se izraz ne spremeni.
Denimo, ko zamenjamo $a$ in $b$, dobimo $|b-a|+|a-c|+|c-b|$, kar je
enako prvotnemu izrazu $|a-b|+|b-c|+|c-a|$. Torej lahko izmed šestih
možnosti
%
\begin{xalignat*}{3}
  & a \geq b \geq c,&
  & a \geq c \geq b,&
  & b \geq a \geq c,\\
  & b \geq c \geq a,&
  & c \geq a \geq b,&
  & c \geq b \geq a
\end{xalignat*}
%
obravnavamo le eno. Seveda pisanje dokazov, pri katerih večji del
dokaza opustimo, zahteva pazljivost in nekaj izkušenj.

\begin{naloga}
  Dokaži izrek~\ref{izrek:abc-vsota-razlik-soda} tako, da obravnavaš
  samo možnost $b \geq c \geq a$ in zraven dopišeš ">brez izgube za
  splošnost"<.
\end{naloga}


%%%%%%%%%%%%%%%%%%%%%%%%%%%%%%%%%%%%%%%%%%%%%%%%%%%%%%%%%%%%%%%%%%%%%%
\section{Definicije}
\label{sec:definicije}


Poznamo tri vrste definicij. Prva in najpreprostejša je definicija, ki
služi kot \textbf{okrajšava} za daljši izraz. To smemo vedno nadomestiti
s prvotnim izrazom. Na primer, funkcija ">hiperbolični tangens"<
$\tanh(x)$ je definirana kot
%
\begin{equation*}
  \tanh(x) = \frac{e^{2 x} - 1}{e^{2 x} + 1}.
\end{equation*}
%
Lahko bi shajali tudi brez zapisa $\tanh(x)$, vendar bi morali potem
povsod pisati daljši izraz $\frac{e^{2 x} - 1}{e^{2 x} + 1}$, kar bi
bilo nepregledno.

Druga vrsta definicije je vpeljava novega matematičnega pojma.
Študenti prvega letnika matematike spoznajo celo vrsto novih pojmov
(grupa, vektorski prostor, limita, stekališče, metrika itn.), s
katerimi si razširijo sposobnost matematičnega razmišljanja.
Matematiki cenijo dobre definicije in vpeljavo novih matematičnih
pojmov vsaj toliko, kot dokaze težkih izrekov.

Tretja vrsta definicije je \textbf{konstrukcija} matematičnega objekta s
pomočjo dokaza o enoličnem obstoju. O tem bomo povedali več v
razdelku~\ref{sec:enolicni-obstoj}.

\section{Pravila sklepanja in dokazi}
\label{sec:pravila-sklepanja-in-dokazi}


Povedali smo že, da je dokaz utemeljitev neke matematične izjave. V
razdelku~\ref{sec:kaj-je-dokaz} smo govorili o tem, da so dokazi
mešanica besedila in simbolov, ki jih matematiki uporabljajo tako za
utemeljitev matematičnih izjav, kakor tudi za razlago in podajanje
matematičnih idej. V tem razdelku se posvetimo \textbf{formalnim
  dokazom}, ki so logične konstrukcije namenjene mehanskemu
preverjanju pravilnosti izjav.

Za vsako logično operacijo bomo podali \textbf{formalna pravila
  sklepanja}, ki jih smemo uporabljati v formalnem dokazu. Pravilo
sklepanja shematsko zapišemo
%
\begin{equation*}
  \inferrule{\phi \\ \psi \\ \rho}{\sigma}
\end{equation*}
%
in ga preberemo: ">Če smo dokazali $\phi$, $\psi$ in $\rho$, smemo
sklepati $\sigma$."< Izjavam nad črto pravimo \textbf{hipoteze}, izjavi
pod črto pa \textbf{sklep}. Hipotez je lahko nič ali več, sklep mora
biti natanko en. Pravilo sklepanja brez hipotez se imenuje
\textbf{aksiom}.

Da bomo lahko pojasnili, kaj je dokaz, podajmo pravila sklepanja za
$\top$ in $\land$, ki jih bomo v naslednjem razdelku še enkrat bolj
pozorno obravnavali:
%
\begin{mathpar}
  \inferrule{\quad}{\top}
  %
  \and
  %
  \inferrule
  {\phi \\ \psi}
  {\phi \land \psi}
  %
  \and
  %
  \inferrule
  {\phi \land \psi}
  {\phi}
  %
  \and
  %
  \inferrule
  {\phi \land \psi}
  {\psi}  
\end{mathpar}
%
Po vrsti beremo:
%
\begin{itemize}
\item Velja $\top$.
\item Če velja $\phi$ in $\psi$, smemo sklepati $\phi \land \psi$.
\item Če velja $\phi \land \psi$, smemo sklepati $\phi$.
\item Če velja $\phi \land \psi$, smemo sklepati $\psi$.
\end{itemize}
%
Formalni dokaz ima drevesno obliko in prikazuje, kako iz danih
\textbf{hipotez} dokažemo neko \textbf{sodbo}. Pri dnu je zapisana izjava,
ki jo dokazujemo, nad njo pa dokaz. Vsako vejišče je eno od pravil
sklepanja. Vsaka veja se mora zaključiti z aksiomom ali s hipotezo.
Oglejmo si dokaz izjave $(\alpha \land \alpha) \land (\top
\land \beta)$ iz hipoteze $\beta \land \alpha$:
%
\begin{equation*}
  \inferrule{
    \inferrule{
      \inferrule{\beta \land \alpha}{\alpha}
      \\
      \inferrule{\beta \land \alpha}{\alpha}}
      {\alpha \land \alpha}
    \\
    \inferrule{
      \inferrule{ }{\top}
      \\
      \inferrule{\beta \land \alpha}{\beta}
    }{\top \land \beta}
  }{(\alpha \land \alpha) \land (\top \land \beta)}
\end{equation*}
%
Dokaz se razveji na dve veji, vsaka od njiju pa še na dve veji. Tako
pri vrhu dobimo štiri liste, od katerih se trije izjava $\beta \land
\alpha$ in en aksiom za $\top$.

\begin{naloga}
  Preveri, da je vsako vejišče v zgornjem dokazu res uporaba enega od
  zgoraj podanih pravil sklepanja.
\end{naloga}

V praksi matematično besedilo bolj ali manj odraža strukturo
formalnega dokaza, le da se besedilo ne veji, ampak so sestavni kosi
dokaza zloženi v zaporedje. Formalni dokazi so uporabni, kadar želimo
preveriti veljavnost najbolj osnovnih logičnih dejstev. Ni mišljeno,
da bi matematiki pisali ali preverjali velike formalne dokaze
pomembnih matematičnih izrekov, to je delo za račualnike. Formalna
pravila sklepanja in formalni dokazi so za matematike pomembni, ker
nam omogočajo, da natančno in v celoti povemo, kakšna so ">pravila
igre"< v matematiki.


\section{Izjavni račun}
\label{sec:izjavni-racun}

Izjavni račun je tisti del logike, ki govori o logičnih konstantah
$\bot$, $\top$ in o logičnih operacijah $\land$, $\lor$, $\lthen$,
$\liff$, $\lnot$. Za vsako od njih podamo formalna pravila sklepanja,
ki so dveh vrst. Pravila \textbf{vpeljave} povedo, kako se izjave
dokaže, pravila \textbf{uporabe} pa povedo, kako se že dokazane izjave uporabi.

\subsection{Konjunkcija}
\label{sec:konjunkcija}

Konjunkcija ima eno pravilo vpeljave in dve pravili uporabe:
%
\begin{mathpar}
  \inferrule
  {\phi \\ \psi}
  {\phi \land \psi}
  \and
  \inferrule
  {\phi \land \psi}
  {\phi}  
  %
  \and
  %
  \inferrule
  {\phi \land \psi}
  {\psi}
\end{mathpar}
%
Pravilo vpeljave pove, da konjunkcijo $\phi \land \psi$ dokažemo
tako, da dokažemo posebej $\phi$ in posebej $\psi$. Pravili uporabe pa
povesta, da lahko $\phi \land \psi$ ">razstavimo"< na izjavi~$\phi$
in~$\psi$.

V matematičnem besedilu je dokaz konjunkcije $\phi \land \psi$ zapisan
kot zaporedje dveh pod-dokazov:
%
\begin{quote}
  \it 
  %
  Dokazujemo $\phi \land \psi$:
  \begin{enumerate}
  \item (Dokaz $\phi$)
  \item (Dokaz $\psi$)
  \end{enumerate}
  Dokazali smo $\phi \land \psi$.
\end{quote}
%
Manj podroben dokaz ne vsebuje začetnega in končnega stavka, ampak
samo dokaza za $\phi$ in $\psi$. Bralec mora sam ugotoviti, da je s
tem dokazana izjava $\phi \land \psi$.

\subsection{Implikacija}
\label{sec:implikacija}

Preden zapišemo pravila sklepanja za implikacijo, si oglejmo primer
neformalnega dokaza.

\begin{izrek}
  Če je $x > 2$, potem je $x^3 + x + 7 > 16$.
\end{izrek}

\begin{dokaz}
  Predpostavimo, da velja $x > 2$. Tedaj je $x^3 > 2^3 = 8$, zato
  velja
  %
  \begin{equation*}
    x^3 + x + 7 > 8 + 2 + 7 = 17 > 16.
  \end{equation*}
  %
  Dokazali smo $x > 2 \lthen x^3 + x + 7 > 16$.
\end{dokaz}

\noindent
%
Prvi stavek dokaza z besedico ">predpostavimo"< uvede \textbf{začasno
  hipotezo} $x > 2$, iz katere nato izpeljemo posledico $x^3 + x + 7 >
16$. Implikacijo $\phi \lthen \psi$ torej dokažemo tako, da začasno
predpostavimo $\phi$ in dokažemo $\psi$. Tako pravilo vpeljave
zapišemo
%
\begin{equation*}
  \inferrule{\infer*{\psi}{[\phi]}}{\phi \lthen \psi}  
\end{equation*}
%
Zapis $[\phi]$ z oglatimi oklepaji pomeni, da $\phi$ ni prava, ampak
samo začasna hipoteza. Zapis
%
\begin{equation*}
  \infer*{\psi}{[\phi]}
\end{equation*}
%
pomeni ">dokaz izjave $\phi$ s pomočjo začasne hipoteze $\phi$."<

Pravilo uporabe za implikacijo se imenuje \textbf{modus ponens} in se
glasi
%
\begin{mathpar}
  \inferrule{\phi \lthen \psi \\ \phi}{\psi}
\end{mathpar}
%
V matematičnem besedilu se modus ponens pojavi kot uporaba že prej
dokazanega izreka izreka oblike $\phi \lthen \psi$.

\subsection{Disjunkcija}
\label{sec:disjunkcija}

Disjunkcija ima dve pravili vpeljave in eno pravilo uporabe:
%
\begin{mathpar}
  \inferrule
  {\phi}
  {\phi \lor \psi}
  \and
  \inferrule
  {\psi}
  {\phi \lor \psi}
  \and
  \inferrule
  {\phi \lor \psi \\ \infer*{\rho}{[\phi]} \\ \infer*{\rho}{[\psi]}}
  {\rho}
\end{mathpar}
%
Pravili sklepanja povesta, da lahko dokažemo disjunkcijo $\phi \lor
\psi$ tako, da dokažemo enega od disjunktov.

Pojasnimo še pravilo uporabe. Denimo, da bi radi dokazali $\rho$, pri
čemer že vemo, da velja $\phi \lor \psi$. Pravilo uporabe pravi, da je
treba obravnavati dva primera: iz začasne hipoteze $\phi$ je treba
dokazati $\rho$ in iz začasne hipoteze $\psi$ je treba dokazati
$\rho$.

Ponazorimo pravilo uporabe v dokazu izjave $(\alpha \lor \gamma) \land
(\beta \lor \gamma)$ iz hipoteze $(\alpha \land \beta) \lor \gamma$.
Dokazno drevo je precej veliko, v njem pa se dvakrat pojavi uporaba
disjunkcije:
%
\begin{equation*}
  \inferrule
  {\inferrule*
    {\inferrule*{}{(\alpha \land \beta) \lor \gamma}
      \\
      \inferrule*{
        \inferrule
        {[\alpha \land \beta]}
        {\alpha}
      }{\alpha \lor \gamma}
      \\
      \inferrule*{[\gamma]}{\alpha \lor \gamma}
    }
    {\alpha \lor \gamma}
    \\
    \inferrule*
    {\inferrule*{}{(\alpha \land \beta) \lor \gamma}
      \\
      \inferrule*{
        \inferrule
        {[\alpha \land \beta]}
        {\beta}
      }{\beta \lor \gamma}
      \\
      \inferrule*{[\gamma]}{\beta \lor \gamma}
    }
    {\beta \lor \gamma}
  }
  {(\alpha \lor \gamma) \land (\beta \lor \gamma)}
\end{equation*}
%
Poglejmo na primer levo vejo tega dokaza, desna je podobna:
%
\begin{equation*}
  \inferrule*
    {\inferrule*{}{(\alpha \land \beta) \lor \gamma}
      \\
      \inferrule*{
        \inferrule
        {[\alpha \land \beta]}
        {\alpha}
      }{\alpha \lor \gamma}
      \\
      \inferrule*{[\gamma]}{\alpha \lor \gamma}
    }
    {\alpha \lor \gamma}
\end{equation*}
%
Res je to uporaba disjunkcije $\phi \lor \psi$, kjer smo vzeli $\phi =
\alpha \land \beta$ in $\psi = \gamma$, dokazali pa smo izjavo $\rho =
\alpha \lor \gamma$.

\begin{naloga}
  Iz hipoteze $(\alpha \lor \gamma) \land (\beta \lor \gamma)$ dokaži
  $(\alpha \land \beta) \lor \gamma$.
\end{naloga}

V besedilu dokažemo disjunkcijo s pravilom za vpeljavo takole:
%
\begin{quote}
  \it
  %
  Dokazujemo $\phi \lor \psi$. Zadostuje dokazati $\phi$:
  \begin{enumerate}
  \item[] (Dokaz $\phi$.)
  \end{enumerate}
  %
  Dokazali smo $\phi \lor \psi$.
\end{quote}
%
Pravilo uporabe disjunkcije se v besedilu zapiše kot obravnava
primerov:
%
\begin{quote}
  \it
  %
  Dokazujemo $\rho$. To bomo dokazali z obravnavo primerov $\phi$ in
  $\psi$:
  \begin{enumerate}
  \item (Dokaz $\phi \lor \rho$)
  \item Predpostavimo, da velja $\phi$. (Dokaz $\rho$)
  \item Predpostavimo, da velja $\psi$. (Dokaz $\rho$)
  \end{enumerate}
  %
  Dokazali smo $\rho$.
\end{quote}
%
Še primer konkretnega dokaza, ki je tako napisan.

\begin{izrek}
  \label{izrek:x-3-5}
  Naj bo $x$ realno število. Če je $|x - 3| > 5$, potem je $x^4 > 15$.
\end{izrek}

\begin{dokaz}
  Dokazujemo $|x - 3| > 5 \lthen x^4 > 15$. Predostavimo $|x - 3| > 5$
  in dokažimo $x^4 > 15$. To bomo dokazali z obravavo primerov $x \leq
  3$ in $x \geq 3$:
  %
  \begin{enumerate}
  \item $x \leq 3 \lor x \geq 3$ velja, ker so realna števila linearno
    urejena z relacijo $\leq$.
  \item Predpostavimo $x \leq 3$. Tedaj je $x - 3 \leq 0$ in zato $|x
    - 3| = 3 - x$, od koder sledi $3 - x = |x - 3| > 5$, oziroma $x <
    -2$. Tako dobimo
    %
    \begin{equation*}
      x^4 > (-2)^4 = 16 > 15.
    \end{equation*}
  \item Predpostavimo $x \geq 3$. Tedaj je $x - 3 \geq 0$ in zato$|x -
    3| = x - 3$, od koder sledi $x - 3 = |x - 3| > 5$, oziroma $x >
    8$. Tako dobimo
    %
    \begin{equation*}
      x^4 > 8^4 = 4096 > 15.
    \end{equation*}
  \end{enumerate}
  %
  Iz predpostavke $|x - 3| > 5$ smo izpeljali $x^4 > 15$. S tem smo
  dokazali $|x - 3| > 5 \lthen x^4 > 15$.
\end{dokaz}

Težji del tega dokaza se skriva v izbiri disjunkcije. Kako je pisec
uganil, da je treba obravnavati primera $x \leq 3$ in $x \geq 3$?
Zakaj ni raje obravnaval $x < 3$ in $x \geq 3$, ali morda $x \leq 17$
in $x \geq 17$? Odgovor se skriva v definiciji absolutne vrednosti:
%
\begin{equation*}
  |a| =
  \begin{cases}
    a & \text{če je $a \geq 0$,}\\
    -a & \text{če je $a \leq 0$.}
  \end{cases}
\end{equation*}
%
Ker v izreku nastopa izraz $|x - 3|$, bo obravnava primerov $x - 3
\geq 0$ in $x - 3 \leq 0$ omogočila, da $|x - 3|$ poenostavimo enkrat
v $x - 3$ in drugič v $3 - x$. Seveda pa je $x - 3 \geq 0$
ekvivalentno $x \geq 3$ in $x - 3 \leq 0$ ekvivalentno $x \leq 3$.

\begin{naloga}
  Ali bi lahko izrek~\ref{izrek:x-3-5} dokazali tudi z obravnavo
  primerov $x < 3$ in $x \geq 3$?
\end{naloga}

\subsection{Resnica in neresnica}
\label{sec:resnica-neresnica}

Logična konstanta $\top$ označuje resnico. Kar je res, je res, in tega
ni treba posebej dokazovati. To dejstvo izraža aksiom
%
\begin{equation*}
  \inferrule{\qquad}{\top}
\end{equation*}
%
Logična konstanta $\top$ nima pravila uporabe, ker iz $\top$ ne moremo
sklepati nič koristnega.

Logična konstanta $\bot$ označuje neresnico. Ker se tega, kar ni res,
ne more dokazati, $\bot$ nima pravila vpeljave. Pravilo uporabe je
%
\begin{equation*}
  \inferrule{\quad\bot\quad}{\phi}
\end{equation*}
%
se imenuje \textbf{ex falso (sequitur) quodlibet}, kar pomeni ">iz
neresnice sledi karkoli"<.

V matematičnem besedilu se $\top$ in $\bot$ ne pojavljata pogosto, ker
matematiki izraze, v katerih se $\top$ in $\bot$ pojavita, vedno
poenostavijo s pomočjo ekvivalenc:
%
\begin{mathpar}
  \top \land \phi \liff \phi
  \and
  \top \lor \phi \liff \phi
  \and
  \bot \land \phi \liff \bot
  \and
  \bot \lor \phi \liff \phi
  \\
  (\top \lthen \phi) \liff \phi
  \and
  (\bot \lthen \phi) \liff \top
  \and
  (\phi \lthen \top) \liff \top
\end{mathpar}
%

\subsection{Ekvivalenca}
\label{sec:ekvivalenca}

Logična ekvivalenca $\phi \liff \psi$ je okrajšava za
%
\begin{equation*}
  (\phi \lthen \psi) \land (\psi \lthen \phi).
\end{equation*}
%
Ker je to konjunkcija (dveh implikacij), so pravila za vpeljavo in
uporabo ekvivalence samo poseben primer pravil sklepanja za
konjunkcijo:
%
\begin{mathpar}
  \inferrule
  {\phi \lthen \psi \\ \psi \lthen \phi}
  {\phi \liff \psi}
  \and
  \inferrule{\phi \liff \psi}{\phi \lthen \psi}
  \and
  \inferrule{\phi \liff \psi}{\psi \lthen \phi}
\end{mathpar}
%
V matematičnem besedilu ekvivalenco dokažemo takole:
%
\begin{quote}
  \it
  %
  Dokazujemo $\phi \liff \psi$:
  %
  \begin{enumerate}
  \item (Dokaz $\phi \lthen \psi$)
  \item (Dokaz $\psi \lthen \phi$)
  \end{enumerate}
  Dokazali smo $\phi \liff \psi$.
\end{quote}

Če sta izjavi $\phi$ in $\psi$ logično ekvivalentni, lahko eno
zamenjamo z drugo. To matematiki s pridom uporabljajo pri dokazovanju
izjav, čeprav pogosto sploh ne omenijo, katero ekvivalenco so
uporabili.

Kadar dokazujemo medsebojno ekvivalenco večih izjav $\phi_1$,
$\phi_2$, \ldots, $\phi_n$, zadostuje dokazati cikel implikacij
%
\begin{equation*}
  \phi_1 \lthen \phi_2 \lthen \cdots \lthen \phi_{n-1} \lthen \phi_n \lthen \phi_1.
\end{equation*}
%
(Ne spreglejte zadnje implikacije $\phi_n \lthen \phi_1$, ki zaključi
cikel). V besedilu to dokažemo:

\begin{quote}
  \it
  %
  Dokazujemo, da so izjave $\phi_1, \phi_2, \ldots, \phi_n$
  ekvivalentne:
  %
  \begin{enumerate}
  \item (Dokaz $\phi_1 \lthen \phi_2$)
  \item (Dokaz $\phi_2 \lthen \phi_3$)
  \item \dots
  \item (Dokaz $\phi_{n-1} \lthen \phi_n$)
  \item (Dokaz $\phi_n \lthen \phi_1$)
  \end{enumerate}
\end{quote}

\noindent
%
Seveda smemo pred samim dokazovanjem izjave $\phi_1, \ldots, \phi_n$
preurediti tako, da je zahtevane implikacije kar najlažje dokazati.
Dokaz lahko tudi razdelimo na dva ločena cikla implikacij
%
\begin{equation*}
  \phi_1 \lthen \cdots \lthen \phi_k \lthen \phi_1
\end{equation*}
%
in
%
\begin{equation*}
  \phi_{k+1} \lthen \cdots \lthen \phi_n \lthen \phi_{k+1}
\end{equation*}
%
in nato dokažemo še eno ekvivalenco $\phi_i \liff \phi_j$, pri čemer
je $\phi_i$ iz prvega in $\phi_j$ iz drugega cikla.

\subsection{Negacija}
\label{sec:negacija}


Negacija $\lnot\phi$ je definirana kot okrajšava za $\phi
\lthen \bot$. Iz pravil sklepanja za $\lthen$ in $\bot$ tako izpeljemo
pravili sklepanja za negacijo:
%
\begin{mathpar}
  \inferrule
  {\infer*{\bot}{[\phi]}}
  {\lnot \phi}
  %
  \and
  %
  \inferrule
  {\lnot\phi \\ \phi}
  {\psi}
\end{mathpar}
%
V besedilu dokazujemo $\lnot\phi$ takole:
%
\begin{quote}
  \it
  %
  Dokazujemo $\lnot\phi$.
  \begin{itemize}
  \item[] Predpostavimo $\phi$.
  \item[] (Dokaz $\bot$.)
  \end{itemize}
  Dokazali smo $\lnot\phi$.
\end{quote}
%
Tu ">Dokaz $\bot$"< pomeni, da iz danih predpostavk izpeljemo
protislovje. Mnogi matematiki menijo, da se takemu dokazu reče ">dokaz
s protislovjem"<, vendar to ni res. To je samo navaden dokaz negacije.
Dokazovanje s protislovjem bomo obravnavali v razdelku~\ref{sec:lem}.

Pravilo uporabe za $\lnot\phi$ v besedilu ni eksplicitno vidno, ampak
ga matematiki uporabijo, ko sredi dokaza, da velja $\psi$, izpeljejo
protislovje:
%
\begin{quote}
  \it
  %
  Dokazujemo $\psi$.
  %
  \begin{itemize}
  \item[] (Dokaz $\phi$.)
  \item[] (Dokaz $\lnot\phi$.)
  \end{itemize}
  %
  To je nesmisel, in ker iz nesmisla sledi karkoli, sledi $\psi$.
\end{quote}

\subsection{Aksiom o izključenem tretjem}
\label{sec:lem}

Aksiom o izključenem tretjem se glasi
%
\begin{equation*}
  \inferrule{ }{\phi \lor \lnot \phi}
\end{equation*}
%
Povedano z besedami, vsaka izjava je bodisi resnična bodisi
neresnična. Torej ni ">tretje možnosti"< za resničnostno vrednost
izjave $\phi$, od koder izhaja tudi ime aksioma.

Aksiom o izključenem tretjem omogoča \emph{posredne} dokaze izjav, od
katerih je najbolj znano \textbf{dokazovanje s protislovjem}: pri tem ne
utemeljimo izjave $\phi$, ampak utemeljimo, zakaj $\lnot\phi$
\emph{ne} velja. Natančneje povedano, izjavo $\phi$ zamenjamo z njej
ekvivalentno izjavo $\lnot\lnot\phi$ in dokažemo $\lnot\lnot\phi$.
Dokaz ekvivalence $\phi \liff \lnot\lnot\phi$ sestoji iz dokazov dveh
implikacij:
%
\begin{mathpar}
  \inferrule
  {\inferrule{
      \inferrule{[\lnot\phi] \\ [\phi]}{\bot}
    }
    {\lnot\lnot\phi}
  }
  {\phi \lthen \lnot\lnot\phi}
  %
  \and
  %
  \inferrule*
  {\inferrule*
    {\inferrule*{ }{\phi \lor \lnot\phi} \\
     [\phi] \\
     \inferrule*{
       \inferrule*{
         [\lnot\lnot\phi] \\ [\lnot\phi]
       }
       {\bot}
     }
     {\phi}
    }
    {\phi}
  }
  {\lnot\lnot\phi \lthen \phi}
\end{mathpar}
%
V dokazu $\lnot\lnot\phi \lthen \phi$ smo uporabili aksiom o
izključenem tretjem. V matematičnem besedilu se dokaz s protislovjem
glasi:
%
\begin{quote}
  \it
  %
  Dokažimo $\phi$ s protislovjem.
  %
  \begin{itemize}
  \item[] Predpostavimo, da bi veljalo $\lnot\phi$.
  \item[] (Dokaz neresnice $\bot$.)
  \end{itemize}
  %
  Ker torej $\lnot\phi$ pripelje do protislovja, velja $\phi$.
\end{quote}
%
Praviloma izvemo o vsebini matematične izjave~$\phi$ več, če jo
dokažemo neposredno. Dokazovanja s protislovjem zato ni smiselno
uporabljati vsepovprek, ampak le takrat, ko je zares potreben ali ko
nam zelo olajša dokazovanje.

Ostali načini za sestavljanje posrednih dokazov slonijo na
ekvivalencah
%
\begin{mathpar}
  (\phi \lor \psi) \liff \lnot (\lnot\phi \land \lnot\psi),\and
  (\phi \lor \psi) \liff (\lnot\phi \lthen \psi),\and
  (\phi \lthen \psi) \liff (\lnot\psi \lthen \lnot\phi),\and
  (\all{x \in S} \phi) \liff \lnot \some{x \in S} \lnot \phi,\and
  (\some{x \in S} \phi) \liff \lnot \all{x \in S} \lnot \phi.
\end{mathpar}
%
V vseh petih primerih implikacija $\lthen$ iz leve na desno velja brez
uporabe aksioma o izključenem tretjem. Za dokaz implikacij
$\Leftarrow$ is desne na levo pa potrebujemo aksiom o izključenem
tretjem.

\begin{naloga}
  Sestavi formalne dokaze za zgornjih pet ekvivalenc. Pri dokazovanju
  ekvivalenc za $\forall$ in $\exists$ si pomagaj s pravili sklepanja
  iz razdelkov~\ref{sec:univerzalni-kvantifikator}
  in~\ref{sec:eksistencni-kvantifikator}.
\end{naloga}

Povejmo, kako zgornje ekvivalence uporabimo v besedilu za posredni
dokaz izjave:
%
\begin{itemize}
\item $(\phi \lor \psi) \liff \lnot (\lnot\phi \land \lnot\psi)$
  uporabimo takole:
  %
  \begin{quote}
    \it
    %
    Dokazujemo $\phi \lor \psi$.
    %
    \begin{itemize}
    \item[] Predpostavimo, da velja $\lnot\phi$ in $\lnot\psi$.
    \item[] (Dokaz neresnice $\bot$.)
    \end{itemize}
    %
    Ker torej nista $\phi$ in $\psi$ oba neresnična, je eden od njiju
    resničen. Dokazali smo $\phi \lor \psi$.
  \end{quote}
\item $(\phi \lor \psi) \liff (\lnot\phi \lthen \psi)$ uporabimo
  takole:
  %
  \begin{quote}
    \it
    %
    Dokazujemo $\phi \lor \psi$.
    %
    \begin{itemize}
    \item[] Predpostavimo $\lnot\phi$.
    \item[] (Dokaz $\psi$.)
    \end{itemize}
    %
    Če torej ne velja $\lnot\phi$, velja $\psi$. Torej velja vsaj
    eden, zato smo dokazali $\phi \lor \psi$.
  \end{quote}
\item $(\phi \lthen \psi) \liff (\lnot\psi \lthen \lnot\phi)$
  uporabimo takole:
  %
  \begin{quote}
    \it
    %
    Dokazujemo $\phi \lthen \psi$.
    %
    \begin{enumerate}
    \item Predpostavimo $\lnot\psi$.
    \item (Dokaz $\lnot\psi$.)
    \end{enumerate}
    %
    Dokazali smo, da iz $\phi$ sledi $\psi$.
  \end{quote}
\item $(\all{x \in S} \phi) \liff \lnot \some{x \in S} \lnot \phi$
  uporabimo takole:
  %
  \begin{quote}
    \it
    %
    Dokazujemo, da za vsak $x \in S$ velja $\phi$.
    %
    \begin{enumerate}
    \item Predpostavimo, da obstaja $x \in S$, za katerega $\phi$
      \emph{ne} velja.
    \item (Dokaz neresnice $\bot$.)
    \end{enumerate}
    %
    Predpostavka, da obstaja $x \in S$, za katerega $\phi$ ne velja,
    pripelje do protislovja. Torej za vsak $x \in S$ velja $\phi$.
  \end{quote}
\item $(\some{x \in S} \phi) \liff \lnot \all{x \in S} \lnot \phi$
  uporabimo takole:
  %
  \begin{quote}
    \it
    %
    Dokazujemo, da obstaja tak $x \in S$, za katerega velja $\phi$.
    %
    \begin{enumerate}
    \item Predpostavimo, da bi veljalo $\lnot\phi$ za vse $x \in S$.
    \item (Dokaz neresnice $\bot$.)
    \end{enumerate}
    %
    Predpostavka, da velja $\lnot\phi$ za vse $x \in S$, pripelje do
    protislovja. Torej obstaja $x \in S$, za katerega velja $\phi$.
  \end{quote}
\end{itemize}

Negacijo poljubne izjave $\phi$ tvorimo preprosto tako, da pred njo
postavimo $\lnot$. Vendar nam to ne pove dosti o matematični vsebini
negirane izjave. V večini primerov je negacijo lažje razumeti, če
simbol~$\lnot$ ">porinemo"< navznoter do osnovnih izjav z uporabo
naslednjih ekvivalenc:
%
\begin{align*}
  \lnot (\phi \land \psi) &\iff \lnot\phi \lor \lnot\psi \\
  \lnot (\phi \lor \psi) &\iff \lnot\phi \land \lnot\psi \\
  \lnot (\phi \lthen \psi) &\iff \phi \land \lnot\psi \\
  \lnot (\lnot \phi) &\iff \phi \\
  \lnot (\all{x \in S} \phi) &\iff \some{x \in S} \lnot\phi \\
  \lnot (\some{x \in S} \phi) &\iff \all{x \in S} \lnot\phi
\end{align*}

\begin{primer}
  Denimo, da bi radi ovrgli izjavo
  % 
  \begin{quote}
    ">Vsako zaporedje pozitivnih realnih števil ima limito~$0$."<
  \end{quote}
  % 
  Da izjavo ovržemo, moramo dokazati njeno negacijo. Načeloma lahko
  negacijo tvorimo tako, da pred izjavo napišemo ">ni res, da velja
  \dots"<, a nam to ne pove, kako bi negacijo dokazali. Zapišimo
  prvotno izjavo v delni simbolni obliki:
  % 
  \begin{equation}
    \label{eq:pozitivno-limita-0}
    \all{a \in \RR^\NN}{\text{$(a_n)_n$ pozitivno zaporedje}
      \lthen \text{$0$ je limita zaporedja $(a_n)_n$}}.
  \end{equation}
  % 
  Zgornja pravila za računanje negacije nam povedo, da se
  $\lnot\forall$ spremeni v $\exists\lnot$ in da se nato implikacija
  oblike $\phi \lthen \psi$ spremeni v $\phi \land \lnot\psi$. Tako
  izrazimo negacijo izjave~\eqref{eq:pozitivno-limita-0}:
  % 
  \begin{equation*}
    \some{a \in \RR^\NN}{\text{$(a_n)_n$ pozitivno zaporedje}
      \land \lnot (\text{$0$ je limita zaporedja $(a_n)_n$})}.
  \end{equation*}
  % 
  To preberemo z besedami:
  % 
  \begin{quote}
    ">Obstaja tako zaporedje $(a_n)_n$, da je $(a_n)_n$ zaporedje
    pozitivnih števil in da $0$ ni limita zaporedja $(a_n)_n$."<
  \end{quote}
  %
  Če se še malo potrudimo, preberemo bolj razumljivo:
  % 
  \begin{quote}
    ">Obstaja tako zaporedje pozitivnih realnih števil, da $0$ ni
    njegova limita."<
  \end{quote}
  %
  S tem še nismo končali, saj je tudi ">Število $0$ ni limita
  zaporedja $(a_n)_n$"< negacija. Izjavo ">$0$ je limita zaporedja
  $(a_n)_n$"< najprej zapišemo simbolno:
  % 
  \begin{equation}
    \label{eq:limita-0}
    \all{\epsilon > 0}
      \some{m}{\NN}
        \all{n \geq m}
          |a_n - 0| < \epsilon.
  \end{equation}
  % 
  Z zgornjimi pravili za negiranje izračunamo negacijo
  izjave~\eqref{eq:limita-0}. Operacijo $\lnot$ postopoma ">porivamo"<
  navznoter:
  % 
  % 
  \begin{align*}
    \lnot \all{\epsilon > 0} \some{m \in \NN} \all{n \geq m} |a_n
          - 0| < \epsilon & \iff
    \\
    \some{\epsilon > 0} \lnot \some{m}{\NN} \all{n \geq m}
          |a_n - 0| < \epsilon &\iff
    \\
    \some{\epsilon > 0} \all{m}{\NN} \lnot \all{n \geq m}
          |a_n - 0| < \epsilon &\iff
    \\
    \some{\epsilon > 0} \all{m}{\NN} \some{n \geq m}
          \lnot (|a_n - 0| < \epsilon) &\iff
    \\
    \some{\epsilon > 0} \all{m}{\NN} \some{n \geq m}
          |a_n - 0| \geq \epsilon &\iff
    \\
    \some{\epsilon > 0} \all{m}{\NN} \some{n \geq m}
          a_n \geq \epsilon.
  \end{align*}
  % 
  V zadnjem koraku smo upoštevali, da za pozitivno število $a_n$ velja
  $|a_n - 0| = |a_n| = a_n$. Tako smo dobili podrobno zapisano
  negacijo prvotne izjave
  % 
  \begin{quote}
    ">Obstaja tako zaporedje pozitivnih števil $(a_n)_n$ in obstaja
    tak $\epsilon > 0$, da za vsak $m \in \NN$ obstaja $n \geq m$, za
    katerega velja $a_n > \epsilon$."<
  \end{quote}
  % 
  To izjavo pa znamo dokazati tako, da podamo konkreten primer
  zaporedja $(a_n)_n$ in konkretno vrednost $\epsilon$, ki zadoščata
  pogoju, denimo $a_n = 2 + n$ in $\epsilon = 1$. Res, če je $m \in
  \NN$ poljuben, lahko vzamemo kar $n = m$, saj potem velja $a_n = a_m
  = 2 + m > 1 = \epsilon$.

  Pričujoči primer smo zapisali zelo podrobno. Izkušeni matematik tega
  seveda ne bo pisal, saj bo izračunal negacijo prvotne izjave kar v
  glavi in takoj podal primer zaporedja, ki dokazuje, da prvotna
  izjava ne velja.
\end{primer}

%%%%%%%%%%%%%%%%%%%%%%%%%%%%%%%%%%%%%%%%%%%%%%%%%%%%%%%%%%%%%%%%%%%%%%
\section{Predikatni račun}
\label{sec:predikatni-racun}

Predikatni račun je tisti del logike, ki obravnava predikate ter
kvantifikatorja~$\forall$ in~$\exists$.

Predikate tvorimo z logičnimi operacijami in kvantifikatorji iz
\textbf{osnovnih predikatov}. Katere osnovne predikate imamo na voljo,
je odvisno od snovi, ki jo obravnavamo.\footnote{Na primer, če
  obravnavamo ravninsko geometrijo, potem so osnovni predikati ">točka
  $x$ leži na premici $y$"<, ">premici $p$ in $q$ se sekata"< itn.}
Vedno imamo na voljo tudi \textbf{enakost} $x = y$, ki jo bomo
obravnavali v razdelku~\ref{sec:enakost}.

V osnovnih predikatih nastopajo \textbf{izrazi} ali \textbf{termi}. Katere
izraze lahko tvorimo je spet odvisno od tega, katere konstante in
operacije imamo na voljo. Na primer, če obravnavamo aritmetiko celih
števil, so na voljo operacije $+$, $-$, $\times$, če pa obravnavamo
realna števila, so na voljo operacije $+$, $-$, $\times$, $/$. V
izrazih vedno lahko nastopajo \textbf{spremenljivke}. Kadar uporabimo
spremenljivko, moramo povedati njen \textbf{tip} oziroma \textbf{množico}
vrednosti, ki jih lahko zavzame spremenljivka. Pogosto je tip
spremenljivke razviden iz spremnega besedila ali iz ustaljene uporabe:
$n$ se uporablja za naravno število, $x$ za realno, $f$ za funkcijo
ipd.

Ponazorimo pravkar definirane pojme s primerom. Predikat
%
\begin{equation*}
  0 < f(x) \land f(x) < \pi/4 \lthen \sin(2 f(x)) = 1/3
\end{equation*}
%
je sestavljen s pomočjo logičnih operacij $\land$ in $\lthen$ iz treh
osnovnih predikatov, zgrajenih iz osnovnih relacij $<$ in $=$,
%
\begin{mathpar}
  0 < f(x)
  \and
  f(x) < \pi/4
  \and
  \sin(2 f(x)) = 1/3,
\end{mathpar}
%
v katerih nastopa pet izrazov:
%
\begin{mathpar}
  0
  \and
  f(x)
  \and
  \pi/4
  \and
  \sin(2 f(x))
  \and
  1/3
\end{mathpar}
%
V teh izrazih nastopa spremenljivka $x$, katere tip je množica
realnih števil (to moramo uganiti) in spremenljivka $f$, ki označuje
funkcijo iz realnih v realna števila (tudi to moramo uganiti).
Nadalje, v izrazih nastopajo konstante $0$, $\pi$, $4$, $2$,
$1$ in $3$, operacija $\sin$ in operacija množenja.


%%%%%%%%%%%%%%%%%%%%%%%%%%%%%%%%%%%%%%%%%%%%%%%%%%
\subsection{Proste in vezane spremenljivke}
\label{sec:spremenljivke}

V predikatih in izrazih se pojavljajo spremenljivke. Pri tem moramo
ločiti med \textbf{prostimi} in \textbf{vezanimi} spremenljivkami. Oglejmo
si naslednja izraza in predikat:
%
\begin{equation*}
  \sum_{i=0}^{n} a_i,
  \qquad
  \int_0^1 f(t) \, d t,
  \qquad
  \forall x \in A .\, \phi(x) \;.
\end{equation*}
%
V vsoti je vezana spremenljivka $i$, spremenljivki $n$ in $a$ sta
prosti. To pomeni, da je $i$ neke vrste ">lokalna
spremenljivka"<,\footnote{Podobnost z lokalnimi spremenljivkami v
  programskih jezikih ni zgolj naključje. Lokalna spremenljivka in
  števec v zanki sta tudi primera vezanih spremenljivk v teoriji
  programskih jezikov.} katere veljavnost je samo znotraj vsote,
medtem ko sta spremenljiki $n$ in $a$ veljavni tudi zunaj samega
izraza. Podobno je v integralu $t$ vezana spremenljivka in $f$ prosta,
v izjavi na desni pa je vezana spremenljivka $x$, spremenljivki $A$ in
$\phi$ sta prosti.

Vezane spremenljivke so ">nevidne"< zunaj izraza in jih lahko vedno
preimenujemo, ne da bi spremenili pomen izraza (seveda se novo ime ne
sme mešati z ostalimi spremenljivkami, ki nastopajo v izrazu): izraza
$\int_0^1 f(t)\, d t$ in $\int_0^1 f(x)\, d x$ štejemo za
\emph{enaka}, ker se razlikujeta le v imenu vezane spremenljivke.
Spremenljivki, ki ni vezana, pravimo \textbf{prosta}. Izrazu, v katerem
ni prostih spremenljivk, pravimo \textbf{zaprt izraz}. Zaprta
logična izjava se imenuje \textbf{stavek}.

Pomembno se je zavedati, da vezana spremenljivka ">zunaj"< svojega
območja ne obstaja. Matematiki so glede tega precej površni in na
primer pišejo
%
\begin{equation*}
  \int x^2 \, d x = x^3/3 + C,
\end{equation*}
%
kar je strogo gledano nesmisel. Na levi strani v integralu stoji
vezana spremenljivka~$x$, ki je na desni ">pobegnila"< iz integrala.
Še več, če je $x \in \RR$ in $C \in \RR$, potem je izraz $x^3/3 + C$
\emph{število} (odvisno od vrednosti $x$ in $C$), saj je vsota dveh
realnih števil. Na desni strani bi morala stati oznaka za
\emph{funkcijo}, recimo
%
\begin{equation*}
  \int x^2 \, d x = (x \mapsto x^3/3 + C),
\end{equation*}
%
vendar tega v praksi nihče ne piše. Seveda pri vsem tem ostane še
vprašanje, kakšno vlogo ima v zgornjem izrazu~$C$. Pri analizi se
učimo, da je~$C$ ">poljubna konstanta"<. Poskusimo to razumeti
natančno s stališča logike. Besedico ">poljubno"< ponavadi razumemo
kot ">za vsak"<, vendar to ne gre, saj je
%
\begin{equation*}
  \all{C \in \RR} \int x^2 \, d x = (x \mapsto x^3/3 + C)
\end{equation*}
%
nesemisel. Če bi to bilo res, bi veljalo za $C = 1$ in za $C = 2$, od
koder bi dobili
%
\begin{equation*}
  (x \mapsto x^3/3 + 1) =
  \int x^2 \, d x =
  (x \mapsto x^3/3 + 2).
\end{equation*}
%
Potemtakem bi morali biti funkciji $(x \mapsto x^3/3 + 1)$ in $(x
\mapsto x^3/3 + 1)$ enaki, od koder sledi nesmisel $1 = 2$. Težave
nastopajo iz dejstva, da poskušamo nedoločeni integral razumeti kot
operacijo, ki slika funkcije v funkcije, kar ni. Nedoločeni integral
preslika funkcijo~$f$ v \emph{množico} vseh funkcij $F$, za katere
velja $F' = f$. Če bi to želeli zapisati zares pravilno, bi dobili
%
\begin{equation*}
  \int x^2 \, d x =
  \set{(x \mapsto x^3/3 + C) \such C \in \RR}.
\end{equation*}
%
Ali naj torej sklepamo, da so matematiki pravzaprav zelo površni pri
pisanju integralov? Da, s stališča formalne logike prav gotovo. Vendar
to ni nujno slabo: matematični zapis v praksi služi ljudem za
sporazumevanje in prav je, da si izberejo tak zapis, s katerim najbolj
učinkovito komunicirajo drug z drugim. Kljub temu pa se je treba
zavedati, kdaj gredo matematiki ">po bližnjici"< in ne zapišejo ali
povedo vsega dovolj natančno, da bi to bilo sprejemljivo za standarde,
ki jih postavlja formalna logika.


%%%%%%%%%%%%%%%%%%%%%%%%%%%%%%%%%%%%%%%%%%%%%%%%%%
\subsection{Substitucija}
\label{sec:substitucija}

\textbf{Substitucija} je osnovna sintaktična operacija, v kateri
\emph{proste} spremenljivke zamenjamo z izrazi. Zapis
%
\begin{equation*}
  \xsubst{e}{x_1 \subto e_1, \ldots, x_n \subto e_n}
\end{equation*}
%
pomeni: ">v izrazu $e$ \emph{hkrati} zamenjaj proste spremenljivke
$x_1$ z $e_1$, $x_2$ z $e_2$, \dots in $x_n$ z $e_n$."<  Na primer,
%
\begin{equation*}
  \subst{x^2 + y}{x \subto 3, y \subto 5, z \subto 12}
\end{equation*}
%
je enako $3^2 + 5$. Nič hudega ni, če se v substituciji omenja
spremenljivko $z$, ki se v izrazu $x^2 + y$ ne pojavi.

Ko naredimo substitucijo, moramo paziti, da se proste spremenljivke ne
">ujamejo"<. Denimo, da želimo v integralu
%
\begin{equation*}
  \int_0^1 \frac{x}{a - x^2} \; dx
\end{equation*}
%
parameter $a$ zamenjati z $y^2$. To naredimo s substitucijo
%
\begin{equation*}
  \xsubst{\left(\int_0^1 \frac{x}{a - x^2} \; dx\right)}{a \subto y^2} =
  \int_0^1 \frac{x}{y^2 - x^2} \; dx.
\end{equation*}
%
Vse lepo in prav. Kaj pa, če želimo $a$ zamenjati z $1 + x$? Ker je
spremenljivka $x$ vezana v integralu, \emph{ne smemo} delati takole:
%
\begin{equation*}
  \xsubst{\left(\int_0^1 \frac{x}{a - x^2} \; dx\right)}{a \subto x^2} =
  \int_0^1 \frac{x}{x^2 - x^2} \; dx ?!
\end{equation*}
%
Ker vstavljamo v integral spremenljivko $x$, moramo vezano
spremenljivko $x$ najprej preimenovati v kaj drugega, na primer $t$,
šele nato vstavimo:
%
\begin{equation*}
  \xsubst{\left(\int_0^1 \frac{x}{a - x^2}\; dt \right)}{a \subto x^2} =
  \xsubst{\left(\int_0^1 \frac{t}{a - t^2} \; dt\right)}{a \subto x^2} =
  \int_0^1 \frac{t}{x^2 - t^2} \; dt.
\end{equation*}
%
Podajmo še nekaj primerov substitucij:
%
\begin{align*}
  \subst{x + y + 1}{x \subto 2} &= 2 + y + 1 \;,
  \\
  \subst{x + y^2 + 1}{x \subto y, y \subto x} &= y + x^2 + 1 \;
  \\
  \subst{\subst{x + y^2 + 1}{x \subto y}}{y \subto x} &=
  x + x^2 + 1 \;,
  \\
  \textstyle
  \subst{x + \int_0^1 x \cdot y \;, d x}{x \subto 2}
  &= \textstyle  2 + \int_0^1 x \cdot y \;, d x \;,
  \\
  \textstyle
  \subst{\int_0^1 x \cdot y \; d x}{y \subto x^2}
  &= \textstyle \int_0^1 t \cdot x^2 \; d t \;.
\end{align*}
%
Ločiti je treba med hkratno in zaporedno substitucijo:
%
\begin{align*}
  \subst{x + y^2}{x \subto y, y \subto x} &= y + x^2
  \\
  \subst{\subst{x + y^2}{x \subto y}}{y \subto x} &=
  \subst{y + y^2}{y \subto x} = x + x^2
  \\
  \subst{\subst{x + y^2}{y \subto x}}{x \subto y} &=
  \subst{x + x^2}{x \subto y} = y + y^2.
\end{align*}
%

V nadaljevanju bomo obravnavali pravila sklepanja za univerzalne in
eksistenčne kvantifkatorje, v katerih se pojavi substitucija. Ker je
sam zapis za substitucijo nekoliko nepregleden, bomo uporabili
nekoliko manj pravilen, a bolj praktičen zapis. Denimo, da imamo
logično formulo $\phi$, v kateri se morda pojavi spremenljivka $x$, ni
pa to nujno. Tedaj pišemo $\phi(x)$. Če želimo zamenjati $x$ z izrazom
$e$, zapišemo $\phi(e)$. To je pravzaprav običajni zapis, kot ga
uporabljajo matematiki za zapis funkcij, mi pa smo ga uporabili za
zapis logičnih formul. Če bi uporabili zapis s substitucijo, bi
formulo označili samo s $\phi$ namesto s $\phi(x)$ in zamenjavo s
$\xsubst{\phi}{x \subto e}$ namesto s $\phi(e)$. Zakaj je ta bolj
priročen zapis hkrati manj pravilen? V formalni logiki strogo ločimo
med \emph{simbolnim zapisom} matematičnega pojma, ki je zaporedje
znakov na papirju, in njegovim \emph{pomenom}, ki je matematična
abstrakcija. Substitucija $\xsubst{\phi}{x \subto e}$ nam pove, kako
niz znakov $\phi$ predelamo v novi niz znakov, torej deluje na novoju
simbolnega zapisa. Ko pišemo $\phi(x)$ pa si že predstavljamo, da je
$\phi$ matematična funkcija, ki deluje na argumentu $x$. S tem nastopi
zmešnjava med simbolnim zapisom in pomenom. Dokler se zmešnjave
zavedamo, je vse v redu.

\subsection{Univerzalni kvantifikator}
\label{sec:univerzalni-kvantifikator}

Univerzalna kvantifikacija $\all{x \in S} \phi$ se prebere ">Za vse $x$
iz $S$ velja $\phi$."< Pravili sklepanja sta
%
\begin{mathpar}
  \inferrule
  {\infer*{\phi(x)}{[x \in S]}}{\all{x \in S} \phi(x)} \ \text{($x$ svež)}
  \and
  \inferrule{\all{x \in S} \phi(x) \\ e \in S}{\phi(e)}
\end{mathpar}
%
pri čemer je $x$ spremenljivka, $\phi(x)$ logična formula in $e$ poljuben izraz.

V besedilu dokažemo se pravilo vpeljave zapiše:
%
\begin{quote}
  \it
  %
  Dokazujemo $\all{x \in S} \phi(x)$:
  %
  \begin{itemize}
  \item[] Naj bo $x \in S$ poljuben.
  \item[] (Dokaz, da velja $\phi(x)$).
  \end{itemize}
  %
  Dokazali smo $\all{x \in S} \phi(x)$.
\end{quote}
%
Pravilo uporabe v besedilu ponavadi ni eksplicitno navedeno, če pa bi
ga že zapisali, bi šlo takole:
%
\begin{quote}
  \it
  %
  Dokazujemo, da velja $\phi(e)$:
  \begin{itemize}
  \item[] (Dokaz, da velja $\all{x \in S} \phi(x)$.)
  \item[] (Dokaz, da velja $e \in S$.)
  \end{itemize}
  %
  Torej velja $\phi(e)$.
\end{quote}


Ob pravilu vpeljave stoji stranski pogoj, da mora biti spremenljivka
$x$ ">sveža"<. To pomeni, da se $x$ ne sme pojavljati drugje v dokazu,
saj bi sicer lahko prišlo do zmešnjave med vezanimi in prostimi
spremenljivkami. V besedilu se dejstvo, da je $x$ svež izraža z
besedico ">poljuben"< ali ">katerikoli"<. Primer, kako gredo stvari
narobe, če ne pazimo in pomešamo spremenljivke:

\begin{izrek}[z napako v dokazu]
  Če je $x$ večji od~$42$, so vsa realna števila večja od~$23$.
\end{izrek}

\begin{dokaz}
  Denimo, da bi nekoliko nerodno zapisali izrek simbolno takole:
  %
  \begin{equation*}
    x > 42 \lthen \all{x \in \RR} x > 23.
  \end{equation*}
  %
  To je sicer dovoljeno, saj se prosti $x$, ki stoji zunaj $\forall$
  ni ujel, ni pa preveč smotrno, ker smo na dobri poti, da bomo
  zunanji prosti $x$ in vezanega znotraj $\forall$ pomešali. Res, če
  ne upoštevamo pravila, da mora biti $x$ svež, dobimo tale nepravi
  ">dokaz"<:
  %
  \begin{equation*}
    \inferrule*
    {
      \inferrule*
      {\inferrule*
        {[x > 42] \\ 42 > 23}
        {x > 23}
      }
      {\all{x \in \RR} x > 23}
    }
    {x > 42 \lthen \all{x \in \RR} x > 23}
  \end{equation*}
  %
  Pri pravilu za vpeljavo $\forall$ smo uporabili spremenljivko $x$,
  ki pa je že nastopala v začasni hipotezi $x > 42$. Z besedilom bi se
  isti dokaz glasil takole:
  %
  \begin{quote}
    ">Dokazujemo $x > 42 \lthen \all{x \in \RR} x > 23$. Predpostavimo,
    da velja $x > 42$ in dokažimo $\all{x \in \RR} x > 23$. Naj bo $x
    \in \RR$. Po predpostavki je $x > 42$ in ker je $42 > 23$, od tod
    sledi $x > 3$."<
  \end{quote}
  %
  Če bi izrek zapisali bolje kot $x > 42 \lthen \all{y \in \RR} y >
    23$, težav ne bi bilo, saj bi se prejšnji dokaz ">zataknil"<:
  %
  \begin{quote}
    ">Dokazujemo $x > 42 \lthen \all{y \in \RR} y > 23$. Predpostavimo,
    da velja $x > 42$ in dokažimo $\all{y \in \RR} y > 23$. Naj bo $y
    \in \RR$. (Kaj zdaj? Lahko sicer dokažemo $x > 23$, a zares bi
    morali dokazati $y > 23$, kar ne gre.)"<
  \end{quote}
\end{dokaz}

Pogoj, da mora biti spremenljivka $x$ v pravilu za vpeljavo ">sveža"<,
se v praksi kaže v tem, da pri uvajanju nove spremenljivke izberemo
zanjo novo ime, ki se še ni pojavilo v dokazu.


\subsection{Eksistenčni kvantifikator}
\label{sec:eksistencni-kvantifikator}

Eksistenčna kvantifikacija $\some{x \in S} \phi$ se prebere ">obstaja
$x$ iz $S$, za katerega velja $\phi$"< ali ">za neki $x$ iz $S$ velja
$\phi$."< Pravili sklepanja za eksistenčni kvantifikator se glasita
%
\begin{mathpar}
  \inferrule
  {\phi(e) \\ e \in S}
  {\some{x \in S} \phi(x)}
  \and
  \inferrule
  {\some{x \in S} \phi(x)
    \\
    \infer*{\psi}{[x \in S \land \phi(x)]}}
  {\psi}\ \text{($x$ svež)}
\end{mathpar}
%
kjer je $e$ poljuben izraz in $x$ spremenljivka. Pri tem mora biti $x$
v pravilu uporabe svež. V besedilu pravilo vpeljave uporabimo takole:
%
\begin{quote}
  \it
  %
  Dokazujemo $\some{x \in S} \phi(x)$:
  %
  \begin{enumerate}
  \item (Skonstruiramo element $e \in S$.)
  \item (Dokažemo, da velja $\phi(e)$.)
  \end{enumerate}
  %
  Dokazali smo $\some{x \in S} \phi(x)$.
\end{quote}
%
Pravilo uporabe pa se v besedilu izraža takole:
%
\begin{quote}
  \it
  %
  Dokazujemo $\psi$:
  %
  \begin{enumerate}
  \item (Dokaz izjave $\some{x \in S} \phi(x)$.)
  \item Predpostavimo, da za $x \in S$ velja $\phi(x)$:
    %
    \begin{itemize}
    \item[] (Dokaz izjave $\psi$.)
    \end{itemize}
  \end{enumerate}
  %
  Dokazali smo $\psi$.
\end{quote}

\subsubsection{Enolični obstoj}
\label{sec:enolicni-obstoj}

Poleg običajnega eksistenčnega kvantifikatorja $\exists$ poznamo tudi
\emph{enolični} eksistenčni kvantifikator $\exists!$. Izjavo
$\exactlyone{x}{S}{\phi}$ preberemo ">obstaja natanko en $x \in S$, za
katerega velja $\phi(x)$"<.

Enolični eksistenčni kvantifikator ni osnovni logični operator, ampak
je $\exactlyone{x}{S}{\phi}$ le okrajšava za
%
\begin{equation}
  \label{eq:uniqe-exists}
  \some{x \in S} \phi(x) \land (\all{y \in S} \phi(y) \lthen x = y).
\end{equation}
%
Z besedami preberemo to izjavo takole: ">obstaja $x$ iz $S$, za
katerega velja $\phi(x)$ in za vsak $y \in S$ za katerega velja
$\phi(y)$ sledi $x = y$"<. To je samo zapleten način, kako povedati,
da obstaja natanko en element množice~$S$, ki zadošča pogoju $\phi$.

Pravilo sklepanja za vpeljavo enoličnega obstoja izpeljemo
iz~\eqref{eq:uniqe-exists}:
%
\begin{equation*}
  \inferrule{
    e \in S
    \\
    \phi(e)
    \\
    \infer*{y = e}{y \in S \land \phi(y)}
  }
  {\exactlyone{x}{S}{\phi}}
\end{equation*}
%
V besedilu dokažemo enolični obstoj takole:
%
\begin{quote}
  \it
  %
  Dokazujemo, da obstaja natanko en $x \in S$, za katerega velja
  $\phi(x)$:
  %
  \begin{enumerate}
  \item Obstoj: (Konstrukcija elementa $e \in S$ in dokaz, da velja $\phi(x)$.)
  \item Enoličnost: denimo da za $y \in S$ velja $\phi(y)$:
    %
    \begin{itemize}
    \item[] (Dokaz, da je $e = y$).
    \end{itemize}
  \end{enumerate}
  %
  Dokazali smo $\exactlyone{x \in S} \phi(x)$.
\end{quote}

Če dokažemo enolični obstoj $\exactlyone{x \in S} \phi(x)$, lahko
vpeljemo novo konstanto $c$, ki označuje tisti element iz $S$, ki
zadošča pogoju~$\phi$, pri čemer moramo seveda paziti, da znaka $c$
nismo že prej uporabili za kak drug pomen. Nova konstanta~$c$ je
opredeljena s praviloma
%
\begin{mathpar}
  \inferrule{ }{\phi(c)}
  \and
  \inferrule{
    y \in S
    \\
    \phi(y)
  }
  {y = c}
\end{mathpar}
%
Če v formuli $\phi$ poleg spremenljivke $x$ nastopajo še druge proste
spremenljivke, denimo $y_1, \ldots, y_n$, potem je nova konstanta~$c$
v resnici \emph{funkcija} parametrov $y_1, \ldots, y_n$.

\subsection{Enakost in reševanje enačb}
\label{sec:enakost}

Enakost $=$ je osnovna relacija, ki zadošča naslednjim aksiomom in
pravilom sklepanja:
%
\begin{mathpar}
  \inferrule{ }{a = a}
  \and
  \inferrule{a = b}{b = a}
  \and
  \inferrule{a = b \\ b = c}{a = c}
  \and
  \inferrule{\phi(a) \\ a = b}{\phi(b)}
\end{mathpar}
%
Po vrsti so so pravilo \emph{refleksivnosti}, \emph{simetrije},
\emph{tranzitivnosti} in \emph{zamenjave}. Zaenkrat enakosti ne bomo
posvečali posebne pozornosti, saj jo v praksi študenti dobro
obvladajo.

V osnovni is srednji šoli se učimo pravil za reševanje enačb: enačbi
smemo na obeh straneh prišteti ali odšteti poljuben izraz, pomnožiti
ali deliti smemo s poljubnim \emph{neničelnim} izrazom, ipd. Od kod
izhajajo ta pravila? Kaj sploh pomeni, da smo enačbo ">rešili"<? Ko
rešimo kvadratno enačbo
%
\begin{equation*}
  x^2 - 5 x + 6 = 0
\end{equation*}
%
običajno zapišemo rešitev takole:
%
\begin{equation*}
  x_1 = 2, \quad x_2 = 3.
\end{equation*}
%
Kako naj to razumemo iz stališča matematične logike? Treba je
pojasniti dvoje: kaj pomenita $x_1$ in $x_2$, saj v prvotni enačbi
nastopa spremenljivka $x$, ter kako naj razumemo vejico med izjavama
$x_1 = 2$ in $x_2 = 3$. Z indeksoma $1$ in $2$ štejemo rešitve enačbe
in sta v resnici nepotrebna,\footnote{Kako pa bi zapisali rešitve
  enačbe $x_1^2 - 5 x_1 + 6 x = 0$?} na kar kaže tudi dejstvo, da
pišemo $x = \ldots$ in ne $x_1 = \ldots$, kadar je rešitev ena sama.
Torej bi lahko rešitev zapisali kot
%
\begin{equation*}
  x = 2, \quad x = 3.
\end{equation*}
%
Sedaj pa je tudi jasno, da bi namesto vejice morala stati disjunkcija,
se pravi
%
\begin{equation*}
  x = 2 \lor x = 3.
\end{equation*}
%
Začetna enačba in tako zapisana rešitev sta logično ekvivalentni:
%
\begin{equation*}
  x^2 - 5 x + 6 = 0 \iff
  x = 2 \lor x = 3.
\end{equation*}
%
Povzemimo: reševanje enačbe je postopek, s katerim dano enačbo $f(x) =
g(x)$ prevedemo v njen \emph{logično ekvivalentno} obliko $x = a_1
\lor x = a_2 \lor \cdots \lor x = a_n$, iz katere so neposredno razvidne
rešitve enačbe.

Pravila za reševanje enačb torej niso nič drugega kot recepti, s
pomočjo katerih enačbo predelamo v njen \emph{ekvivalentno} obliko, ki
je korak bližje končni obliki, v kateri bi radi zapisali rešitev. To
pojasnjuje srednješolska pravila za reševanje enačb. Na primer, za
realna števila $a, b, c \in \RR$ vedno velja
%
\begin{equation*}
  a = b \lthen c \cdot a = c \cdot b,
\end{equation*}
%
medtem ko obratna implikacija
%
\begin{equation*}
  c \cdot a = c \cdot b \lthen a = b
\end{equation*}
%
za splošne $a$ in $b$ velja le v primeru, ko je $c \neq 0$. Ker pri
reševanju enačb potrebujemo implikacijo v obe smeri, srednješolce
učimo, da smejo enačbo množiti samo z od nič različnimi števili.

\begin{naloga}
  Kako bi srednješolcem pojasnil, od kod izvira pravilo za množenje
  enačbe z neničelnim številom?
\end{naloga}

\begin{naloga}
  Enačbo $f(x) = g(x)$ smo ">rešili"< z zaporedjem korakov
  %
  \begin{align*}
    f(x) = g(x) &\liff \\
    f_1(x) = g_1(x) &\liff \\
    \vdots & \\
    f_k(x) = g_k(x) &\lthen \\
    f_{k+1}(x) = g_{k+1}(x) &\liff \\
    \vdots & \\
    x = a_1 \lor \cdots \lor x = a_n
  \end{align*}
  %
  kjer smo v $k$-tem koraku namesto ekvivalence pomotoma naredili
  implikacijo. Smo s tem dobili preveč ali premalo rešitev prvotne
  enačbe?
\end{naloga}



%%% Local Variables: 
%%% mode: latex
%%% TeX-master: "lmn"
%%% End: 

\input{dokazovanje.tex}
\input{konstrukcije.tex}
\input{preslikave.tex}
\input{relacije.tex}
\chapter{Strukture}


Informacija, ki jo posamična množica podaja, je zgolj, katere elemente vsebuje. Izkušnje hitro pokažejo, da ta informacija ni najbolje naravnana za matematično delo. Po eni strani je del te informacije pogosto odveč: tipično si lahko z neko množico pomagamo enako, če njene elemente preimenujemo, tj.~če obravnavamo izomorfno množico. Po drugi strani pa je te informacije premalo: ni dovolj, da vemo, katere elemente imamo na voljo, želimo vedeti tudi, kaj lahko s temi elementi počnemo. Podatek o tem imenujemo \df{struktura} množice.

Vzemimo za primer množico realnih števil~$\RR$. Njene elemente lahko poljubno seštevamo, odštevamo in množimo, tj.~izvajamo določene operacije na njih (seveda imamo še cel kup drugih operacij, vključno z delnimi, kot so deljenje, potenciranje, logaritmiranje\ldots). Strukturo, ki je dana z operacijami, imenujemo \df{algebrska} (ali \df{algebrajska} ali \df{algebraična}).

Množico lahko opremimo tudi z raznimi relacijami, tipično z relacijami urejenosti. Na primer, na $\RR$ imamo relaciji $\leq$ in $<$. To imenujemo \df{struktura urejenosti} (ali \df{urejenostna struktura}).

Realna števila si lahko predstavljamo kot točke na številski premici. Vidimo, da lahko potem računamo razdaljo med njimi. Pravimo, da realna števila tvorijo \df{metrični prostor} oziroma da imajo realna števila \df{metrično strukturo}.

Za realne intervale tudi znamo povedati, kdaj so odprti oz.~zaprti. Kadar imamo pojem odprtosti oz.~zaprtosti, to imenujemo \df{topološka struktura}. Prav tako znamo povedati dolžino intervalov. Kadar imamo pojem velikosti podmnožic, to imenujemo \df{merska struktura}.

Te in še nadaljnje strukture boste podrobneje spoznavali pri raznih matematičnih predmetih, v tej knjigi pa se bomo osredotočili zgolj na nekatere osnovne algebrske in urejenostne strukture.

Tipično velja: več kot imamo strukture na neki množici, bolj uporabna je (še zlasti, kadar se strukture med sabo prepletajo --- na primer, dejstvo, da je seštevanje na $\RR$ monotono, povezuje algebrsko in urejenostno strukturo na $\RR$). Ker imajo realna števila tako bogato strukturo, ni presenetljivo, da jih kar naprej uporabljamo. Za primerjavo: množico vseh permutacij $n$ elementov, ki se imenuje simetrična grupa in označi z $S_n$, uporabljate redkeje (je pa še vedno uporabna, saj premore nekaj operacij --- permutacije lahko sklapljamo in obračamo).

Množico, opremljeno z neko strukturo, imenujemo \df{strukturirana množica}. V tem kontekstu golo množico (brez njene dodatne strukture) imenujemo \df{nosilna množica} (te strukture).

Proučevanje struktur je ena temeljnih matematičnih dejavnosti. Na primer, pri predmetu Algebra spoznavate algebrske strukture, pri Topologiji topološke strukture, pri Analizi metrične in gladke strukture itd.

Za proučevanje strukture pa ne zadostuje opazovati zgolj množic, opremljenih s to strukturo, pač pa tudi preslikave med njimi, ki to strukturo na smiseln način ohranjajo. Tovrstnim preslikavam rečemo \df{homomorfizmi}. Kaj točno to pomeni, bomo spoznali pri konkretnih strukturah v nadaljevanju tega poglavja.

\note{nekje (ne nujno tu) debata, kako strukturirano množico podamo preko njene karakterizacije --- potrebna obstoj in enoličnost do izomorfizma}


\section{Algebrske strukture}

Kot rečeno, algebrska struktura je struktura, dana z operacijami. Operacije, na katere ste navajeni, imajo \df{mestnost}, tj.~koliko podatkov (ki jih imenujemo \df{argumenti} ali \df{operandi}) sprejmejo, da vrnejo rezultat. Na primer, seštevanje vzame dva podatka (seštevanca ali sumanda), ki ju zapišemo na levo in desno stran plusa, da dobimo rezultat (vsoto). Seštevanje je torej dvomestna operacija.

Odštevanje je prav tako dvomestna operacija --- od zmanjševanca odštejemo odštevanec in dobimo razliko. To je dvomestni minus, imamo pa tudi enomestni minus, ki vzame število in vrne njegovo nasprotno število. To sta dve različni operaciji in posledično imate zanju tudi dve različni tipki na kalkulatorju. Dvomestni minus je običajno označen kot $-$, enomestni pa kot ${}^+/_-$\;.

Še en primer enomestne operacije je faktoriela: za vsak $n \in \NN$ lahko naračunamo $n!$, kar je spet naravno število. Primer tromestne operacije je mešani produkt vektorjev v trorazsežnem prostoru: za poljubne tri vektorje je njihov mešani produkt število, katerega absolutna vrednost pove prostornino paralelepipeda, ki ga ti vektorji razpenjajo, predznak pa pove orientacijo tega paralelepipeda.

V splošnem je $n$-mestna operacija na množici $A$ dana kot preslikava $A^n \to A$, vsaj ko gre za operacijo, ki tako vzame kot vrne podatke iz množice $A$ --- taki operaciji rečemo \df{notranja}. Če to ne velja, je operacija \df{zunanja}. Vektorji lepo ponazorijo razliko. Seštevanje vektorjev v prostoru je preslikava $\RR^3 \times \RR^3 \to \RR^3$, torej dvomestna notranja operacija. Množenje vektorjev s skalarji $\RR \times \RR^3 \to \RR^3$ je dvomestna zunanja operacija, kjer enega od argumentov vzamemo iz neke druge množice (v tem primeru iz $\RR$). Skalarno množenje $\RR^3 \times \RR^3 \to \RR$ je prav tako dvomestna zunanja operacija, le da je tokrat rezultat iz druge množice. Prej omenjeni mešani produkt je tromestna zunanja operacija $\RR^3 \times \RR^3 \times \RR^3 \to \RR$.

V definiciji $n$-mestne operacije lahko vzamemo tudi $n = 0$. Ničmestna (notranja) operacija je torej preslikava $\one \to A$, se pravi izbira elementa iz $A$.

Obstajajo še splošnejše vrste operacij (npr.~takšne, ki so odvisne od neskončno argumentov), ampak v tej knjigi se ne bomo ukvarjali z njimi.


\subsection{Magme}

Operacije, s katerimi imamo najpogosteje opravka, so tipično dvomestne. Če želimo obravnavati takšne operacije na splošno, si definiramo strukturo, ki zajema zgolj eno tako operacijo.

\begin{definicija}
	\df{Magma} je množica, opremljena z dvomestno notranjo operacijo.
\end{definicija}

Strukturirano množico običajno zapišemo tako, da znotraj okroglih oklepajev najprej zapišemo simbol za nosilno množico, nato pa naštejemo vse sestavne dele strukture (ločene z vejicami). Če imamo strukturo magme na množici $A$ in dano operacijo označimo z $\oper$, tedaj to magmo zapišemo kot $(A, \oper)$. Če hočemo poudariti, da je $\oper$ dvomestna notranja operacija, lahko še natančneje zapišemo $(A,\ \oper\colon A \times A \to A)$.

Imejmo magmi $(A, \oper)$ in $(B, \soper)$. Za preslikavo $f\colon A \to B$ rečemo, da je \df{homomorfizem magem}, kadar ohranja magemsko strukturo v naslednjem smislu: za vse $x, y \in A$ mora veljati
\[f(x \oper y) = f(x) \soper f(y).\]
Z drugimi besedami, vseeno mora biti, če najprej izvedemo magemsko operacijo in nato izvrednotimo preslikavo ali obratno.

Če imamo magmo $(A, \oper)$, lahko posamične elemente množice $A$ povezujemo z operacijo in na ta način generiramo nove. Na primer, iz $x \in A$ lahko sestavimo računske izraze $x \oper x$, $(x \oper x) \oper x$, $x \oper (x \oper x)$ itd. Če začnemo z večimi elementi, recimo $x, y, z \in A$, lahko dobimo bolj raznotere izraze, npr.~$(x \oper y) \oper z$, $z \oper ((y \oper x) \oper z)$ in tako naprej. Vsi ti izrazi so med sabo različni, njihove vrednosti pa so lahko bodisi enake bodisi različne. Na primer, v magmi $(\NN, +)$ so $2 + 5$, $5 + 2$, $4 + 3$ in $(1 + 2) + (2 + 2)$ različni izrazi, ki pa imajo iste vrednosti.

Magemske izraze smo pisali kot zaporedja znakov, ki so vključevala elemente nosilne množice, simbol za operacijo in oklepaje (slednji so pomembni, saj v splošni magmi operacija ni družilna). Primernejši način podajanja takih izrazov so pa pravzaprav dvojiška drevesa. Vsakemu magemskemu izrazu ustreza neprazno dvojiško drevo, katerega listi so opremljeni z oznakami za elemente nosilne množice.

\note{nekaj primerov magemskih izrazov, podanih tako z dvojiškim drevesom kot z oklepajnim nizom}

Namen teh slikic pa ni zgolj ličen način, kako podati računanje neke operacije, pač pa se zadaj skrivajo vsaj tri temeljne ideje, ki so zelo pomembne za algebrske strukture in ki si jih bomo za začetek ogledali na preprostem primeru magem. Te tri ideje so:
\begin{itemize}
	\item
		prosta struktura,
	\item
		homomorfizem kot preslikava, ki ohranja izraze,
	\item
		podajanje algebrske strukture z generatorji in relacijami.
\end{itemize}

Začnimo s pojmom proste strukture. Če imamo katerokoli množico $A$ (ki jo v tem kontekstu običajno imenujemo \df{baza}), jo lahko razširimo do magme na kanoničen način. Naj $T(A)$ označuje množico vseh magemskih izrazov, ki jih lahko dobimo iz elementov množice $A$, tj.~množico vseh nepraznih dvojiških dreves, katerih listi so opremljeni z elementi množice $A$. Množico $T(A)$ opremimo z naslednjo dvojiško operacijo: če imamo izraza $T_1$ in $T_2$, tvorimo drevo, ki sestoji iz korena, katerega levo poddrevo je $T_1$, desno pa $T_2$. Označimo to dobljeno drevo s $T_1 \tconc T_2$.

Vsak element množice $A$ lahko predstavimo z elementom množice $T(A)$: elementu $x \in A$ pripišemo drevo, ki vsebuje zgolj koren, ki je že kar list in je označen z $x$. Po domače povedano: vsaka vrednost je na trivialen način tudi izraz. To preslikavo $\eta_A\colon A \to T(A)$ imenujemo \df{vložitev baze} oz.~\df{vložitev generatorjev}. Izraz `vložitev' je primeren, saj je ta preslikava očitno injektivna --- izvorni element lahko preberemo z edinega lista v njegovi sliki.

Množico $T(A)$ skupaj z dano operacijo imenujemo \df{prosta magma} nad množico $A$ (razlog za to poimenovanje bo postal jasen kasneje, ko si bomo ogledali podajanje algebrske strukture z generatorji in relacijami). Označimo jo z $F(A) \dfeq (T(A), \tconc)$. Ker lahko $A$ vložimo v $T(A)$, smo v tem smislu dejansko razširili poljubno množico do magme.

Kaj pa se zgodi, če je množica $A$ že opremljena s kakšno magemsko operacijo $\oper$? Tedaj imamo homomorfizem magem $a\colon T(A) \to A$, ki vsakemu računskemu izrazu priredi njegovo vrednost v $A$. Na primer, v primeru magme $(\NN, +)$ slikamo med drugim $4 + ((3 + 1) + ((5 + 0) + 2)) \mapsto 15$. Premisli, da je $a$ v splošnem res homomorfizem magem!

Preslikava $a$ je pomembna, ker zajema vso informacijo o algebrski strukturi na $A$. Z drugimi besedami, enakovredno je podati strukturo $(A, \oper)$ oziroma preslikavo $a$. Če imamo $A$ in $\oper$, lahko podamo $a$ kot zgoraj. Obratno, če imamo $a$, tedaj je $A$ njena kodomena, magemsko operacijo pa rekonstruiramo kot $x \oper y = a\big(\eta_A(x) \tconc \eta_A(y)\big)$. Poanta je sledeča: $x \oper y$ je prav tako računski izraz, tako da lahko z $a$ dobimo njegovo vrednost.

\subsection{Polgrupe, monoidi, grupe}

\subsection{Polkolobarji}
\subsection{Kolobarji}
\subsection{Obsegi}
\section{Strukture urejenosti}
\subsection{Mreže}
\subsection{Boolove mreže}
\section{Kategorije}


%%% Local Variables:
%%% mode: latex
%%% TeX-master: "ucbenik-lmn"
%%% End:

\input{stevilske-mnozice.tex}
\input{indukcija.tex}
\input{kumulativna-hierarhija.tex}
\input{kardinalna-stevila.tex}
\input{ordinalna-stevila.tex}
\Closesolutionfile{resitve}
\chapter{Rešitve vaj}
\begin{Resitev}{2.1}
Množica~$A$ ima kvečjemu en element, tj.~množica~$A$ je bodisi prazna bodisi enojec. Tudi: množica~$A$ je podmnožica kakega enojca oz.~edina preslikava $A \to \one$ je injektivna.
\end{Resitev}

\appendix
\input{razlaga-makrojev.tex}
\else
\fi

\end{document}

%%% Local Variables:
%%% mode: latex
%%% TeX-master: "ucbenik-lmn"
%%% End:
